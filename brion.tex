\documentclass{article}

\usepackage{lyon-style}

\title{Possibly non-affine algebraic groups}
\author{Michel Brion}
\date{June 9--13, 2014}

\begin{document}
\maketitle





Our topic is the structure of algebraic groups (possibly non-affine) over 
a field (possibly non-perfect). 

\begin{theorem}[Chevalley, Barsotti, Rosenlicht]
Let $G$ be a connected algebraic group over a perfect field. Then there exists 
a largest connected linear subgroup $L\subset G$. Moreover, $L$ is normal 
in $G$ and $G/L$ is an abelian variety. 
\end{theorem}

In other words, every algebraic group is the extension of an abelian variety by 
an affine algebraic group. 

\begin{theorem}[Rosenlicht, Demazure-Gabriel]\label{thm:B}
Let $G$ be a group scheme of finite type over a field $k$. Then $G$ has a 
largest linear quotient $L'$, i.e.~a smallest normal subgroup scheme $H$ such 
that $G/H$ is linear. Moreover, $H$ is smooth, connected, commutative, and 
$\sO(H) = \Gamma(H,\sO_H) = k$. 
\end{theorem}

These theorems are ``dual'' in some informal sense. Note that the assumptions 
of Theorem \ref{thm:B} are much weaker. The group $H$ in Theorem \ref{thm:B} 
is an abelian variety when $k$ is perfect. 
A good reference is \cite{m14}. 

\begin{definition}
A \emph{$k$-group scheme} is a scheme $G$ over $k$, equipped with morphisms 
$m:G\times G\to G$ (multiplication), $i:G\to G$ (inverse) and 
$e\in G(k)$ (identity) such that for all $k$-schemes $S$, these give 
$G(S)$ the structure of a group. An \emph{algebraic group} is a smooth group 
scheme of finite type. 
\end{definition}

\emph{Warning}: this definition is less restrictive than Gille's. 

If $k=\bar k$, then this is the same notion as in \cite{b91} or \cite{s09}. 
They identify $G$ with the set $G(k)$ of $k$-points. 
Every group scheme is separated, i.e.\ the diagonal $\Delta_G\subset G\times G$ 
is closed. 

Let $K/k$ be a field extension, $G$ a $k$-group scheme. Then 
$G_K=G\times_{\spectrum(k)} \spectrum(K)$ is the base-extension of $G$ to 
$K$. The functor $(-)_K$ commutes with products, so $G_K$ is a $K$-group 
scheme. 

Our main examples of algebraic groups are the \emph{additive group} 
$\dG_\additive$, the \emph{multiplicative group} $\dG_\mult$, and the 
\emph{general linear group} $\generallinear_n$. These are easily defined 
via their functors of points. A \emph{linear group scheme} is a (closed) 
subgroup scheme of some $\generallinear_n$. It turns out that for 
$k$-groups, linear $\Leftrightarrow$ affine of finite type. We can regard 
$\generallinear_n$ as a closed subgroup of $\speciallinear_{n+1}$ in the 
usual way. 

Recall that closed subgroup schemes of $\dG_\additive$ correspond to 
(certain) 
ideals $I\subset \sO(\dG_\additive) = k[t]$. Such ideals will be 
generated by a polynomial $f\in k[t]$. Such a polynomial needs to 
satisfy $f(x+y) \in \langle f(x),f(y)\rangle \subset k[x,y]$, 
$f(0)=0$, and $f(-t) \in \langle f\rangle$. Some elementary manipulations 
show that if $k$ has characteristic zero, such an $f$ must be of the 
form $a t$ for $a\in k^\times$. In characteristic $p>0$, such an $f$ can 
be any of the polynomials $\sum_{i\geqslant 1} a_i t^{p^i}$. These are 
called \emph{$p$-polynomials}. For a $p$-polynomial $f$, the subgroup 
scheme $V(I)\subset \dG_\additive$ will be nonreduced. For each 
$n\geqslant 1$, we get a scheme $\boldsymbol\mu_n\subset \dG_\mult$ defined 
by 
\[
  \boldsymbol\mu_n(A) = \{a\in A^\times:a^n=1\} .
\]

Our final example of group schemes are \emph{elliptic curves}. For us, 
an elliptic curve over $k$ is a smooth projective curve $E$ of genus $1$ 
with a distinguished point $0\in E(k)$. Using Riemann-Roch, such a curve 
embeds into $\dP_k^2$. If $6\in k^\times$, then 
$E$ has affine form $y^2=x^3+a x+b$, the standard equation for an elliptic 
curve. One defines a group law on $E$ in the usual way. 

For a group scheme $G$, locally of finite type over $k$, the following 
conditions are equivalent:
\begin{enumerate}
  \item $G$ is smooth
  \item $G$ is geometrically reduced
  \item $\sO_{G,e}\otimes_k \bar k$ is reduced
\end{enumerate}
If $k$ has characteristic zero, then \emph{all} $k$-groups locally of 
finite type are smooth. In general, we have a largest reduced 
subscheme $G_\reduced\subset G$; this is a group scheme if $k$ is 
perfect. 





\section{Homogeneous spaces and quotients}

The reference here is \cite[VI$_\mathrm{A}$ \S 3]{sga3}. Let $G$ be a $k$-group 
scheme, locally of finite type, and $H\subset G$ a subgroup scheme. Then there is 
a morphism of schemes $q:G\to G/H$ such that $q$ is $H$-invariant (i.e.~if 
$m,p:G\times H\to G$ are the multiplication and projection morphisms, we have 
$q m = q p$) and any $H$-invariant morphism $G\to X$ factors uniquely 
through $q$. Moreover, $q$ is faithfully flat, and the diagram 
\[\xymatrix{
  G\times H \ar[r]^-m \ar[d]^-p  
    & G \ar[d]^-q \\
  G \ar[r]^-q 
    & G/H 
}\]
is cartesian. It follows that $q:G\to G/H$ is an $H$-torsor (for the 
fppf topology). The map $q$ is locally of finite type. 

If $H\subset G$ is normal, then the quotient $G/H$ has the structure of 
a group scheme, uniquely characterized by the property that 
$q:G\to G/H$ is a morphism of group schemes. 

\begin{example}
Let $G=\dG_\mult$ and $H=\boldsymbol\mu_n$. Then $G/H=\dG_\mult$ via the 
$n$-th power map $x\mapsto x^n$. But the map $G(k) \to (G/H)(k)$ is usually 
not surjective. But the map $G(\bar k) \to (G/H)(\bar k)$ is surjective. 
\end{example}

If we worked over an arbitrary base ring $k$, we would need to pass to 
fppf extensions in order for $G\to G/H$ to be pointwise surjective. In 
other words, for $s\in (G/H)(A)$, there is an fppf cover $A\to B$ and 
$\widetilde s\in G(B)$ such that $\widetilde s|_A = s$. 

If $G$ is smooth, so is $G/H$. 

if $G$ and $H$ are affine, $\sO(G/H)=\sO(G)^H$. (Here 
$\sO(G)^H$ consists of $H$-invariant morphisms $G\to \dA^1$.)





\section{Lie algebras}

Let $G$ be a $k$-group scheme. Let $k[\tau]=k[t]/(t^2)$. Consider the base 
change $G_{k[\tau]}$ as a $k$-group scheme. The maps $k\rightleftarrows k[\tau]$ 
induce morphisms $G_{k[\tau]} \leftrightarrows G$. It turns out that 
$G_{k[\tau]} = G\ltimes \ker(\pi)$, where $\pi:G_{k[\tau]} \to G$ is the canonical 
projection. Now $\pi:G_{k[\tau]}\to G$ is the ``tangent bundle'' of 
$G$, so its fiber $\pi^{-1}(g)$ for $g\in G(k)$ is the tangent space 
$T_g G$. In particular, $\pi^{-1}(0) = \lie G$. Since $\lie G$ lives in the 
semidirect product $G_{k[\tau]} = G\ltimes \ker(\pi)$, it has an action of 
$G$ by conjugation. If we put $\fg=\lie G$, this gives a homomorphism 
$\adjoint:G\to \generallinear(\fg)$. Differentiating this gives a representation 
(also called the adjoint representation) 
$\adjoint:\fg \to \lie(\generallinear\fg) = \End(\fg)$. Define 
$[x,y] = \adjoint(x) y$; this is the usual bracket on $\fg$. 


\section{Neutral component}

Let $G$ be a $k$-group scheme, locally of finite type. Let $G^\circ$ be the 
connected component of $e$ in $G$. Then $G^\circ$ is a normal subgroup scheme 
of $G$, and the quotient $G/G^\circ$ is \'etale (smooth of dimension $0$). The 
group $G^\circ$ is of finite type, and is geometrically irreducible. 

\begin{example}[automorphism group schemes]
Let $X$ be a $k$-scheme. Define a functor $\automorphism_X$ on $k$-schemes 
by $\automorphism_X(S) = \automorphism_{S\textnormal{-}\mathsf{sch}}(X_S)$. 
Grothendieck proved that if $X$ is proper, then $\automorphism_X$ is represented by a 
$k$-group scheme, locally of finite type. One has 
$\lie(\automorphism_X) = \operatorname{Der}_k(\sO_X) = \h^0(X,T_X)$. Suppose 
$X$ is projective. Then to $f\in \automorphism_X(S)$, we can associate its 
graph $\Gamma_f\subset (X\times X)_S$. So the automorphisms of $X$ are 
parameterized by an open subscheme of the Hilbert scheme of $X$. The Hilbert 
scheme is projective, so $\automorphism_X$ is quasi-projective. 
\end{example}

When $X$ is proper, then there is an alternative construction of 
Matsumura-Oort using a general representability criterion. 
For the simplest types of projective varieties (projective space) everything 
can be written down. One has $\automorphism_{\dP(V)} = \generallinear(V)$. 
If $X$ is a smooth projective curve of genus $0$, then we can write 
$X\simeq V(q)\subset \dP^2$ for a quadratic form $q$. One has 
$\automorphism_X=\operatorname{PO}(q)$; this is a $k$-form of 
$\projectivegenerallinear_2$. 

For $X$ a nice curve of genus $1$, then $\automorphism_X^\circ$ is an 
elliptic curve, and the component group $\automorphism_X/\automorphism_X^\circ$ 
is finite. If $X$ is a nice curve of genus $g\geqslant 2$, then 
$\automorphism_X$ is finite. 

For smooth projective surfaces $X$, the quotient group 
$\automorphism_X/\automorphism_X^\circ$ may be infinite. For example, if 
$X=E\times E$ for $E$ an elliptic curve, then $\automorphism_X^\circ=E\times E$. 
But $\pi_0(\automorphism_X)\supset \generallinear_2(\dZ)$. 

[Here I write $\pi_0 G$ for the quotient $G/G^\circ$.]

It is currently an open problem whether $\pi_0(\automorphism_X)$ is 
finitely generated for $X$ a smooth projective variety. 

Even if $X$ is a smooth projective surface, the group $\automorphism_X$ may 
not be reduced. A good example is an Igusa surface in characteristic $2$. 
Let $E,F$ be elliptic curves. Let $X=(E\times F)/\sigma$, where 
$\sigma$ is the involution defined by 
$\sigma(x,y) = (-x,y+y_0)$, where $y_0\in F(k)$ is chosen to have order $2$. 
In characteristic $2$, $T_X$ is trivial! 
Now $\lie(\automorphism_X)=k^2$. Let 
$\pi:E\times F\to X$ be the quotient map. This is \'etale, so 
$\pi^\ast T_X = (T_{E\times F})^\sigma = (\sO_{E\times F}^2)^\sigma = \sO_{E\times F}^2$, 
in characteristic $2$. 

An open question: which group schemes occur as $\automorphism_X^\circ$ for 
$X$ proper, normal? Over perfect fields, all connected algebraic groups occur. 





\bibliographystyle{alpha}
\bibliography{lyon-sources}

\end{document}
