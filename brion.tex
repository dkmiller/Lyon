% !TEX root = main.tex

%\documentclass{article}
%
%\usepackage{lyon-style}
%
%\title{Possibly non-affine algebraic groups}
%\author{Michel Brion}
%\date{June 9--13, 2014}
%
%\begin{document}
%\maketitle
%\tableofcontents
%



\section{Possibly non-affine algebraic groups}
\thanksauthor{Michel Brion}



Our topic is the structure of algebraic groups (possibly non-affine) over 
a field (possibly non-perfect). 

\begin{theo}[Chevalley, Barsotti, Rosenlicht]\label{thm:A}
Let $G$ be a connected algebraic group over a perfect field. Then there exists 
a largest connected linear subgroup $L\subset G$. Moreover, $L$ is normal 
in $G$ and $G/L$ is an abelian variety. 
\end{theo}

In other words, every algebraic group is the extension of an abelian variety by 
an affine algebraic group. 

\begin{theo}[Rosenlicht, Demazure-Gabriel]\label{thm:B}
Let $G$ be a group scheme of finite type over a field $k$. Then $G$ has a 
largest linear quotient $L'$, i.e.~a smallest normal subgroup scheme $H$ such 
that $G/H$ is linear. Moreover, $H$ is smooth, connected, commutative, and 
$\sO(H) = \Gamma(H,\sO_H) = k$. 
\end{theo}

These theorems are ``dual'' in some informal sense. Note that the assumptions 
of Theorem \ref{thm:B} are much weaker. The group $H$ in Theorem \ref{thm:B} 
is an abelian variety when $k$ is perfect. 
A good reference is \cite{m14}. 

\begin{defi}
A \emph{$k$-group scheme} is a scheme $G$ over $k$, equipped with morphisms 
$m:G\times G\to G$ (multiplication), $i:G\to G$ (inverse) and 
$e\in G(k)$ (identity) such that for all $k$-schemes $S$, these give 
$G(S)$ the structure of a group. An \emph{algebraic group} is a smooth group 
scheme of finite type. 
\end{defi}

\emph{Warning}: this definition is less restrictive than Gille's. 

If $k=\bar k$, then this is the same notion as in \cite{b91} or \cite{s09}. 
They identify $G$ with the set $G(k)$ of $k$-points. 
Every group scheme is separated, i.e.\ the diagonal $\Delta_G\subset G\times G$ 
is closed. 

Let $K/k$ be a field extension, $G$ a $k$-group scheme. Then 
$G_K=G\times_{\spectrum(k)} \spectrum(K)$ is the base-extension of $G$ to 
$K$. The functor $(-)_K$ commutes with products, so $G_K$ is a $K$-group 
scheme. 

Our main examples of algebraic groups are the \emph{additive group} 
$\dG_\additive$, the \emph{multiplicative group} $\dG_\mult$, and the 
\emph{general linear group} $\generallinear_n$. These are easily defined 
via their functors of points. A \emph{linear group scheme} is a (closed) 
subgroup scheme of some $\generallinear_n$. It turns out that for 
$k$-groups, linear $\Leftrightarrow$ affine of finite type. We can regard 
$\generallinear_n$ as a closed subgroup of $\speciallinear_{n+1}$ in the 
usual way. 

Recall that closed subgroup schemes of $\dG_\additive$ correspond to 
(certain) 
ideals $I\subset \sO(\dG_\additive) = k[t]$. Such ideals will be 
generated by a polynomial $f\in k[t]$. Such a polynomial needs to 
satisfy $f(x+y) \in \langle f(x),f(y)\rangle \subset k[x,y]$, 
$f(0)=0$, and $f(-t) \in \langle f\rangle$. Some elementary manipulations 
show that if $k$ has characteristic zero, such an $f$ must be of the 
form $a t$ for $a\in k^\times$. In characteristic $p>0$, such an $f$ can 
be any of the polynomials $\sum_{i\geqslant 1} a_i t^{p^i}$. These are 
called \emph{$p$-polynomials}. For a $p$-polynomial $f$, the subgroup 
scheme $V(I)\subset \dG_\additive$ will be nonreduced. For each 
$n\geqslant 1$, we get a scheme $\boldsymbol\mu_n\subset \dG_\mult$ defined 
by 
\[
  \boldsymbol\mu_n(A) = \{a\in A^\times:a^n=1\} .
\]

Our final example of group schemes are \emph{elliptic curves}. For us, 
an elliptic curve over $k$ is a smooth projective curve $E$ of genus $1$ 
with a distinguished point $0\in E(k)$. Using Riemann-Roch, such a curve 
embeds into $\dP_k^2$. If $6\in k^\times$, then 
$E$ has affine form $y^2=x^3+a x+b$, the standard equation for an elliptic 
curve. One defines a group law on $E$ in the usual way. 

For a group scheme $G$, locally of finite type over $k$, the following 
conditions are equivalent:
\begin{enumerate}
  \item $G$ is smooth
  \item $G$ is geometrically reduced
  \item $\sO_{G,e}\otimes_k \bar k$ is reduced
\end{enumerate}
If $k$ has characteristic zero, then \emph{all} $k$-groups locally of 
finite type are smooth. In general, we have a largest reduced 
subscheme $G_\reduced\subset G$; this is a group scheme if $k$ is 
perfect. 





\subsection{Homogeneous spaces and quotients}

The reference here is \cite[VI$_\mathrm{A}$ \S 3]{sga3}. Let $G$ be a $k$-group 
scheme, locally of finite type, and $H\subset G$ a subgroup scheme. Then there is 
a morphism of schemes $q:G\to G/H$ such that $q$ is $H$-invariant (i.e.~if 
$m,p:G\times H\to G$ are the multiplication and projection morphisms, we have 
$q m = q p$) and any $H$-invariant morphism $G\to X$ factors uniquely 
through $q$. Moreover, $q$ is faithfully flat, and the diagram 
\[\xymatrix{
  G\times H \ar[r]^-m \ar[d]^-p  
    & G \ar[d]^-q \\
  G \ar[r]^-q 
    & G/H 
}\]
is cartesian. It follows that $q:G\to G/H$ is an $H$-torsor (for the 
fppf topology). The map $q$ is locally of finite type. 

If $H\subset G$ is normal, then the quotient $G/H$ has the structure of 
a group scheme, uniquely characterized by the property that 
$q:G\to G/H$ is a morphism of group schemes. 

\begin{enonce}{Example}
Let $G=\dG_\mult$ and $H=\boldsymbol\mu_n$. Then $G/H=\dG_\mult$ via the 
$n$-th power map $x\mapsto x^n$. But the map $G(k) \to (G/H)(k)$ is usually 
not surjective. But the map $G(\bar k) \to (G/H)(\bar k)$ is surjective. 
\end{enonce}

If we worked over an arbitrary base ring $k$, we would need to pass to 
fppf extensions in order for $G\to G/H$ to be pointwise surjective. In 
other words, for $s\in (G/H)(A)$, there is an fppf cover $A\to B$ and 
$\widetilde s\in G(B)$ such that $\widetilde s|_A = s$. 

If $G$ is smooth, so is $G/H$. 

if $G$ and $H$ are affine, $\sO(G/H)=\sO(G)^H$. (Here 
$\sO(G)^H$ consists of $H$-invariant morphisms $G\to \dA^1$.)





\subsection{Lie algebras}

Let $G$ be a $k$-group scheme. Let $k[\tau]=k[t]/(t^2)$. Consider the base 
change $G_{k[\tau]}$ as a $k$-group scheme. The maps $k\rightleftarrows k[\tau]$ 
induce morphisms $G_{k[\tau]} \leftrightarrows G$. It turns out that 
$G_{k[\tau]} = G\ltimes \ker(\pi)$, where $\pi:G_{k[\tau]} \to G$ is the canonical 
projection. Now $\pi:G_{k[\tau]}\to G$ is the ``tangent bundle'' of 
$G$, so its fiber $\pi^{-1}(g)$ for $g\in G(k)$ is the tangent space 
$T_g G$. In particular, $\pi^{-1}(0) = \lie G$. Since $\lie G$ lives in the 
semidirect product $G_{k[\tau]} = G\ltimes \ker(\pi)$, it has an action of 
$G$ by conjugation. If we put $\frakg=\lie G$, this gives a homomorphism 
$\adjoint:G\to \generallinear(\frakg)$. Differentiating this gives a representation 
(also called the adjoint representation) 
$\adjoint:\frakg \to \lie(\generallinear\frakg) = \End(\frakg)$. Define 
$[x,y] = \adjoint(x) y$; this is the usual bracket on $\frakg$. 


\subsection{Neutral component}

Let $G$ be a $k$-group scheme, locally of finite type. Let $G^\circ$ be the 
connected component of $e$ in $G$. Then $G^\circ$ is a normal subgroup scheme 
of $G$, and the quotient $G/G^\circ$ is \'etale (smooth of dimension $0$). The 
group $G^\circ$ is of finite type, and is geometrically irreducible. 

\begin{enonce}{Example}[automorphism group schemes]
Let $X$ be a $k$-scheme. Define a functor $\automorphism_X$ on $k$-schemes 
by $\automorphism_X(S) = \automorphism_{S\textnormal{-}\mathsf{sch}}(X_S)$. 
Grothendieck proved that if $X$ is proper, then $\automorphism_X$ is represented by a 
$k$-group scheme, locally of finite type. One has 
$\lie(\automorphism_X) = \operatorname{Der}_k(\sO_X) = \h^0(X,T_X)$. Suppose 
$X$ is projective. Then to $f\in \automorphism_X(S)$, we can associate its 
graph $\Gamma_f\subset (X\times X)_S$. So the automorphisms of $X$ are 
parameterized by an open subscheme of the Hilbert scheme of $X$. The Hilbert 
scheme is projective, so $\automorphism_X$ is quasi-projective. 
\end{enonce}

When $X$ is proper, then there is an alternative construction of 
Matsumura-Oort using a general representability criterion. 
For the simplest types of projective varieties (projective space) everything 
can be written down. One has $\automorphism_{\dP(V)} = \generallinear(V)$. 
If $X$ is a smooth projective curve of genus $0$, then we can write 
$X\simeq V(q)\subset \dP^2$ for a quadratic form $q$. One has 
$\automorphism_X=\operatorname{PO}(q)$; this is a $k$-form of 
$\projectivegenerallinear_2$. 

For $X$ a nice curve of genus $1$, then $\automorphism_X^\circ$ is an 
elliptic curve, and the component group $\automorphism_X/\automorphism_X^\circ$ 
is finite. If $X$ is a nice curve of genus $g\geqslant 2$, then 
$\automorphism_X$ is finite. 

For smooth projective surfaces $X$, the quotient group 
$\automorphism_X/\automorphism_X^\circ$ may be infinite. For example, if 
$X=E\times E$ for $E$ an elliptic curve, then $\automorphism_X^\circ=E\times E$. 
But $\pi_0(\automorphism_X)\supset \generallinear_2(\dZ)$. 

[Here I write $\pi_0 G$ for the quotient $G/G^\circ$.]

It is currently an open problem whether $\pi_0(\automorphism_X)$ is 
finitely generated for $X$ a smooth projective variety. 

Even if $X$ is a smooth projective surface, the group $\automorphism_X$ may 
not be reduced. A good example is an Igusa surface in characteristic $2$. 
Let $E,F$ be elliptic curves. Let $X=(E\times F)/\sigma$, where 
$\sigma$ is the involution defined by 
$\sigma(x,y) = (-x,y+y_0)$, where $y_0\in F(k)$ is chosen to have order $2$. 
In characteristic $2$, $T_X$ is trivial! 
Now $\lie(\automorphism_X)=k^2$. Let 
$\pi:E\times F\to X$ be the quotient map. This is \'etale, so 
$\pi^\ast T_X = (T_{E\times F})^\sigma = (\sO_{E\times F}^2)^\sigma = \sO_{E\times F}^2$, 
in characteristic $2$. 

An open question: which group schemes occur as $\automorphism_X^\circ$ for 
$X$ proper, normal? Over perfect fields, all connected algebraic groups occur. 

\begin{enonce}{Example}
Let $E$ be an elliptic curve with origin $0\in E(k)$. Let $L\to E$ be a line 
bundle. Then $L\smallsetminus \{\text{zero section}\}$ is a $\dG_\mult$-bundle 
on $E$. The map $\pi:L\smallsetminus0 \to E$ is in fact a (Zariski-locally 
trivial) $\dG_\mult$-torsor. This induces a class $[L]\in \h^1(E,\dG_\mult)$. 
We claim that if $\deg(L)=0$, then $G=L\smallsetminus 0$ has the unique structure 
of a commutative group such that $1 \to \dG_\mult \to G \xrightarrow\pi E \to 0$ 
is exact. This result is due to Serre. This is an example of the sequence produced 
by Theorem \ref{thm:A}. We would like to see how this fits into Theorem 
\ref{thm:B}. To do this, we need to compute $\sO(G)$. We compute 
\begin{align*}
  \sO(G) &= \h^0(G,\sO_G) \\
    &= \h^0(E,\pi_\ast \sO_G) \\
    &= \h^0\left(E,\bigoplus_{n\in \dZ} L^n\right) \\
    &= \bigoplus_{n\in \dZ} \h^0(E,L^n) .
\end{align*}
As an exercise, show that if $M\to E$ is a line bundle of degree $0$, then 
$\h^0(E,M)\ne 0$ if and only if $M\simeq \sO_E$. If $L$ has infinite order 
in $\operatorname{Pic} E$, then all $L^n\not\simeq M$, so $\sO(G)=k$. If 
$L$ has order $m$, then $\sO(G)=\simeq k[t^{\pm 1}]$, where 
$\deg t=m$. We have obtained a morphism $G\to \dG_\mult$. We claim that this 
fits into an exact sequence 
\[
  1 \to F \to G \to \dG_\mult \to 1 ,
\]
where $F$ is an elliptic curve. The map $\pi:G\to E$ induces an isogeny 
$F\twoheadrightarrow E$ of degree $m$. 
\end{enonce}





\subsection{Proof of \texorpdfstring{Theorem \ref{thm:B}}{Theorem B}}

\begin{lemm}
Let $G$ be a group scheme of finite type over $k$. Then the following are 
equivalent:
\begin{enumerate}
  \item $G$ is linear
  \item $G$ is affine 
  \item $G_{\bar k}$ is affine
  \item $(G_{\bar k})_\reduced$ is affine
\end{enumerate}
\end{lemm}
\begin{proof}
Clearly $1 \Rightarrow 2 \Rightarrow3\Rightarrow 4$. Moreover, 
$2\Rightarrow 1$ exactly as for algebraic groups (as in Borel's book). The 
non-trivial part is $4\Rightarrow 3$, which holds for any scheme of finite 
type (also noetherian). This is an exercise in Hartshorne. The implication 
$3\Rightarrow 2$ is a type of descent. Recall that a scheme $X$ is affine if 
and only if the functor $\h^0:\mathsf{qc}(X) \to \mathsf{Vect}_k$ is 
exact. But $\h^0(G_{\bar k},\sF) = \h^0(G,\sF)_{\bar k}$ (just use the 
\v Cech complex). This is exact, so we're done. 
\end{proof}

\begin{lemm}
Let $G$ be a linear group scheme, $\pi:X\to Y$ a $G$-torsor. Then the morphism 
$\pi$ is affine. 
\end{lemm}
\begin{proof}
We know that $\pi$ is faithfully flat, $G$ acts on $X$, $\pi$ is $G$-invariant, 
and the diagram 
\[\xymatrix{
  G\times X \ar[r]^-m \ar[d]^-p 
    & X \ar[d]^-\pi \\
  X \ar[r]^-\pi 
    & Y
}\]
is Cartesian (this is what it means for $X$ to be a $G$-torsor over $Y$). 
Let $V\subset Y$ be affine open, $U=\pi^{-1}(V)$. We need to prove that 
$U$ is also affine. We do this by proving that $\h^0(U,-)$ is exact. But 
this is $\h^0(V,\pi_\ast -)$. But $V$ is affine, so $\h^0(V,-)$ is 
exact. All that remains is to prove that $(\pi|_U)_\ast$ is exact. We have a 
cartesian diagram 
\[\xymatrix{
  G\times U \ar[r]^-m \ar[d]_p 
    & U \ar[d]^-{\pi|_U} \\ 
  U \ar[r]^-{\pi|_U} 
    & V .
}\]
One has $\pi_U^\ast \pi_{U\ast} \simeq p_\ast m^\ast$. Since 
$\pi_U^\ast$ is exact, it suffices to show that 
$\pi_U^\ast \pi_{U\ast} = p_\ast m^\ast$ is exact. But $m^\ast$ is exact by 
faithful flatness, and $p_\ast$ is exact because $p$ is affine. 
\end{proof}

\begin{prop}
Let $G$ be a group scheme of finite type, $H\subset G$ a normal subgroup 
scheme. Then $G$ is affine if and only if $H$ and $G/H$ are affine. 
\end{prop}
\begin{proof}
We may assume $k=\bar k$. 

$\Rightarrow$. Clearly closed subgroups $H\subset G$ are affine. The fact that 
$G/H$ is affine is harder -- see \cite{b91} for a proof that 
$G_\reduced/H_\reduced$ is affine. But we need to show that 
$(G/H)_\reduced$ is affine. There is a map 
$G_\reduced/H_\reduced \to (G/H)_\reduced$, which is a group homomorphism. 
It is a bijection on $\bar k$-points. Let $K$ be its (scheme-theoretic) 
kernel. The scheme $K$ is of finite type, and has a unique $k$-point. This 
implies that $K$ is a finite group scheme. In \cite{m08} for a proof of the 
fact that if $X$ is an affine group scheme, and $K$ is a finite group scheme acting 
on $X$ then $X/K$ is affine. 
\end{proof}

Equivalently, affine groups are stable under extensions. 


\subsubsection{Affinization}

Let $X$ be a $k$-scheme. Then there is a universal morphism from $X$ to an affine 
scheme -- namely $\varphi_X:X \to \spectrum \sO(X)$. One has 
\[
  \hom_{k\textnormal{-}\mathsf{sch}}(X,\spectrum A) = \hom_{k\textnormal{-}\mathsf{alg}}(A,\sO(X)) .
\]
\emph{Warning}: if $X$ is of finite type, then $\sO(X)$ may not be!

\begin{enonce}{Example}
Let $L\to E$ be a line bundle of infinite order as before. Let $M \to E$ be a 
line bundle of negative degree. Consider the vector bundle 
$X=L\oplus M$ over $E$. One has 
\[
  \sO(X) = \bigoplus_{m,n\geqslant 0} L^{-m} \otimes M^{-n} .
\]
[\ldots stuff I didn't understand\ldots]
\end{enonce}

Note that $\varphi_X$ commutes with products (since 
$\sO(X\times Y) = \sO(X)\otimes_k \sO(Y)$). Thus, if $G$ is a group scheme, then 
$\spectrum \sO(G)$ is an affine group scheme, and $\varphi_G$ is a homomorphism. 

\begin{theo}
If $G$ is of finite type, then $H=\ker(\varphi)$ is the smallest normal subgroup 
scheme of $G$ such that $G/H$ is affine. Moreover, $\sO(H)=k$ and 
$\sO(G)=\sO(G/H)$. In particular, $G/H=\spectrum \sO(G$ is finitely generated. 
\end{theo}
\begin{proof}
Let $H_1\subset G$ be normal such that $G/H_1$ is affine. Then 
$G\to G/H_1$ factors through $\varphi_X$ as in 
\[\xymatrix{
  G \ar[r]^-\varphi \ar[dr]_-q 
    & \spectrum \sO(G) \ar@{.>}[d] \\
  & G/H_1 .
}\]
Then $H=\varphi^{-1}(1) \subset H_1=q^{-1}(1)$. It remains to show that $G/H$ is 
affine. Write $\sO(G)=\bigcup V_i$, where each $V_i$ is a $G$-stable finite-dimensional 
subspace of $\sO(G)$. Then $H$ is the kernel of the right $G$-action on 
$\sO(G)$, i.e. 
\[
  H = \bigcap_i \ker(G \to \generallinear(V_i)) .
\]
By noetherianness, $H=\ker(G\to \generallinear(V_{i_0}))$ for some $i_0$, 
so snce $G/H \to \generallinear(V_{i_0})$ has trivial kernel, it is a 
closed immersion by a theorem proved by Gille. 

[\ldots didn't follow whole proof\ldots]
\end{proof}

\begin{theo}[Demazure-Gabriel]
Let $G$ be a $k$-group scheme of finite type such that $\sO(G)=k$. Then $G$ 
is smooth, connected, and every homomorphism $G\to H$ (the latter a connected 
group scheme) factors through the center of $H$. 
\end{theo}
\begin{proof}
Since $\sO(G_{\bar k}) = \sO(G)_{\bar k}$, we may assume $k=\bar k$. Note that 
$\pi_0 G = G/G^\circ$ is a finite group, and 
$\sO(G/G^\circ)\hookrightarrow \sO(G)=k$, so $G=G^\circ$. Also, 
$G/G_\reduced$ is finite, and $\sO(G/G_\reduced)=k$, so $G$ is reduced (hence 
smooth). We use a rigidity lemma: let $\sO(X)=k$, $Y$ be irreducible, and 
if $f:X\times Y\to Z$ is a morphism of schemes 
of finite type such that $f(X\times y_0)=z_0$ for some $y_0\in Y(k)$ and 
$z_0\in Z(k)$, then $f(x,y) = f(x_0,y)$ for all $(x,y)\in X\times Y$. In the 
literature, one often replaces ``$\sO(X)=k$'' with ``$X$ proper.'' 

In our case, define $G\times H \to H$ by $(g,h)\mapsto f(g) h f(g)^{-1} h^{-1}$. 
Note that this map is identically $=1$ if and only if our Theorem holds. But 
$G\times 1_H\mapsto 1_H$, so our rigidity lemma gives the result. 
\end{proof}
In particular, $G$ itself is commutative. 

\begin{proof}[Proof of rigidity lemma]
We use infinitesimal neighborhoods. The point $y_0\in Y(k)$ can be written 
as $y_0=\spectrum(\sO_{Y,y_0}/\fm)$. For each $n$, define 
$y_n = \spectrum(\sO_{Y,y_0}/\fm^{n+1})$. Then $\{y_n:n\geqslant 0\}$ is 
an increasing sequence of closed subschemes of $Y$, and the union 
$\bigcup y_n$ is dense in $Y$. (This is a reformulation of Krull's 
intersection theorem, which says that $\bigcap \fm^n=0$.) We claim that 
$f:X\times y_n \to z_n$ for all $n$, where $n$ is defined in the obvious 
way.  So $f_n(x,y) = f_n(x_0,y)$ identically. Let 
$W\subset X\times Y$ be defined by $f(x,y) = f(x_0,y)$. If 
$g:X\times Y\to Z\times Z$ is $(f,f(x_0,-))$, then 
$W=g^{-1}(\Delta_Z)$, and contains $X\times y_n$ for all 
$n$. By our density argument, $W=X\times Y$. 
\end{proof}

By our rigidity lemma, abelian varieties are commutative. Moreover, if 
$f:A\to G$ is a morphism of schemes, $G$ is a connected algebraic group, 
and $f(0)=1$, then $f$ is a homomorphism of group schemes. To see this, 
consider the morphism $(x,y)\mapsto f(x+y) f(y)^{-1} f(x)^{-1}$ from 
$A\times A$ to $G$. It sends $A\times 0$ and $0\times A$ to $1$, so it is 
identically $1$, whence the result. 





%\bibliographystyle{alpha}
%\bibliography{lyon-sources}

%\end{document}
