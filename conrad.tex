% !TEX root = main.tex


\section{Pseudo-reductive groups}
\thanksauthor{Brian Conrad}





\subsection{Motiviation and examples}

Suppose we have some problem concerning a ``general'' affine algebraic group over 
a field $k$. If $k$ is not perfect, it becomes quite difficult to reduce problems 
to the reductive case. 

\begin{enonce}[remark]{Example}
Let $G=\generallinear_n$. Let $X$ be the variety of non-degenerate quadratic forms 
on $k^n$. We assume the action of $G(\bar k)$ on $X(\bar k)$ is transitive. Then 
$X$ is a ``homogeneous space'' for $G$. So any choice of $x\in X(\bar k)$ induces 
$G/\stabilizer_{G_{\bar k}}(x) \isomorphism X_{\bar k}$. Now choose 
$x_0\in X(\bar k)$. Assume $k$ is a global field, $S$ is a finite set of places of 
$k$, and $x\in X(k)$. We want 
\[
  \# G(k)\backslash \{x\in X(k):x\in G(k_v)\cdot x_0\text{ for all }x\notin S\} < \infty .
\]
To study this set, introduce the transporter scheme $T_x=\transporter(x_0,x)$ for 
$x\in X(\bar k)$. Our set above injects into 
$\sha_S^1(k, G_{x_0})$. So the question is: is $\sha_S^1(k,G)$ finite for all affine 
algebraic $k$-groups $G$?
\end{enonce}

If $G$ is finite \'etale, finiteness of $\sha_S^1(k,G)$ follows from the 
\v Cebotarev Density Theorem. This can be applied to $G/G^\circ$ in general. 
If $G$ is connected reductive, Borel-Serre (in characteristic zero) and 
Harder / Borel-Prasad (in positive characteristic) have shown that these 
Tate-Shavarevich sets are finite. 

In characteristic $p$, we have the problem that $G_\reduced$ is often not a 
subgroup scheme of $G$, and even when it is, it may not be smooth. By a trick, 
using the fact that the extensions $k_v/k$ are separable, we can still replace 
$G$ with its maximal smooth subgroup. So we can reduce to the case where $G$ is 
smooth and connected. How far is such a $G$ from being reductive? For perfect 
$k$, not too far. Over perfect fields, $\sR_\unipotent(G_{\bar k})\subset G_{\bar k}$ 
descends to $U\subset G$. The unipotent group $U$ has a canonical filtration 
with successive quotients isomorphic to $\dG_\additive$. So we have an 
exact sequence 
\[
  1 \to U \to G \to G/U \to 1
\]
where the group on the left is ``manageable'' and the group on the right is 
reductive. 

If $k$ is imperfect, such $k$-descent of $\sR_\unipotent(G_{\bar k})$ may not exist 
in $G$. 

\begin{defi}
$\sR_{\unipotent,k}(G)$ is the maximal smooth connected unipotent normal $k$-subgroup of 
$G$. Similarly $\sR_k(G)$ is the maximal smooth connected solvable normal 
$k$-subgroup of $G$. 
\end{defi}

If $K/k$ is an extension, then $\sR_{\unipotent,k}(G)\subset \sR_{\unipotent,K}(G_K)$ and 
$\sR_k(G)_k\subset \sR_K(G_K)$. If $K/k$ is separable (e.g~$k^s/k$, 
$k(V)/k$ for smooth $V$, or $k_v/k$ for global $k$) then we have equalities, 
i.e.\ formation of radical commutes with separable field extension. This uses 
a ``smooth specialization'' argument. 

\begin{defi}
Say $G$ is \emph{pseudo-reductive} if $\sR_{\unipotent,k}(G)=1$. Say $G$ is 
\emph{pseudo-semisimple} if $\sR_{\unipotent,k}(G)=1$ and $G=\sD G$. 
\end{defi}
We will see later that pseudo-semisimple groups have $\sR_k G=1$, but the 
converse fails. We can make examples of commutative pseudo-reductive groups 
$C$ with $C(k^s)[p]\ne 1$. These are very ``far'' from being tori, and there 
is no nice classification (akin to Galois lattices) for such groups. 

Note that for general smooth connected affine $G$, in the following exact sequence:
\[
  1 \to \sR_{\unipotent,k}(G) \to G \to G/\sR_{\unipotent,k}(G) \to 1
\]
the group on the right is pseudo-reductive. The unipotent group 
$\sR_{\unipotent,k} G$ can be pretty horrible. Tits developed useful filtrations for 
them. See Appendix B in \cite{cgp10} for details. 

We will develop a structure theory for pseudo-reductive groups. This can 
be used to show that $\sha_S^1(k,G)$ is finite for function fields. 





\subsection{Weil restriction}

Let $k$ be a field, and $k'$ a finite $k$-algebra, i.e.~$\dim_k k'<\infty$. Let 
$X'$ be an affine algebraic $k'$-scheme. Note that the algebra $k'$ is a finite 
product of local artinian $k$-algebras. We define the \emph{Weil restriction} 
of $X'$, $X=R_{k'/k}(X')$ as ``$X$ viewed as a $k$-scheme using the $k$-basis of 
$k'$ to make coordinates.'' So if $X'$ is smooth of pure dimension $d$ over a 
field $k'$, then $X$ should be $k$-smooth of dimension $d[k':k]$. The 
functor of points of $X$ is 
\[
  X(A) = X'(k'\otimes_k A) .
\]
This should be viewed as analogous to viewing a $d$-dimensional complex 
manifold as a $2d$-dimensional real manifold. 

Some people write $\Pi_{k'/k}$ instead of $R_{k'/k}$ for Weil restriction. 

\begin{enonce}[remark]{Example}
$R_{k'/k} \dA_{k'}^n = \dA_k^{n[k':k]}$ via a $k$-basis of $k'$. 
\end{enonce}

\begin{enonce}[remark]{Example}
Note that $R_{k'/k}(\speciallinear_n)(A)=\speciallinear_n(k'\otimes_k A)$. 
This is given by ``$\det=1$'' on $M_n(k'\otimes_k A) \subset M_{n[k':k]}(A)$. 
This group can be pretty messy. 
\end{enonce}

\begin{enonce}[remark]{Example}
Suppose $k=\prod k_i'$. Then $\spectrum(k')=\coprod \spectrum(k_i')$, so 
$X'=\coprod X_i'$ for $k_i'$-schemes $X_i'$. One has 
$X=R_{k'/k}=\prod_i R_{k_i'/k}(X_i')$. Checking this is a good exercise. 
\end{enonce}

\begin{enonce}[remark]{Example}
Let $G'$ be a smooth algebraic $k'$-group, $\frakg'=\lie G'$. Then 
$\lie G$ is $\frakg'$ viewed as a ($p$-) Lie algebra over $k$. See 
\cite[A.7.6, A.7.13]{cgp10} for details. 
\end{enonce}

Why should we care about Weil restriction? Suppose $k'/k$ is a (finite) 
field extension, and $K/k$ is a ``big'' field extension (e.g.~$K=k^s$). 
Then $K'=k'\otimes_k K$ will be a finite $K$-algebra. If $k'/k$ is a 
separable extension, $K'$ will be a product of fields. If $k'/k$ is 
inseparable, $K'$ may have nilpotents. One always has 
$R_{k'/k}(X')_K = R_{K'/K}(X_{K'}')$. This is easily checked by looking 
at the functors of points on $K'$-algebras.

\begin{enonce}[remark]{Example}
Let $k'/k$ be a separable field extension. Then let 
$A=\prod_{k'\hookrightarrow k^s} k^s$. We have 
\[
  R_{k'/k}(X)_{k^s} = \prod_{A/k^s} (X'\otimes_{k'} A) = \prod_\sigma \sigma^\ast X' .
\]
This explains the $\prod$ notation for Weil restriction. 
\end{enonce}

\begin{enonce}[remark]{Example}
Suppose $k'=k(\sqrt[p] a)$ for $a\in k\smallsetminus k^p$. Consider 
$R_{k'/k}(X')_{\bar k} = R_{(k'\otimes \bar k)/\bar k}(X'\otimes_{k'} \bar k)$. 
This lives over the (non-reduced) ring $\bar k[t]/t^p$. 
\end{enonce}

\begin{enonce}[remark]{Example}
Let $k'/k$ be a non-separable extension. Then the group 
$R_{k'/k}(\generallinear_n)$ is \emph{never} reductive, but is always 
pseudo-reductive. For $\bar k$-points, 
$\generallinear_n(\bar k\otimes k') = \generallinear_n(\bar k[t]/t^p)$, 
which has a ``huge $p^\infty$-torsion'' subgroup. In general, for 
connected reductive $G'\ne 1$ over $k'$, the Weil restriction 
$R_{k'/k}(G')$ is never reductive, but always pseudo-reductive. (Again, 
$k'/k$ is non-separable here.)
\end{enonce}

\begin{theo}[Borel-Tits IHES 27, 6.21iii]
Let $G$ be a connected, semisimple, simply-connected $k$-group. Then there 
exists (unique up to unique isomorphism) a pair $(k'/k,G')$ where $k'$ is a finite 
\'etale $k$-algebra, $G'=\coprod G_i'$, where the $G_i'$ are absolutely 
simple, simply connected semisimple groups over the $k_i'$ such that 
$G\simeq R_{k'/k} G$. 
\end{theo}

The uniqueness in this theorem means the following. If $(k'',G'')$ is another 
such pair, then any $k$-isomorphism $R_{k'/k}(G')\simeq R_{k''/k} (G'')$ 
arises from a unique compatible pair 
$(\varphi:k'\isomorphism k'',\alpha:G'\isomorphism G'')$

\begin{proof}
By Galois descent, we can assume $k=k^s$. Then $k'=\prod k^s$, so the theorem 
just says that $G$ is a product of simple factors, and that isomorphisms between 
such products arise from bijections of their index sets, and compatible 
group isomorphisms. 
\end{proof}

\begin{lemm}
1. Suppose $k'$ is a field, and $X'$ is smooth over $k'$. Then $X$ is smooth over $k$. 

2. If $X'$ is geometrically connected and smooth over $k'$, then the same are true 
for $X$ over $k$. 
\end{lemm}
\begin{proof}
1. Use the infinitesimal criterion. It's enough to use finite algebras. 

2. In the group setting: the key case is $k=k^s$, $k'$ a field. We'll look at 
$k=\dF_p(t,u)$ and $k'=k(\sqrt[p] t)$. Then 
$G(\bar k)=G'(k'\otimes \bar k)\xrightarrow q G'(\bar k)$. The kernel of $q$ 
has a filtration by $U_i=\{g\equiv 1 \mod \fm^i\}$; these are normal in 
$G_{\bar k}$, and $U_i/U_{i+1} \simeq \frakg'\otimes_{\bar k} \fm^i/\fm^{i+1}$. 
So ``we can conclude'' that $G_{\bar k}$ has a composition series with 
connected successive quotients. 
\end{proof}

Part 2 is false without the smoothness assumption! You can make examples of a 
geometrically integral curve $X'\subset \dA_{k'}^2$, smooth away from one point, 
regular at the missing point, such that $X$ has two connected components. 

\begin{prop}
Let $k'/k$ be a non-zero finite reduced $k$-algebra, $G'$ a pseudo-reductive 
$k'$-group with connected fibers. Then $G=R_{k'/k} (G')$ is a pseudo-reductive 
$k$-group. 
\end{prop}
\begin{proof}
Without loss of generality, $k=k^s$ and $k'$ is a field. 
We know that $G$ is smooth and connected. We need to show that it has no 
nontrivial smooth connected unipotent normal $k$-subgroups. Let 
$U\subset G$ be such a subgroup; we want to show $U=1$. The definition 
of Weil restriction means that $U\hookrightarrow G$ corresponds to a map 
$f:U_{k'} \to G'$ over $k'$. This map is equivariant for the 
$G(k)=G'(k')$-action, and $G(k)\subset G$ is Zariski-dense (since $G$ is smooth 
and $k=k^s$). Thus $f(U_{k'})$ is normal in $G_{k'}$, so $f=1$. Thus the 
map $U\hookrightarrow G$ is trivial, so $U=1$. 
\end{proof}





\subsection{Some disorienting examples}

\begin{enonce}[remark]{Example}
Let $k'=k(\sqrt[p] a)$ for $a\in k\smallsetminus k^p$ and $k$ of characteristic 
$p>0$. Consider the fppf-exact sequence 
\[
  1 \to \boldsymbol\mu_p \to \dG_\mult \to \dG_\mult \to 1 
\]
over $k'$. Restrict to $k$ (and note that Weil restriction is left exact): 
\[\xymatrix{
  1 \ar[r] 
    & R_{k'/k}(\boldsymbol\mu_p) \ar[r] \ar@{.>}[dr] 
    & R_{k'/k}(\dG_\mult) \\
  & & \dG_\mult \ar@{^{(}->}[u] 
}\]
We get a fppf-exact sequence 
\[
  1 \to R_{k'/k}(\boldsymbol\mu_p) \to R_{k'/k} \dG_\mult \to \dG_\mult \to 1 
\]
where $R_{k'/k}(\boldsymbol\mu_p)$ has dimension $p-1$. However 
$R_{k'/k}(\boldsymbol\mu_p)$ has no $k^s$-points. Moreover, 
$R_{k'/k}(\dG_\mult)/\dG_\mult$ is killed by $p$. (We can check this on 
$k^s$-points, because $(H_1/H_2)(k^s)=H_1(k^s)/H_2(k^s)$ if 
$H_2$ is smooth) 
\end{enonce}

\begin{enonce}[remark]{Example}
Consider 
$1 \to \boldsymbol\mu_p \to \speciallinear_p \xrightarrow f \projectivegenerallinear_p \to 1$ 
over $k'$ and Weil restrict: 
\[
  1 \to R_{k'/k}(\boldsymbol\mu_p) \to R_{k'/k}(\speciallinear_p) \to R_{k'k}(\projectivegenerallinear_p) \to 1. 
\]
It turns out that 
\[
  R_{k'/k}(\speciallinear_p)/R_{k'/k}(\boldsymbol\mu_p) \hookrightarrow R_{k'/k}(\projectivegenerallinear_p) 
\]
is normal with abelian (unipotent) cokernel of dimension $p-1$. Look at an open 
cell: $U^-\times \widetilde T\times U^+ \xrightarrow f U^- \times T \times U^+$. 
On tori (in suitable coordinates) $f$ looks like $(t_1,\dots,t_{p-1})\mapsto (t_1^p,t_2,\dots,t_{p-1})$. 
On Weil restriction, we can write down the image directly to get that the cokernel 
is $R_{k'/k} (\dG_\mult)/\dG_\mult$. 
\end{enonce}

Using $\widetilde T=\dG_\mult^{\Delta^\vee}$, we can prove the following theorem: 

\begin{theo}
The Weil restriction of a simply connected group is perfect. 
\end{theo}

One reduces to $\speciallinear_2$ via open cells. 

[Note: the map Michel constructed can be seen as the composite 
\[
  X \hookrightarrow R_{K/k}(X_K) \isomorphism R_{K/k}(G_K) \xrightarrow{trace} G
\]
Also, think of Weil restriction as pushforward of fppf sheaves. 
]





\subsection{Elementary properties and examples}

\begin{lemm}
If $G$ is pseudo-reductive, every smooth connected normal $k$-subgroup 
$N\subset G$ is pseudo-reductive. 
\end{lemm}
\begin{proof}
We can assume $k$ is separably closed. We claim that $\sR_{\unipotent,k}(N)$ is also normal 
in $G$, hence $\sR_{\unipotent,k}(N)=1$. It suffices to check that $\sR_{\unipotent,k}(N)$ is 
stable under $G(k)$-conjugation (because $G(k)\subset G$ is Zariski-dense, which 
works because $k=k^s$ and $G$ is smooth). But $G(k)$-conjugation preserves $N$, 
so it certainly preserves $\sR_{\unipotent,k}(N)$, because it's stable under all 
$k$-automorphisms of $N$. 
\end{proof}

So, for example the derived subgroup $\sD G$ is pseudo-reductive. We've already 
seen that this fails for quotients. 

\begin{enonce}[remark]{Example}
Let $k'/k$ be a finite field extension. Let $G'$ be a connected, simply connected 
semisimple group over $k'$. Let $\widetilde G' \to G'$ be the simply-connected 
cover, and $\boldsymbol\mu=\ker(\widetilde G' \to G')$; this is of multiplicative 
type, because a maximal torus $\widetilde T'\subset \widetilde G'$ is its own 
centralizer, hence it contains all central subgroups. In fact, 
$\boldsymbol\mu\subset \widetilde T'[n]$ for some $n\geqslant 1$. Let 
$G=R_{k'/k}(\widetilde G')/R_{k'/k}(\boldsymbol\mu)\hookrightarrow R_{k'/k}(G')$. 
This injection has commutative cokernel (because this cokernel lives inside 
$\mathsf R_\mathrm{fppf}^1 f_\ast \boldsymbol\mu$, where 
$f:\spectrum k' \to \spectrum k$). The group $G$ is perfect! We see that 
$G=\sD(R_{k'/k} G)$, so $G$ is pseudo-reductive. By open cell consideration, 
$(R_{k'/k}(G')/G)_{k^s}$ can be described in terms of ``Weil-restricted $\dG_\mult$ 
modulo $\dG_\mult$,'' which is unipotent. Precisely, it is 
$\coker(R_{k'/k}(\widetilde T') \to R_{k'/k}(T'))$. If $k'/k$ is purely 
inseparable, the cokernel has no nontrivial tori, so it's unipotent. 
\end{enonce}

\emph{Warning}: over every imperfect $k$, there exists pseudo-semisimple $G$ 
and normal pseudo-semisimple $N\subset G$ such that $G/N$ is \emph{not} 
pseudo-reductive. 

Suppose a pseudo-reductive group $G$ is \emph{pseudo-split} if it has a split 
maximal $k$-torus. Does $G$ have a pseudo-split $k^s$-form? (i.e.\ does there 
exist pseudo-split $H/k$ such that $G_{k^s}\simeq H_{k^s}$) This fails even 
if $k$ is perfect! When such a Chevalley group exists, it is unique. 

The good news is that there is a good structure theory of (high-dimensional) 
root groups, root system (which could be non-reduced), open-cell, Bruhat 
decomposition on rational points, \ldots for pseudo-reductive groups. Also, 
Cartan subgroups are commutative (i.e.\ if $T\subset G$ is a maximal torus, 
then $Z_G(T)$ is commutative and pseudo-reductive). Also, 
$\sD(\sD G)=\sD G$. 

\begin{enonce}{Magic Lemma}
Let $G$ be a pseudo-reductive $k$-group, and $X\subset G$ a geometrically 
integral closed subscheme that meets $1$. If $X_{\bar k}\subset R_u(G_{\bar k})$, 
then $X=1$. 
\end{enonce}
\begin{proof}
Without loss of generality, $k=k^s$. Let $H\subset G$ be the $k$-subgroup 
normally generated by $X$ (so $H$ is the Zariski-closure of the smallest normal 
subgroup of $G(k)$ containing by $X(k)$). It is a basic fact that $H$ is 
smooth connected normal. Also, formation of $H$ commutes with arbitrary 
extension of the ground field. But $X_{\bar k}\subset \sR_\unipotent(G_{\bar k})$, so 
$H_{\bar k}\subset \sR_\unipotent(G_{\bar k})$. It follows that $H\subset \sR_\unipotent(G)$, so 
$H=1$, whence $X=1$. 
\end{proof}

\begin{coro}
1. If $G$ is pseudo-semisimple, then $\sR_k(G)=1$. 

2. If $f_1,f_2:G\twoheadrightarrow H$ are morphisms between pseudo-reductive 
groups such that $f_{1,\bar k}=f_{2,\bar k}$. Then $f_1=f_2$. 
\end{coro}
\begin{proof}
1. Let $X=\sR_k(G)$. Then 
$X_{\bar k}\subset R_{\bar k}(G_{\bar k}) = \sR_{\bar k}(G_{\bar k})$. Apply the 
magic lemma. 

2. Let $X$ be the Zariski-closure of the map $g\mapsto f_1(g) f_2(g)^{-1}$. Note 
that $X=1$ if and only if $f_1=f_2$. But 
$X_{\bar k}\subset \sR_{\bar k}(H_{\bar k})$, so apply the magic lemma. 
\end{proof}

\emph{Warning}: the converse to 1 is false!

\begin{theo}
Suppose $G$ is pseudo-reductive over $k$. 

1. If $G$ is solvable, then it is commutative. 

2. $\sD G$ is perfect (i.e.\ pseudo-semisimple). 

3. If $S\subset G$ is a $k$-torus, then $Z_G(S)$ is pseudo-reductive. 

4. If $T\subset G$ is a maximal $k$-torus, then $Z_G(T)$ is commutative. 
\end{theo}
\begin{proof}
1. Let $X=\sD G$. Note that $G_{\bar k}^\reduced$ is solvable and reductive, 
hence commutative. So $\sD(G_{\bar k}^\reduced)=1$. But 
$\sD(G)_{\bar k} \to G_{\bar k}^\reduced$ has image in $\sD(G_{\bar k}^\reduced)=1$. 
But $X_{\bar k}=\sD(G)_{\bar k} = \sD(G_{\bar k})$, so we can apply the 
magic lemma. 

2. This is much trickier. 

3-4. [\ldots didn't write down proof\ldots]
\end{proof}

Later on, we will work with root groups over $k$. We will show that if 
$N'\subset N\subset G$ are all smooth connected normal (in the next), then 
$N'\subset G$ is normal. (So normality is transitive for pseudo-reductive 
groups.) The first step in the proof is to pass to derived groups. A 
reference for this is \cite[1.2.7,3.1.10]{cgp10}. 





\subsection{Standard construction}

Let's start with motivation via reductive groups built from standard semisimple 
groups. If $G$ is connected reductive, then $G=\sD G\cdot Z$, where $Z$ is a 
maximal $k$-torus and $Z\cap \sD G$ is finite central. So 
$G=(Z\times \sD G)/\boldsymbol\mu$. 

Let $T\subset G$ be a maximal $k$-torus. Then $S=T\cap \sD G$ is a maximal 
$k$-torus of $\sD G$ (this is valid with $\sD G$ replaced by any smooth 
connected normal $k$-subgroup). Moreover, $T = Z S$ is an almost direct product. 
We have a commutative diagram 
\[\xymatrix{
  1 \ar[r] 
    & \boldsymbol\mu \ar[r] \ar@{^{(}->}[d] 
    & \sD G\times Z \ar[r] \ar@{^{(}->}[d] 
    & G \ar[r] \ar@{=}[d] 
    & 1 \\
  1 \ar[r] 
    & S \ar[r] 
    & \sD G\rtimes T \ar[r] 
    & G \ar[r] 
    & 1
}\]
The square on the left is a central pushout. We can write 
$G=(\sD G\times T)/S$. This is better because Weil restriction commutes with 
quotients by \emph{smooth} subgroups. 

[\ldots more stuff I didn't understand\ldots stopped taking notes\ldots]

Suppose $G$ is a pseudo-reductive group over $k$. We have 
$\sR_\unipotent(G_{\bar k})\subset G_{\bar k}$. By Galois descent, this comes from 
$U\subset G_{k^\mathrm{perf}}$. This descends to some $U_0\subset G_K$, for a 
finite purely inseparable extension $K/k$. 

\begin{enonce}{Fact}
If $K/k$ is a field extension and $X$ is a $k$-scheme, then any closed 
$Z\subset X_K$ desends to $Z_0\subset X_0$, for a unique 
$k_0\subset K$ and $X_0$ over $k_0$. 
\end{enonce}
This is \cite[4.8]{ega4}. 

In the discussion above, take $K/k$ minimal such that $\sR_{\unipotent,K}(G_K)$ is a 
$K$-descent of $\sR_\unipotent(G_{\bar k})$. We can form 
$G'=G_K^\reduced$. This corresponds to 
$i_G:G\to R_{K/k}(G')$. Unfortunately, $i_G$ is very far from being an isomorphism. 
First problem: if we pass to $G_{\bar k} \twoheadrightarrow G_{\bar k}^\reduced \twoheadrightarrow G_{\bar k}^\mathrm{ad} = \prod G_i$, where the $G_i$ are adjoint 
semisimple. The kernel of $G\to G_i$ has a minimal field of definition 
$k_i'/k$. But $K$ lumps together all the $k_i'$. 

Now we can outline the construction. Let $k'$ be a finite reduced $k$-algebra. 
Let $G'=\coprod G_i'$ be a reductive $k'$-group with each fiber 
$G_i'$ connected. The main case we're interested in is when all $G_i'$ are 
absolutely simple and simply connected semisimple. Let 
$T=\coprod T_i'$ be a maximal $k'$-torus in $G'$. The Weil restriction 
$R_{k'/k}(T') \subset R_{k'/k} G'$ is a Cartan subgroup -- see 
\cite[A.5.15(i)]{cgp10}. Consider a factorization 
$R_{k'/k}(T')\xrightarrow\phi C \to R_{k''/k}(\overline T') \hookrightarrow R_{k'/k}(G^\mathrm{ad})$, where $C$ is commutative pseudo-reductive. Let 
\[
  G=\frac{R_{k'/k}(G')\rtimes C}{R_{k'/k}(T')}. 
\]
The natural map $C\to G$ turns out to be an inclusion, and it makes $C$ a 
Cartan subgroup of $G$. 

\begin{theo}
1. $G$ is pseudo-reductive. 

2. If $G$ is not commutative, then for any maximal torus $\sT\subset G$ and 
$\sC=Z_G(\sT)$, we can find a four-tuple $(G'',k'',T'',C'')$ such that 
\begin{itemize}
  \item $C''=\sC$
  \item all fibers $G''_j$ over factor fields $k_j''$ are absolutely simple, simply-connected, semisimple. 
\end{itemize}
3. This ``better 4-tuple'' is unique up to unique isomorphism 
\end{theo}
We call the $G$ as in 2 ``standard.'' 
For a proof of pseudo-reductivity of the standard construction: 
$(\sG\rtimes C)/\sC$ ($\sC\subset \sG$ self-centralizing, $\sG$ 
pseudo-reductive, and $\sC\to C$ acts on $\sG$ compatible with the 
$\sG$-action) see \cite[1.4.3]{cgp10}. 





\subsection{Fields of definition and splitting of central extensions}

Last time we ``massaged'' our 4-tuple $(G',k',T',C)$ in standard construction 
to have the extra property that all the fibers $G_i'$ over factor fields are 
absolutely simple and simply connected. With these extra properties, we claimed 
a very strong uniqueness result for the 4-tuple, given $(G,T)$ ($C=Z_G(T)$). 
Uniqueness reduces existence proofs for ``standardness'' to the case of a 
separably closed ground field. 

The key issue for the uniqueness aspect is: suppose $k'/k$ is a finite field 
extension, $G'$ is an absolutely simple connected semisimple $k'$-group, 
and $G=R_{k'/k}(G')$. In what sense does $G$ determine $(G',k'/k)$ uniquely 
up to unique isomorphism? 

\begin{enonce}[remark]{Example}
Suppose $k'/k$ is separable, and $G'$ is simply connected. We saw that 
$(G',k'/k)$ is unique via an indirect reason: if $(G'',k''/k)$ is another pair, 
any $k$-isomorphism $R_{k'/k}(G')\isomorphism R_{k''/k}(G'')$ arises from a 
unique pair $(\varphi,\alpha)$, where 
$\varphi:k'\isomorphism k''$ and $\alpha:G'\isomorphism G''$ are compatible. 
\end{enonce}

We want a more ``explicit'' way to extract $(G',k')$ from $G$. There is a 
natural map $q:G_{k'}  = R_{k'/k}(G')_{k'} \to G'$ (much like 
$f^\ast f_\ast \sF \to \sF$ for sheaves). For a $k'$-algebra $A'$, when 
evaluated at $A'$ this is $G'(k'\otimes_k A') \to G'(A')$, induced from 
$k'\otimes_k A' = (k'\otimes_k k')\otimes_{k'} A' \twoheadrightarrow k'\otimes_{k'} A' = A'$. 
Over $k'$, $q$ gives an absolutely simple conected semisimple quotient of 
$G_{k'}$. If $k'\otimes_k k'$ has several copies of $k'$ as quotient fields 
(e.g.\ $k'\otimes_k k' \twoheadrightarrow k'$ given by 
$a\otimes b\mapsto a\sigma(b)$, for $\sigma\in \automorphism(k'/k)$), then we 
get quotients $q_\sigma:G_{k'}\twoheadrightarrow \sigma^\ast (G')$, with 
\emph{different} kernels than $q$. The morphism $q$ is a surjection with 
smooth connected kernel \cite[A.5.11(1,3)]{cgp10}. 

\begin{enonce}[remark]{Example}
Suppose $k'/k$ is purely inseparable. Then $\ker(q)$ is \emph{unipotent}. So 
$\ker(q)=\sR_{\unipotent,k'}(G_{k'})$ is a $k'$-descent of $\sR_\mathrm{u}(G_{\bar k})$. 
It turns out that this $k'/k$ is the \emph{minimal} field of definition over 
$k$ of $\sR_\unipotent(G_{\bar k})\subset G_{\bar k}$. Note that 
$G'=G_{k'}/\sR_{\unipotent,k'}(G_{k'})$ recovers $G'$ from $G$ in such cases. 
\end{enonce}

\begin{prop}
Let $H$ be a non-solvable smooth connected affine $k$-group. Consider pairs 
$(k'/k,q:H_{k'} \to \sH')$ where 
\begin{itemize}
  \item $k'$ is a nonzero finite reduced $k$-algebra ($=\prod k_i'$)
  \item $\sH'=\coprod \sH_i'$ is a $k'$-group where each $\sH_i'$ is 
    an absolutely simple connected semisimple $k_i'$-group of adjoint type. 
  \item $q:H_{k_i'} \to \sH_i'$ is a maximal absolutely simple adjoint 
    semisimple quotient. 
\end{itemize}
There is an initial such pair, which we denote by 
$(K/k, q_H:H_K \to \overline\sH)$. That is, for any 
$(k'/k,q:H_{k'} \twoheadrightarrow \sH')$, there is a unique 
$\varphi:K \to k'$ as $k$-algebras such that there is a (unique) isomorphism 
making the following diagram commute:
\[\xymatrix{
  H_{k'} \ar@{->>}[r]^-{q} \ar@{->>}[dr]_-{\varphi^\ast(q_H)} 
    & \sH' \\
  & \overline \sH_K \ar[u]^\wr
}\]
\end{prop}
\begin{proof}
This is \cite[4.2.1]{cgp10}. It's enough to prove the existence and uniqueness 
under the assumption that $k=k^s$. Look at 
$H_{\bar k} \twoheadrightarrow H_{\bar k}^\mathrm{ss}\twoheadrightarrow H_{\bar k}^{\adjoint} = \prod H_i'$ consider some composite 
$q_i:H_{\bar k}\twoheadrightarrow H_i'$. Let $k_i'/k$ be the minimal field of 
definition over $k$ of $\ker(q_i)\subset H_{\bar k}$. Put 
$K=\prod k_i'$ and $q_i:H_{k_i'}\twoheadrightarrow H_{k_i'}/(\text{descent of }\ker(q_i'))$. 
\end{proof}

\begin{enonce}[remark]{Example}
If $H=R_{k'/k}(G')$, where $k'/k$ is Galois, then $K=k'$ and 
$q:H_{k'}\twoheadrightarrow G'$ is the canonical quotient. For any 
$\gamma\in \galois(k'/k)$, we also have 
$q_\gamma:H_{k'}\twoheadrightarrow \gamma^\ast (G')$. This map is 
$\gamma^\ast(q_H)$. So by keeping track of the quotient map, we eliminate all 
ambiguity in the field maps. 
\end{enonce}

\begin{defi}
The \emph{simply connected datum} attached to $H$ is $(K/k,\widetilde\sH,f)$, 
where $\widetilde\sH\to \overline\sH$ is the simply-connected simple cover and 
$f:H\to R_{K/k}(\overline\sH)$ corresponds to 
$q:H_K \twoheadrightarrow\overline\sH$. 
\end{defi}

\begin{enonce}[remark]{Example}
Suppose $G$ is in standard pseudo-reductive form $(G',kk'/k,T',C)$, with all 
$G_i'$ absolutely simple over $k_i'$ and simply connected. The simply connected 
datum is $(k'/k,G',f)$, where 
\[
  f:G=\frac{R_{k'/k}(G')\rtimes C}{R_{k'/k}(T')} \to \frac{R_{k'/k}(R^{\adjoint})\rtimes R_{k'/k}(\overline T')}{R_{k'/k}(\overline T')} = R_{k'/k}(G^{\adjoint}) .
\]
\end{enonce}
\begin{proof}
Without loss of generality, $k=k^s$. Chase fields of definition, and do a bit 
more work with adjoint quotients. 
\end{proof}

So if we want to prove some non-commutative (hence non-solvable) 
pseudo-reductive group $G$ is standard, its simply connected datum gives a good 
candidate for $(k'/k,G')$. But how are we to relate $G$ to 
$(R_{k'/k}(G')\rtimes C)/R_{k'/k}(T')$? All we have is a map 
$R_{k'/k}(G') \to R_{k'/k}({G'}^{\adjoint})$. In the end, we need to exploit the 
minimality of $k'$. 

Any pseudo-reductive $G$ is generated by its derived subgroup and a single 
Cartan: $G=C\cdot \sD G$. This is because $G\twoheadrightarrow G/\sD G$ is 
a \emph{commutative} quotient. The Cartan $C\subset G$ will have to map 
surjectively onto $G/\sD G$, whence the claim. By fiddling around with $C$, we 
can show that $G$ is standard if and only if $\sD G$ is standard. 
So its enough to show that $\sD G$ is standard. Thus we can assume 
$k=k^s$ and $G$ is pseudo--semisimple. 

\begin{theo}
If $6\in k^\times$, then $G$ is standard. If $k$ has characteristic $3$, or 
$k$ has characteristic $2$ with $[k:k^2]=2$, then $G$ is ``generalized 
standard.'' 
\end{theo}

How to prove the main theorem? A key tool is root groups. We want to reduce to 
the case where $G$ is an ``irreducible root system'' (akin to ``simple 
factors'' of a connected semisimple group). 

\begin{prop}
Let $G$ be pseudo-semisimple. Let $\{N_i\}$ be a set of minimal nontrivial 
normal pseudo-semisimple $k$-subgroups. Then the $N_i$'s pairwise commute, 
$\#\{N_i\}<\infty$, and $\prod N_i \to G$ is surjective with central kernel. 
\end{prop}
\begin{proof}
Without loss of generality, $k=k^s$. We have a split maximal torus. Fiddle 
around with ``root groups.'' 
\end{proof}

\emph{Warning}: the kernel of $\prod N_i \twoheadrightarrow G$ could have 
positive dimension. 

The upshot of this is that when $k=k^s$, we can reduce to the case where $G$ 
has \emph{no} nontrivial smooth connected normal $k$-subgroups $N\ne G$ 
(i.e.~$G$ pseudo-simple, which is equivalent to the root system of $G$ being 
irreducible). 

We reformulate the problem. Let $G$ be an absolutely pseudo-simple (i.e.\ 
$G_{k^s}$ is pseudo-simple) group. Let $K$ be the minimal field of definition 
of $\sR_\unipotent(G_{\bar k})\subset \sR(G_{\bar k})$. We have 
$G_K\twoheadrightarrow G_K^\mathrm{ss}=G'$, and 
\[
  \sD(R_{K/k}(G')) = \frac{R_{K/k}(G')}{R_{K/k}(\boldsymbol\mu)} ,
\]
where $G'=\widetilde G'/\boldsymbol\mu$. We would like there to exist 
$j_G$ as in the diagram:
\[\xymatrix{
  & R_{K/k}(\widetilde G') \ar[d] \ar@{.>}[dl]_-{j_G} \\
  G \ar[r]_-{\xi_G} 
    & \frac{R_{K/k}(\widetilde G')}{R_{K/k}(\boldsymbol\mu)} .
}\]
If $j_G$ exists, then it is unique. Any two are related through multiplication 
against a homomorphism $R_{K/k}(\widetilde G') \to R_{K/k}(\boldsymbol\mu)$. 
But the first group is perfect and the second is commutative, so such 
homomorphisms are trivial. 

\begin{lemm}
$G$ is standard if and only if $j_G$ exists. 
\end{lemm}

\begin{theo}
$G$ is standard if and only if $\xi_G$ is surjective and $\ker(\xi_G)\subset G$. 
\end{theo}
\begin{proof}
This is \cite[5.3.8]{cgp10}. The important direction is $\Leftarrow$. 
\end{proof}

We have an exact sequence 
\[\xymatrix{
  1 \ar[r] 
    & G \ar[r] 
    & Z \ar[r]^-{\xi_G} 
    & \frac{R_{K/k}(\widetilde G')}{R_{K/k}(\boldsymbol\mu)} \ar[r] 
    & 1 .
}\]
The group $Z$ is central with no nontrivial connected smooth $k$-subgroups. 
Extend this to a diagram 
\[\xymatrix{
  1 \ar[r] 
    & Z \ar[r] \ar@{=}[d] 
    & E \ar[r] \ar[d] 
    & R_{K/k}(\widetilde G') \ar[r] \ar[d] 
    & 1 \\
  1 \ar[r] 
    & G \ar[r] 
    & Z \ar[r]^-{\xi_G} 
    & \frac{R_{K/k}(\widetilde G')}{R_{K/k}(\boldsymbol\mu)} \ar[r] 
    & 1 .
}\]
where the square on the right is a pullback. The miracle is: for $Z$ a 
commutative affine algebraic $k$-group scheme with no nontrivial smooth 
connected $k$-subgroup (i.e.\ $Z(k^s)$ is finite)there is a criterion for 
central extensions $1 \to Z \to E \to \sG \to 1$ to be split, at least for 
pseudo-reductive $k$-groups $\sG$. This criterion is: ``$Z_\sG(T)_{k^s}$ is 
rationally expressed in terms of root groups.'' In other words, there are 
rational maps $h_i:Z_\sG(T)_{k^s} \to U_{\alpha_i}$ such that 
$\prod h_i:Z_\sG(T)_{k^s} \to \sG$ is the inclusion. 

\begin{enonce}[remark]{Example}
Take $\sG=R_{k'/k}(\widetilde G')$. Then $Z_\sG(T)=R_{k'/k}(\widetilde T')$, 
where $\widetilde T'=\prod_{a\in \Delta}a^\vee \dG_\additive$. Here we use the 
formula 
\[
  \smat{t}{}{}{1/t} = \smat{1}{t}{}{1} \smat{1}{}{-1/t}{1} \smat{1}{t-1}{}{1} \smat{1}{}{1}{1} \smat{1}{1}{}{1} .
\]
\end{enonce}



