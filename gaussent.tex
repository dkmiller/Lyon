\documentclass{article}

\usepackage{lyon-style}

\title{Path models in representation theory}
\author{St\'ephane Gaussent}
\date{June 9--13, 2014}

\begin{document}
\maketitle
\tableofcontents





\section{Introduction}

We are interested in so-called ``saturation results.'' Good references are 
\cite{km08}, \cite{bcgr13}, and a survey of Kumar (with an appendix by 
Kapovich). 

We start with some notation. Let $G$ be a connected algebraic semisimple group 
over $\dC$. Let $T\subset G$ be a maximal torus, and $\Phi=\Phi(G,T)$ the 
corresponding root system. Let $X=X^\ast(T)=\hom(T,\dG_\mult)$ and 
$Y=X_\ast(T)=\hom(\dG_\mult,T)$. There is a perfect pairing (evaluation) 
$Y\times X \to \hom(\dG_\mult,\dG_\mult) = \dZ$. We can create the dual 
system $\Phi^\vee$, which corresponds to a semisimple group $G^\vee$, the 
\emph{Langlands dual} of $G$. For example, if 
$G=\symplectic_{2n}$, then $G^\vee=\specialorthogonal_{2n+1}$. 

[Note: Gaussent writes $\{\alpha_i:i\in I\}$ for the set of roots of $G$.]

Let $Y^+ = \{\nu\in Y:\alpha(\nu)\geqslant 0\text{ for all }\alpha\in \Phi\}$ 
be the set of \emph{dominant coweights}. For $\lambda\in Y^+$, $V(\lambda)$ is 
the irreducible representation of $G^\vee$ of highest weight $\lambda$. Let 
$\fZ=\fZ(G,\dC\laurent t) = G(\dC\laurent t)\times \dA/\sim$ be the Bruhat-Tits 
building associated to $G$ and $\dC\laurent t$. Here $\dA=Y_\dR$. 

An integer $k$ is a \emph{saturation factor} for $G^\vee$ if for any 
dominant coweights $\lambda,\mu,\nu$ such that 
$\lambda+\mu+\nu\in Q^\vee = \langle \alpha^\vee:\alpha\in \Phi\rangle\subset Y$, 
then there exists $N\geqslant 1$ such that 
$(V(N\lambda)\otimes V(N\mu)\otimes V(N\nu))^{G^\vee}\ne 0$ implies 
$(V(k\lambda)\otimes V(k\mu)\otimes V(k\nu))^{G^\vee}\ne 0$. 

\begin{theorem}[Kapovich-Millson, Leeb]
Let $\theta^\vee=\sum k_i \alpha_i^\vee$ be the highest coroot and 
$k_\Phi=\operatorname{lcm}(k_i)$. Then $k_\Phi^2$ is a saturation factor for 
$G^\vee$. 
\end{theorem}

This is not an isolated result. Knutson and Tao \cite{kt99} and 
Derksen and Weyman \cite{dw00} proved this for $G^\vee$ of type A, using 
combinatorics (resp.\ quivers). More recently, Belkale and 
Kumar \cite{bk10} were able to show that if $G^\vee$ is one of 
$\specialorthogonal_{2n+1}$ or $\symplectic_{2n}$, then $2$ is a 
saturation factor. Also Sam [2012] was able to show that if $G^\vee$ is one 
of $\specialorthogonal_{2n+1}$, $\symplectic_{2n}$, or 
$\specialorthogonal_{2n}$, then $2$ is a saturation factor. 

In a 2014 preprint, Hang-Shen was able to show that if $G^\vee$ is 
$\operatorname{Spin}_{2n+1}$, then $2$ is a saturation factor. Also, 
$72$ is a saturation factor for $F_4$. 
\begin{center}
\begin{tabular}{c|c|c|c|c|c|c|c|c|c}
  $G^\vee$ & A & B & C & D & $E_6$ & $E_7$ & $E_8$ & $F_4$ & $G_2$ \\ \hline 
  \cite{kt99} & 1 \\
  \cite{dw00}, Sam & 1 & 2 & 2 & 2\\
  \cite{bk10} & & 2 & 2 & 1 \\
  HS & & 2 & & & & & & 72 \\
  \cite{km08} & 1 & 4 & 4, 2 & 4, 1 & 36 & 144 & 900 & 144 & 363
\end{tabular}
\end{center}

\begin{conjecture}[Kapovich-Millson]
If $G^\vee$ is simply-laced, then $1$ is a saturation factor. Otherwise 
$2$ is a saturation factor. 
\end{conjecture}

We will concentrate on the results in \cite{km08}. Here is a sketch of the 
proof. 

\paragraph{Step 1 (unfolding)}
The fact that $(V(N\lambda)\otimes V(N\mu)\otimes V(N\nu))^{G^\vee}\ne 0$ 
implies that there exists a geodesic triangle $T(0,A,B)$ in the building 
$\fZ$ with side lengths $N\lambda$, $N\mu$, $N\nu$. 

\paragraph{Step 2 (Gauss map)}
We follow the geodesics in our triangle $T(0,A,B)$ off to infinity, to get 
a configuration of weighted points $(\xi_1,N\lambda)$, $(\xi_2,N\mu)$, 
$(\xi_3,N \nu)$ at infinity. This configuration is semistable. 

\paragraph{Step 3 (invert Gauss map)}
Semistability is stable by dilation. Then 
$((\xi_1,\lambda),(\xi_2,\mu),(\xi_3,\nu))$ is also semistable. Invert the 
map $\psi:T(0,A,B)\mapsto ((\xi_1,N\lambda),(\xi_2,N\mu),(\xi_3,N\nu))$. 
This gives us $\phi:\fZ\to \fZ$. We will show that $\phi$ has a fixed point. 

\paragraph{Step 4 (folding, $1^\mathrm{st}$ dilation)}
We have a triangle $T(x_0,x_1,x_2)$ in $\fZ$. The condition 
$\lambda+\mu+\nu\in Q^\vee$ implies that the fixed point $x_0$ is a vertex of 
$\fZ$. Fold this triangle onto the fundamental apartment $\dA$. We get a 
polyline $P[x_0,a_1,a_2,\dots,a_n,x_0]$ in $\dA$. Apply multiplication by 
$k_\Phi$ in $\dA$. We get a Hecke path of shape $k_\Phi \mu$. 

\paragraph{Step 5 (one-skeleton, $2^\mathrm{nd}$ dilation)}
Use reflections, bring everything inside $C^+$. In this situation, one can 
replace the path [drawn] in red by a LS path, up to multiplication by 
$k_\Phi$. We can now apply a theorem of Littleman to conclude that 
$(V( k_\Phi^2 \lambda)\otimes V(k_\Phi^2\mu)\otimes V(k_\Phi^2 \nu))^{G^\vee}\ne 0$. 





\section{The path model}

This is due to Littleman \cite{l94}. 


\subsection{The standard apartment}

This is the standard apartment of $\fZ$. It is denoted by $\dA$. It is an 
Euclidean affine space denoted by $V=Y_\dR = \hom(X,\dR)$. Let 
$C^+=\{v\in V:\alpha(v)\geqslant 0\text{ for all }\alpha\in \Phi\}$. The Weyl 
group $W$ of $(G,T)$ acts isometrically on $\dA$, and $C^+$ is a fundamental 
domain for this action. For any $v\in V$, there exists a unique 
$v_0\in C^+\cap W v$. So, define a map $\operatorname{pi}_{C^+}:V\to C^+$ by 
$v\mapsto v_0$. For any $x,y\in \dA$ we set 
$d_{C^+}(x,y) = \operatorname{pi}_{C^+}(y-x)\in C^+$. This is not a distance: 
$d_{C^+}(y,x) = - w_0 d_{C^+}(x,y)$, were $w_0$ is the largest element in $W$. 

The walls of $\dA$ are the hyperplanes, for all $\alpha\in \Phi$, 
$k\in \dZ$, defined by 
\[
  H_{\alpha,k} = \{x\in \dZ:\alpha(x)+k=0\} .
\]
The affine Weyl group $W^\affine=W\ltimes Q^\vee$ is generated by the 
reflections $s_{\alpha,k}$ along $H_{\alpha,k}$. An \emph{alcove} is the 
closure (in the Euclidean topology) of a connected component of 
\[
  \dA\smallsetminus \bigcup_{\substack{\alpha\in \Phi \\ k\in \dZ}} H_{\alpha,k} .
\]
An alcove is a fundamental domain for the action of $W^\affine$ on $\dA$. A 
vertex of $\dA$ is a vertex of an alcove. A vertex is \emph{special} if 
$\alpha(x)\in \dZ$ for all $\alpha\in \Phi$. 
Dilation by $k_\Phi$ sends any vertex on a special one and it is the smallest integer 
that does so. 





\bibliographystyle{alpha}
\bibliography{lyon-sources}

\end{document}
