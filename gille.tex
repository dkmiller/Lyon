\documentclass{article}

\usepackage{lyon-style}

\title{Affine groups in positive characteristic}
\author{Philippe Gille}
\date{June 2-6, 2014}

\begin{document}
\maketitle
\tableofcontents





\section*{Introduction}

Everything to be discussed is classical, in the sense that it can (probably) be 
found in \cite{dg80} or \cite{milneAGS}. 

The classical theory of affine algebraic groups over $\dC$ treats such groups 
as (Zariski-closed) subgroups of various $\generallinear_n(\dC)$. This theory 
is closely connected to that of compact Lie groups. For example, the torus 
$\dG_\mult=\generallinear_1=\spectrum(\dC[t^{\pm 1}])$ corresponds to the 
one-dimensional real torus $S^1$. Everything can be treated as varieties over 
$\dC$. 

Let $k$ be an arbitrary field, e.g.\ $\dF_q$, $\dF_q(t)$, $\dR$, \ldots. The 
natural approach to algebraic groups over $k$ is a ``schematic'' one, i.e.\ 
using schemes. For fields of positive characteristics, there are some recent 
improvements to the theory of algebraic groups over $k$. On the one hand, there 
is theory of \cite{cgp10} on so-called pseudo-reductive groups, and 
for commutative groups, there is recent work of Brion and Titano(sp?). 

New phenomenon occur even over $k=\overline{\dF_p}$. For example, 
$\generallinear_2$ is no longer ``linearly reductive.'' In other words, its 
category of representations is \emph{not} semisimple. Also, there are 
``extensions'' of $\dG_\additive=\spectrum(k[t])$ by itself which are not 
split (coming from Witt vectors). For example, one could put 
$(x_0,x_1)+(y_0,y_1) = (x_0+y_0, x_1+y_1-S_1(x_0,y_0))$, where 
\[
  S_1(x,y) = \frac{(x+y)^p - x^p-y^p}{p} \in \dZ[x,y] .
\]
This is the group $W_2$ of ``additive Witt vectors of length two.'' It fits 
into an exact sequence $0 \to \dG_\additive \to W_2 \to \dG_\additive \to 0$ 
which is not split. Indeed, evaluated at $\dF_p$, the sequence is 
$0 \to \dZ/p \to \dZ/p^2 \to \dZ/p \to 0$. 

Also, in positive characteristic, the intersection of (smooth) groups is not 
necessarily smooth! For example, let $G_1=\{x^p+x=y\}$ and 
$G_2=\{x=y\}$ as subgroups of $\dG_\additive^2$. The intersection 
$G_1\times_{\dG_\additive^2} G_2 = \{x^p=0, x=y\}$, which is not even reduced. 

Over non-perfect fields of characteristic $p$ (for example function fields or 
local fields) things can be even worse. The following example over 
$k=\dF_p(t)$ is due to Tits. Let 
$G=\{x+t x^p+y^p=0\}\subset (\dG_{\additive,k})^2$. One can check that $G$ is 
smooth and connected. However, if $p\geqslant 3$, then 
$G(k)=\{(0,0)\}$. In contrast, over an infinite perfect field $F$, the set 
$G(F)$ is Zariski dense in $G$ for all affine connected $G$. To see that 
$G$ is smooth, base-change to $k(\sqrt[p] t)=\dF_p(t')$. In that field, 
\[
  x+t x^p+y^p = x+(t'x)^p + y^p = x+(t'x+y)^p ,
\]
so $\dG_{k(\sqrt[p] t)}\simeq \dG_{\additive,k(\sqrt[p] t)}$. Now we show 
that $G(k)=1$. If there is $(x,y)\in G(k)$ with $x=0$, then one easily sees 
that $y=0$. Suppose $x(t)=\frac{P(t)}{Q(t)}$. We can assume $P$ and $Q$ are 
relatively prime. Some simple algebra gives $P' Q-P Q' + P^p Q^{2-P} = 0$, 
which cannot be. 





\section{Generalities}

The theory over positive characteristic is already complicated enough to make 
it no additional effort to work over an arbitrary commutative ring. For 
example, if $k'/k$ is a purely inseparable extension of the form 
$k(\sqrt[p] a)$, then $k'\otimes_k k' = k'[t]/(t-\sqrt[p] a)^p$ is not even 
a product of fields. But it is an Artinian local ring. One could restrict to 
this (as is done in \cite[VIa]{sga3}). 

For the rest of this section, fix a (commutative, unital) ring $R$. An 
\emph{$R$-functor} is a covariant functor 
$F:R\algebras\to \mathsf{Set}$. If $X$ is an affine scheme over $R$, we write 
$R[X]=\Gamma(X)$; note that $X=\spectrum(R[X])$. Let 
$h_X(S)=X(S)=\hom_{R\algebras}(R[X],S)$. We say that an $R$-functor $F$ is 
\emph{representable} by an affine scheme $X$ over $R$ if there is an 
equivalence $h_X \isomorphism F$. The Yoneda Lemma tells us that such an $X$ is 
unique up to unique isomorphism. Even better, 

\begin{lemma}[Yoneda]
The map $\hom_{R\text{-}\mathsf{Fun}}(h_X,F) \to F(X)$ that sends 
$\eta$ to $\eta_X(1_X)$ is a bijection. 
\end{lemma}

In light of this, we will usually define group schemes via their functor of 
points. It makes sense to speak of ``group-valued $R$-functors,'' and we can 
define an \emph{affine group scheme over $R$} to be an affine scheme $X$, 
together with the structure of a group-valued functor on $h_X$. In other words, 
for each $S$, we have a ``multiplication'' $X(S)\times X(S) \to X(S)$, 
``inverse'' $X(S) \to X(S)$, and ``identity'' $1\in X(S)$. These are required 
to be compatible with maps $S\to S'$ in the obvious way. To give $X$ the 
structure of a group scheme is equivalent to giving $R[X]$ the structure of a 
Hopf algebra. 

\begin{example}[constant groups]
Let $\Gamma$ be a finite (abstract) group. Let 
$G=\coprod_{\sigma\in \Gamma}\spectrum R$, with multiplication induced by 
$\Gamma$. People sometimes write $G=\Gamma_R$, or $G=\underline\Gamma_R$. The 
same definition works if $\Gamma$ is not finite, but $\underline\Gamma$ will no 
longer be affine. 
\end{example}

\begin{example}[vector groups]
For $N$ an $R$-module, define $\dV(N)(S)=\hom_S(N\otimes_R S,S)$. This is 
clearly a $R$-group functor, represented by 
$\spectrum(\symmetric_R(N^\vee))$. 
\end{example}

\begin{example}
If $M$ is an $R$-module, we put $\dW(M)(S)=M\otimes_R S$. This is not always 
representable. If $R$ is noetherian, $\dW(M)$ is representable if and only if 
$M$ is locally free. 
\end{example}

\begin{example}[linear groups]
Let $A$ be an (associative) $R$-algebra. We define 
$\generallinear_1(A)(S) = (A\otimes_R S)^\times$. This functor is not always 
representable. However, if $A$ is finitely-generated, locally free (as an 
$R$-module), then $\generallinear_1(A)$ is representable. Sometimes we will 
just write $A^\times$ for $\generallinear_1(A)$ (especially if $A$ is a 
division ring over a field). 
\end{example}

\begin{example}[diagonalizable groups]
Let $M$ be an (abstract) commutative group. Let $R[M]$ be the corresponding 
group ring. This is naturally a Hopf algebra via 
$\Delta(m)=m\otimes m$. Put
$\diagonal_R(M)=\spectrum(R[M])$. For example, if $M=\dZ$, then 
$R[M]=R[t^{\pm 1}]$, so $\diagonal_R(M)=\dG_{\mult, R}$. If 
$M=\dZ/n$, then $R[M]=R[t]/(t^n-1)$, so 
$\diagonal_R(M)=\mu_{n,R}$. A homomorphism 
$f:M_1 \to M_2$ induces $f_\ast:R[M_1] \to R[M_2]$, hence 
$f^\ast:\diagonal_R(M_2) \to \diagonal_R(M_1)$. 
\end{example}

\begin{exercise}
The functor 
$\diagonal_R:\mathsf{Ab}^\circ \to \{\text{diagonalisable groups over }R\}$ is 
an equivalence of categories. [See SGA 3, exp.8]
\end{exercise}

If $D$ is a group scheme, put $\widehat D=\hom(D,\dG_\mult)$; this is the group 
of \emph{characters} of $D$. 

\begin{exercise}
Show that $\hom(\diagonal(M), \dG_{\additive,R}) = 0$ for all $M$. 
\end{exercise}

If $R=k$ is a field, then any closed subgroup $G\subset \diagonal(M)$ is of 
the form $\diagonal_k(M/M')$ for some $M'\subset M$. In other words, over a 
field, subgroups (and qotients, in fact) of diagonalisable groups are 
diagonalisable. 





\section{Basic facts on affine groups over \texorpdfstring{$k$}{k}}

Throughout, let $k$ be a field. 

\begin{definition}
An \emph{affine algebraic group over $k$} is a group scheme $G$ over $k$, such 
that $G$ is affine of finite type. 
\end{definition}

\begin{definition}
An affine algebraic group over $k$ is \emph{linear} if it is also smooth. 
\end{definition}

It is a non-trivial fact that if $G$ is an affine group over $k$, then there is 
a closed embedding $G\hookrightarrow \generallinear_n$. There is a paper by 
[Godeles, Popov, 2003]. Supposing $k$ is infinite and $G$ is smooth, there 
exists $G\hookrightarrow \generallinear(V)$ and 
$f\in V^\vee\otimes_k V^\vee\otimes_k V$, such that 
$G=\{g\in \generallinear(V):g^\vee\circ f = f\}$. Moreover, there exists a 
(possibly non-associative, non-unital) $k$-algebra $A$ such that 
$G=\automorphism(A)\subset \generallinear_k(A)$. (The two statements are the 
clearly equivalent.) 

If $G$ an affine algebraic group over $k$, then the associated reduced scheme 
$G_\reduced$ is not necessarily a $k$-group. Essentially, the problem is that 
$(-)_\reduced$ does not commute with products. If $k$ is perfect, then 
$G_\reduced\times_k G_\reduced$, and $G_\reduced\subset G$ is a closed 
$k$-subgroup of $G$. 

\begin{example}
Let $G=\mu_3\rtimes \dZ/2$ over $\dF_3$, the action being via inversion. Then 
$G_\reduced=\dZ/2$ is not a normal subgroup. 
\end{example}

\begin{lemma}
Let $G$ be an affine $k$-group. Then the following are equivalent: 
\begin{enumerate}
  \item $G$ is smooth 
  \item $G$ is geometrically reduced 
  \item $\sO_{G,e}\otimes \bar k$ is reduced 
  \item $G$ admits a nonempty smooth open $U\subset G$. 
\end{enumerate}
\end{lemma}
\begin{proof}
The only nontrivial part is 4$\Rightarrow$1g. But one can base-change to 
$\bar k$, and note that $G_{\bar k}=\bigcup_{g\in G(\bar k)} g U_{\bar k}$. 
\end{proof}

\begin{proposition}
Let $G$ be an affine $k$-group. Let $\{f_i:V\to G\}$ be morphisms from 
geometrically reduced affine $k$-schemes. Then there is a unique smallest 
closed $k$-subgroup, written $\Gamma_G(\{f_i\})$, such that the $f_i$ 
factor through $\Gamma_G(\{f_i\})$. 
\end{proposition}
\begin{proof}
First we suppose we have a single $f:H\to G$, where $H$ is a smooth group. 
Then the schematic image $\image(f)\subset G$ works. 

Next, we suppose that $G$ admits a maximal smooth $k$-subgroup $G^\dag$. (If 
$k$ is perfect, then $G^\dag=G_\reduced$.) 
\end{proof}

If $f:G\times G\to G$ is 
$(g,h)\mapsto [g,h]$, then $\Gamma_G(f) = \sD G$, the derived subgroup of 
$G$. 


\subsection{Maximal smooth subgroups}

Let $k$ be a field, $A$ a $k$-algebra. Recall we say that $A$ is 
\emph{geometrically reduced} (Bourbaki says \emph{separable}) if 
$A\otimes_k K$ is reduced for all extensions $K/k$. If $A$ is of finite 
type, one only needs $A\otimes_k \bar k$ to be reduced. 

If $A$ is of finite type and geometrically reduced, then for $X=\spectrum A$, 
\begin{enumerate}
  \item $X$ is generically smooth over $k$ (i.e.\ there exists an 
    affine dense open smooth subscheme of $X$ over $k$)
  \item $X(k^s)$ is dense in $X$ 
\end{enumerate}

Now, as before, if $G$ is an affine algebraic $k$-group, then it admits a 
maximal smooth $k$-subgroup $G^\dagger$. This subgroup enjoys the following 
properties: 
\begin{enumerate}
  \item If $G$ is connected, so is $G^\dagger$. 
  \item $G^\dagger$ is the largest geometrically reduced closed $k$-subscheme of $G$. 
  \item $G^\dagger(k) = G(k)$. 
  \item If $k$ is separably closed, $G^\dagger$ is the schematic closure of 
    $G(k^s)$ in $G$. 
  \item If $k$ is perfect, then $G^\dagger = G_\reduced$. 
  \item If $K/k$ is a separable field extension, then 
    $(G^\dagger)\times_k K \isomorphism (G_K)^\dagger$. 
\end{enumerate}

We will discuss 2 and 3 in greater depth. For 2, suppose 
$i:X\hookrightarrow G$ and $X$ is affine and geometrically reduced. We can 
attach to this data a smooth group $\Gamma_G(i)\subset G$. But 
$\Gamma_G(i)\subset G^\dagger$, so $X\subset G^\dagger$. 

Regarding 3: the inclusion $G^\dagger(k)\subset G(k)$ is obvious. Let 
$g\in G(k)$. Then $X=g G^\dagger\subset G$ is also smooth, whence 
$g G^\dagger\subset G^\dagger$, which yields $g\in G(k)$. 

\begin{example}
Here we show that $(-)^\dagger$ does not necessarily commute with inseparable 
base-change. Let $k=\dF_p(t)$, $G=\{x^p+t y^p=0\}\subset (\dG_{\additive,k})^2$. 
We claim that $G^\dagger=0$, but $(G_{k'})^\dagger = \dG_{\additive, k'}$ (hence 
$(G_{k'})^\dagger \ne 0$) if $k'=k(\sqrt[p] t)$. 
First we show that $G^\dagger=0$. It suffices to show that $G(k^s)=0$. Let 
$(x,y)\in G(k^s)$ be nonzero. But $x^p+t y^p=0$ tells us that $t$ is a 
$p$-power in $k^s$, a contradiction. Over $k'$, the equation defining $G$ 
is $(x+t' y)^p=0$, which clearly has maximal smooth subscheme
$\{x+t' y=0\} \simeq \dG_{\additive,k'}$. 
\end{example}


\begin{proposition}
Let $f:H\to G$ be a homomorphism of affine algebraic $k$-groups. Assume that 
$G$ is reduced. Then the following are equivalent:
\begin{enumerate}
  \item $f$ is faithfully flat
  \item $f$ is surjective on $\bar k$-points
  \item $f$ is dominant 
\end{enumerate}
\end{proposition}

Recall that a ring homomorphism $f:A\to B$ is \emph{flat} (or that $B$ is flat 
over $A$) if the tensor-product functor $-\otimes_A B$ is exact. We say that $B$ 
is \emph{faithfully flat} if it is flat over $A$, and moreover the map 
$\spectrum B \to \spectrum A$ is surjective. 

\begin{proposition}
Let $f:H\to G$ be a homomorphism of algebraic $k$-groups. Then the following are 
equivalent: 
\begin{enumerate}
  \item $f$ is a closed immersion
  \item $f$ is an immersion 
  \item $f$ is a monomorphism (i.e.\ for all $A\in k\algebras$, the map 
    $f_\ast:H(A) \to G(A)$ is injective)
\end{enumerate}
\end{proposition}

This fails over rings that are not fields (e.g.\ the ring $\dZ_2$ of $2$-adic 
integers). One might ask whether injectivity on $\bar k$-points suffices. The 
answer is no. Let $F:\dG_\additive \to \dG_\additive$ be Frobenius over a field 
of characteristic $p$. It is not an immersion, but is injective on 
$\bar k$-valued points. If we take $\dF_p[t]/(t^p)$-valued points, then 
Frobenius is no longer injective. One puts 
$\boldsymbol\alpha_p = \spectrum(\dF_p[t]/t^p)$. 





\section{Using group functors}

Throughout, let $R$ be our base ring. We are given three $R$-group functors 
$G_1$, $G_2$, $G_3$, and maps 
$\alpha:G_1 \to G_2$, $\beta:G_2 \to G_3$. We say that the sequence of $R$-group 
functors 
\[
  1 \to G_1 \to G_2 \to G_3 \to 1 
\]
is \emph{exact} if for all $S\in R\algebras$, the sequence 
\[
  1 \to G_1(S) \to G_2(S) \to G_3(S) \to 1 
\]
is exact as a sequence of abstract groups. 

Let $f:G\to H$ be a morphism of $R$-group functors. We define an $R$-group 
functor $\ker(f)$ by $\ker(f)(S)=\ker(G(S) \to H(S))$. There is an exact sequence 
\[
  1 \to \ker(f) \to G \to H .
\]
If $G$ and $H$ are (representable by) affine algebraic groups, so is $\ker(f)$. 

\begin{example}[Witt vectors]
Over $\dF_p$, we have seen that there is an exact sequence 
$0 \to \dG_\additive \to W_2 \to \dG_\additive \to 0$. In fact, this sequence 
splits on the right. 
\end{example}

Note that if $1 \to G_1 \to G_2 \xrightarrow f G_3 \to 1$ is an exact 
sequence of affine algebraic groups, then $f:G_2 \to G_3$ admits a section. 
To see this, just look at the surjection 
$G_2(R[G_3]) \twoheadrightarrow G_3(R[G_2])$. 


\subsection{Semidirect products of group schemes}

Suppose $G$ and $H$ are affine $R$-group schemes, and 
$\theta:G\to \automorphism(H)$ is a morphism of $R$-group functors. Here, 
$\automorphism(H)$ is defined by 
\[
  \automorphism(H)(S) = \automorphism_{S\textnormal{-}\mathsf{gp}}(H_S) .
\]
So for each $S$, we have a homomorphism 
$\theta_S:G(S) \to \automorphism(H_S)$. We can define the semidirect product 
(on the scheme $H\times G$) by 
$(h_1,g_1) \cdot (h_2,g_2) = (h_1 \theta(g) h_2, g_1 g_2)$ as usual. This 
gives us a group scheme $H\rtimes^\theta G$ that fits into an exact sequence of 
$R$-functors:
\[
  1 \to H \to H\rtimes^\theta G \to G \to 1 .
\]
For example, we earlier saw the scheme $\boldsymbol\mu_n\rtimes \dZ/2$. Also, 
one can form $\dG_\additive^n\rtimes \generallinear_n$. 

Let $A$ be a commutative affine $R$-group scheme, and $G$ an $R$-group scheme 
with an action on $A$ via $\theta:G\to \automorphism(A)$. We wish to classify 
$R$-group extensions of $G$ by $A$ with respect to this action. It turns out 
that there is a group $\h^2(G,A)$ (second Hochschild cohomology) which does this. 

Let's look at the special case where $A=\dV(N)$ for $N$ an $R$-module. 

\begin{theorem}[Grothendieck]
If $G=\diagonal_R(M)$ acts on $\dV(N)$, then $\h_0^i(G,\dV(N))=0$ for all 
$i>0$. In particular, extensions of$G$ by $\dV(N)$ are split. 
\end{theorem}


\subsection{Actions, centralizers, etc.}

Let $G$ be an affine $R$-group scheme, $X$ an affine $R$-scheme. An 
\emph{action} of $G$ on $X$ is a homomorphism 
$\theta:G\to \automorphism(X)$, where $\automorphism(X)$ is the $R$-group 
functor defined by 
\[
  \automorphism(X)(S) = \automorphism_{S\textnormal{-}\mathsf{Sch}}(X_S) .
\]
In other words, for each $S$, we have a homomorphism 
$\theta_S:G(S) \to \automorphism_{S\textnormal{-}\mathsf{Sch}}(X_S)$. 

\emph{Warning}: the functor $\automorphism(X)$ is very rarely representable. 
In characteristic zero, $\automorphism(\dG_\additive)$ is represented by 
$\dG_\mult$, i.e.\ for all $\dQ$-algebras $S$, the natural map 
$S^\times \to \automorphism_{S\textnormal{-}\mathsf{gp}}(\dG_{\additive,S})$ is 
an isomorphism. But in characteristic $p>0$, 
$\automorphism_{S\textnormal{-}\mathsf{gp}}(\dG_{a,S})$ consists of 
non-commutative polynomials $a_0+a_1 t^p + \cdots + a_r t^{p^r}$, with 
$a_0\in S^\times$ and the $a_i$ nilpotent. 

Let $G$ act on $X$, and let $X_1,X_2\subset X$ be closed $R$-subschemes. We 
define various functors: 
\begin{align*}\tag{transporter}
  \transporter_G(X_1,X_2)(S) &= \{g\in G(S):\theta(g)X_1(S')\subset X_2(S')\text{ for all }S'/S\} \\  \tag{strict transporter}
  \stricttransporter_G(X_1,X_2)(S) &= \{g\in G(S):\theta(g)\text{ induces }X_{1,S}\isomorphism X_{2,S}\} \\ \tag{normalizer}
  \normalizer_G(X_1) &= \transporter_G(X_1,X_1) \\ \tag{centralizer} 
  \centralizer_G(X_1) &= \{g\in G(S):\theta(g) = 1\text{ on }X_{1,S}\} .
\end{align*}

\begin{theorem}[Grothendieck]
Let $R=k$ be a field, $X$ an affine scheme, $G$ an affine algebraic $k$-group 
acting on $X$, and $X_1,X_2\subset X$ closed $k$-subschemes. Then all the above 
functors are representable by closed $k$-subschemes of $G$. 
\end{theorem}

\begin{example}
Consider the action of $G$ on itself by inner automorphisms. We see that if 
$H\subset G$ is a $k$-subgroup, then 
$\normalizer_G(H)$ and $\centralizer_G(H)$ exist as subgroup schemes of $G$. 
\end{example}

The functors we have defined are the \emph{schematic} transporter, normalizer, 
\ldots. These can be different from the classical objects of the same name. 

Let $G$ be a reductive group, $P\subset G$ a parabolic subgroup, and 
$U\subset P$ its unipotent radical. Classically, one says that 
$P=N_G(U)$. This is true in the language of reduced varieties. 

\begin{example}
Let $k$ be a field of characteristic $2$ ($\overline{\dF_2}$ for example). 
Let $G=\projectivegenerallinear(2)_k$, $B\subset G$ the standard Borel, and 
$U\subset B$ the unipotent radical. We claim that 
$B\subsetneq N_G(U)$. Indeed, writing $\smat a b c d$ for projective 
coordinates of $G$, we have $B=\{c=0\}$. But $J=\{c^2=0\}$ is an 
$R$-subgroup of $G$ strictly containing $B$, but $J\subset N_G(U)$. Indeed, 
if $x^2=0$, then 
\[
  \smat{1}{}{x}{1} \smat{1}{b}{}{1} \smat{1}{}{x}{1} = \smat{1+b x}{b}{}{1+b x} = \smat{1}{b(1+x)}{}{1} 
\]
within $\projectivegenerallinear(2)$. 
\end{example}


\subsection{Weil restriction}

Recall the extension $\dC/\dR$ given by adding a root of $z^2=2+i$. 
Writing $(x+i y)^2 = 2+i$, we get two equations with coefficients in $\dR$. 
This corresponds to Weil restriction. 

Let $R\to S$ be a ring extension. Let $F$ be an $S$-functor. Following 
Grothendieck (Weil writes $\operatorname{R}_{S/R}$) we define an $R$-functor 
$\Pi_{S/R} F$ by 
\[
  (\Pi_{S/R} F)(R') = F(R'\otimes_R S) .
\]
If $F$ is representable, it is natural ask under what conditions the 
restriction of scalars $\Pi_{S/R} F$ is representable. 

\begin{theorem}
Assume $R\to S$ is finite and locally free. Let $Y$ be an affine $S$-scheme. 
Then $\Pi_{S/R} Y$ is representable by an affine $R$-scheme. Moreoveer, if 
$Y$ is of finite presentation, so is $\prod_{S/R} Y$. 
\end{theorem}

\begin{example}[vector groups]
If $N$ is an $S$-module, then recall $\dV(N)(T) = \hom(N\otimes_S T,T)$. Let 
$M$ be the ``scalar restriction'' of $N$ to $R$ (i.e.\ $N$ considered as an 
$R$-module). Then $\Pi_{S/R} \dV(N) = \dV(M)$. 
\end{example}

If $S=\overbrace{R\times \cdots \times R}^d$, then an $S$-scheme $Y$ is just a 
$d$-tuple of $R$-schemes $Y_1,\dots,Y_d$. One has 
$\Pi_{S/R} Y = Y_1\times_R \times \cdots \times_R Y_d$. 

Weil restriction does not transform open covers into open covers! Consider 
$G=\dG_{\additive, \dC\times \dC}$. Then 
$\Pi_{\dC\times \dC/\dC} G = \dG_\additive \times \dG_\additive$. If 
$U_0=\{t\ne 0\}$ and $U_1=\{t\ne 1\}$, then 
$\{\Pi U_0,\Pi U_1\}$ is \emph{not} an open cover of $\dG_\additive^2$. 

The Weil restriction functor takes affine group schemes to affine group 
schemes. (That $\Pi_{S/R}$ preserves smoothness follows trivially once we 
know that formally smooth $\Rightarrow$ smooth.) However, as we have seen, 
it does \emph{not} preserve surjectivity. For example, let $k$ be a field 
of characteristic $p>0$, and let $F:\dG_{\additive,k} \to \dG_{\additive,k}$ be the 
Frobenius $x\mapsto x^p$. Let $k'/k$ be a purely inseparable extension. We have 
an induced morphism $\Pi_{k'/k} \dG_\additive \to \Pi_{k'/k} \dG_\additive$. 
Extend scalars once again via $k'\otimes_k k' \supset k'$. The ring 
$k'\otimes_k k'$ is a local Artinian ring over $k$. In the 
commutative diagram 
\[\xymatrix{
  \text{infinitesimal} \ar[r] \ar[d] 
    & \Pi_{k'\otimes k' / k'} \dG_\additive \ar[r] \ar[d] 
    & \dG_\additive \ar[d] \\
  \text{infinitesimal} \ar[r] 
    & \Pi_{k'\otimes k' / k'} \dG_\additive \ar[r] 
    & \dG_\additive
}\]
The first vertical arrow is not surjective. 

\begin{example}
Let $p=2$, and let $k'=k(\sqrt a)$ be a purely inseparable extension. 
Then $k'\otimes_k k'\simeq k'[t]/(t^2)$. Write $A=k[t]/t^2$. Then for 
$X/k$ affine, $Y=\Pi_{A/k} X_A$ is the tangent bundle of $X$. 

[\ldots more that I didn't understand\ldots ]
\end{example}

Again, let $k'/k$ be a purely inseparable extension. The functor 
$\Pi_{k'/k}$ transforms affine covers into affine covers. Moreover, 
$\Pi_{k'/k}$ extends to a functor on quasi-compact separated 
$k'$-schemes. Let $G=\Pi_{k'/k} \dG_{\additive,k'}$; this is an 
interesting group. If $[k':k]=p^r$, then $G$ embeds into 
$\generallinear_{p^r} = \generallinear(k')$. This is one of the simplest 
possible examples of a pseudo-reductive group. It is not a torus -- the 
base-change $G_{k'}$ contains a unipotent part $\dG_{a,k'}$.





\section{Descent and quotients}

Throughout, we will use faithfully flat descent as a kind of ``black box'' to 
prove certain results. Vistoli's notes are a fantastic reference to this 
topic. 


\subsection{Embedded descent}

Let $k$ be a field, $K/k$ a field extension. Let $X_0$ be an affine $k$-scheme 
of finite type. Write $X=X_0\times_k K$. We are given a closed subscheme 
$Z\hookrightarrow X$. We say that $Z$ \emph{descends} to $k$ if there is a closed 
subscheme $Z_0\subset X_0$ such that $Z=Z_0\times_k K$. If such a $Z_0$ exists, 
it is unique. Indeed, to define $Z_0\subset X_0$, it is equivalent to 
define the associated ideal $I_{Z_0}\subset k[X_0]$. If $Z_0$ and $Z_0'$ are 
both descents of $Z$ to $k$, then we have $I_{Z_0}$ and $I_{Z_0'}$, ideals 
in $k[X_0]$ such that $I_{Z_0}\otimes_k K=I_{Z_0'}\otimes_k K$. This implies 
$I_{Z_0} = I_{Z_0'}$ (special case of faithfully flat descent). 


\subsection{The Galois case}

Suppose $K/k$ is a (possibly infinite) Galois extension. Then $Z$ descends to 
$k$ if and only if for all $\gamma\in \galois(K/k)$, we have 
$^\gamma(I_Z) = I_Z$ in $K[X_0] = K\otimes_k k[X_0]$. One direction is easy, 
and for the other one uses Galois descent (via Speiser's lemma). Define 
$I_0=\h^0(k, I_Z)$. Speiser's lemma tells us that $K\otimes_k I_0 = I_Z$. 

\begin{corollary}[Galois case]
There exists a minimal ``field of definition'' of $Z$.
\end{corollary}
\begin{proof}
The field of definition $F$ is by definition the field $F\subset K$ for which 
$\galois(K/F)=\{\gamma\in \galois(K/k):\gamma(I_Z)=I_Z\}$. 
\end{proof}

In [EGA 4, \S 4.8] there is a section called ``field of definition.'' It 
depends on Bourbaki's Algebra II.8. Let $V_0$ be a $k$-vector space, 
$V=K\otimes V_0$. Let $W\subset V$ be a $K$-vector space. Then there is a 
minimal field of definition for $W$. The proof is an exercise in linear 
algebra. 

Back to a general field extension $K\supset k$. For $Z\subset X$, we have 
a field of definition $F$ of the subspace $I_Z\subset K[X_0]$. If $F+k$, 
then $I_Z=(I_0)\otimes_k K$ for a $k$-space $I_0$ which defines $Z_0$. 

If $K^p\subset k\subset K$ (i.e.~$K$ is a height $\leqslant 1$ extension of 
$k$) there are some tools, due to Cartan and Jacobson, to replace Galois theory. 


\subsection{Flat sheaves}

The notion of ``exact sequence of algebraic groups'' that we defined earlier is 
too restrictive to be very useful. The correct context is that of flat 
sheaves. 


An fppf (or flat for short, from ``fid\`element plat de pr\'esantation finie,'' or 
``faithfully flat of finite presentation'') cover is a ring map $R\to S$ which 
is faithfully flat and of finite presentation. In other words, $S$ is the 
form $R[t_1,\dots,t_n] / (g_1,\dots,g_m)$. We say that an $R$-functor 
$F:R\algebras\to \mathsf{Set}$ is an \emph{fppf sheaf} if 
\begin{enumerate}
  \item $F(S_1\times S_2)\isomorphism F(S_1)\times F(S_2)$ 
  \item For each flat cover $S\to S'$, the following sequence is exact: 
    \[
      F(S) \to F(S') \rightrightarrows F(S'\otimes_S S') .
    \]
    The two maps $F(S') \to F(S'\otimes_S S')$ come from the two 
    maps $S'\to S'\otimes_S S'$. By ``exact'' we mean that the sequence is 
    an equalizer. 
\end{enumerate}

Zariski covers (and \'etale covers as well) are fppf. Write 
$\sheaf(R_\fppf)$ for the category of fppf sheaves over $R$. 

\begin{example}
Our groups $\dV(N)$ are fppf sheaves. More generally, for any affine $R$-scheme 
$X$, the representable functor $h_X$ is fppf due to the following result. 
\end{example}

\begin{proposition}[Grothendieck]
If $R\to S$ is faithfully flat, then the sequence 
\[
  0 \to R \to S \to S\otimes_R S \to S\otimes_R S \otimes_R S \to \cdots
\]
is exact. 
\end{proposition}

Just as with sheaves on topological spaces, there is an operation of 
``fppf sheafification'' for $R$-functors. It is a functor 
$R\textnormal{-}\mathsf{Fun}\to \sheaf(R_\fppf)$. 

Let $f:F\to F'$ be a morphism of flat fppf sheaves (in groups). We can define 
$\ker(f)$ ``pointwise'' as before: 
\[
  \ker(f)(S) = \ker(f_S:F(S) \to F(S')) .
\]
This is a flat sheaf. However, we also want to define the image 
$\image(f)$ and coimage $\coimage(f)$. These are the sheafifications of 
$S\mapsto \image(F(S)\to F'(S))$ and $S\mapsto F(S)/\ker(f)(S)$, 
respectively. Fortunately, it is still the case that the natural map 
$\coimage(f) \to \image(f)$ is an isomorphism. We say that a sequence 
$1 \to F_1 \to F_2 \to F_3 \to 1$ of fppf sheaves (of groups) is \emph{exact} 
if 
\begin{enumerate}
  \item for all $R\to S$, the sequence $1\to F_1(S) \to F_2(S)\to F_3(S)$ is exact 
  \item for all $R\to S$ and all $\alpha\in F_3(S)$, there is a flat cover 
    $S'/S$ such that $\alpha\in \image(F_2(S') \to F_3(S'))$. 
\end{enumerate}

Instead of property 2 we could have required that the natural map 
$\image(F_2\to F_3) \to F_3$ be an isomorphism. 

\begin{example}
The sequence $1 \to \boldsymbol\mu_n \to \dG_m \xrightarrow n \dG_m \to 1$ is 
not an exact sequence of $R$-functors for any $R$. However, it is fppf-exact. 
This is easily seen via lifting over the fppf cover 
$S\to S[\sqrt[n] s]$ for $s\in S^\times$. 
\end{example}

In other words, we have an ``equality of fppf sheaves'' 
$\dG_m = \dG_m/\boldsymbol\mu_n$. 

Let $H\subset G$ be a subgroup (as closed affine $R$-schemes). Using fppf 
sheafification, we can talk about $Q=G/H$. This is, \emph{a priori}, simply 
an fppf sheaf. Is it representable by an $R$-scheme? Unfortunately, this is 
rarely the case. See \cite{r70} for a counterexample. 

\begin{theorem}
Let $k$ be a field. If $H\subset G$ are affine algebraic groups over $k$, then 
the fppf sheaf $G/H$ is representable by a quasi-projective $k$-scheme. 
\end{theorem}

Note that $(G/H)\times_k R$ will be an fppf quotient for each $R/k$. 

One annoying feature of fppf sheafification is that it is non-trivial to find 
the points of a quotient $G/H$. Fortunately, if $H\subset G$ are over a field 
$k$, then for $X=G/H$, we have $X(\bar k)) = G(\bar k)/H(\bar k)$. To see this, 
use the fact that if $\bar k\to S$ is a flat cover, there is a section 
$S\to \bar k$. Moreover, one has 
\[
  X(k) = \{[g]\in G(\bar k)/H(\bar k):d_{1,\ast}(g) = d_{2,\ast}(g)\mod H(\bar k\otimes_k \bar k)\} .
\]
If $k$ is perfect, then $X(K)=(G(k^s)/H(k^s))^{G_k}$. 

\begin{proposition}
Let $G$ be an affine algebraic group over $k$. Suppose $G$ acts on a separated 
$k$-scheme $X$ of finite type. For $x\in X(k)$, let 
$G_x=\centralizer_G(x)\subset G$ be the scheme-theoretic centralizer. Then the 
fppf quotient $G/G_x$ is representable by a quasi-projective $k$-scheme, and 
the orbit map $G\to X$, $g\mapsto g\cdot x$, induces an immersion 
$G/G_x\hookrightarrow X$. 
\end{proposition}

That this proposition implies the theorem comes from the Chevalley construction. 
Given $H\subset G$, Chevalley shows that there is a representation 
$G\to \generallinear(V)$ such that there is a point 
$x\in \dP(V)$ such that $G_x=H$. 

One indication that this approach is ``correct'' is that it generalizes. 
There is a generalization to the ring case by Grothendieck-Murre. This is 
despite the fact that there is no Chevalley theorem over general base rings. 


\subsection{Some properties}

Let $X=G/H$. Then $X\times_X H_X\to G\times_X G$ is an isomorphism. In other 
words, $G\to G/H$ is an $H$-torsor. If $G$ is smooth, so is $G/H$. Finally, 
$G\to G/H$ is affine. If $H\subset G$ is normal, then $G/H$ has the natural 
structure of a (affine) group scheme, and the sequence 
$1\to H \to G \to G/H \to 1$ is exact (as fppf sheaves). 

The converse is not true (i.e.\ if $G/H$ is affine, it is not necessarily the 
case that $H$ is normal). For example, let $T\subset \generallinear(n)$ be a 
maximal torus. The quotient $\generallinear(n)/T$ is affine, but $T$ is not 
normal. 

We call a \emph{$k$-orbit in $G=X/H$} a locally closed $k$-subscheme such that 
$Z(\bar k)$ is an $H(\bar k)$-orbit in $G(\bar k)$. 





\bibliographystyle{alpha}
\bibliography{lyon-sources}

\end{document}
