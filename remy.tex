% !TEX root = main.tex





\section{Bruhat-Tits theory}
\thanksauthor{Bertrand R\'emy}





\subsection{Introduction}

Bruhat-Tits theory is the non-archimedean analogue of the theory of symmetric 
spaces for semisimple real Lie groups 

\begin{enonce}{Example}
Put $G=\speciallinear_2(\dR)$. This is a semisimple Lie group. To understand the 
structure of $G$, it is useful to use the $G$-action on its symmetric space 
$X=G/K$, where $K$ is a maximal compact subgroup. The symmetric space $X$ is 
naturally a $C^\infty$ Riemannian manifold with nice metric properties. In 
fact, it is non other than the well-known upper-half plane 
$\{z\in \dC:\Im z>0\}$. 
\end{enonce}

The starting point for Bruhat-Tits theory is a non-archimedean valued field $k$ 
(instead of the \emph{archimedean} valued field $\dR$) and $G$, a semisimple 
algebraic $k$-group. We would like there to be a useful / interesting 
$G(k)$-space $X$ such that the $G(k)$-action allows us to derive structure 
properties for the group $G(k)$ of rational points. This action should have 
certain properties:
\begin{enumerate}
  \item transitivity, or some reasonable weakening of it 
  \item non-positive curvature properties for $X$
\end{enumerate}

The group $G(k)$ and its homogeneous spaces are all totally disconnected. 
Usually, the maximal compact subgroups of $G(k)$ are \emph{not} all conjugate. 
So we really need new ideas (rather than taking a homogeneous space) to attach 
a nice metric space with $G(k)$-action to $G$. 





\subsection{Euclidean buildings}


\subsubsection{Simplicial definition}

This definition is due to Jacques Tits, from the late 1950s. See the exercises 
in \cite[IV]{b02}. A good general reference is \cite{ab08}. 

Roughly speaking, a Euclidean building is a simplicial complex covered by 
subcomplexes isomorphic to a given Coxeter tiling, with some incidence 
properties for the copies of the tiling. The ``slices'' in the building are 
called Coxeter complexes. 

Let $(W,S)$ be a Coxeter system. Then there (always) exists a polysimplicial 
complex (product of simplicial complexes) $\Sigma$, on the maximal 
cells of which $W$ acts freely and transitively. As an ordered set, 
$\Sigma=\{w W_I:I\subset S,W_I=\langle I\rangle, w\in W\}$. 

\begin{enonce}{Example}
Let $W=D_\infty$, $S=\{s_0,s_1\}$ (reflections about $0$ and $1$). Then 
$\Sigma$ is the real line tessellated by the integers. 
\end{enonce}

\begin{enonce}{Example}
This is the $\widetilde A_2$ case. Here $\Sigma$ is $\dR^2$, tiled by regular 
triangles. 
\end{enonce}

The complex $\Sigma$ can also be a spherical tiling (e.g.\ a tiling of the 
circle by similar segments) or a hyperbolic tiling. Here, we will only be 
interested in Euclidean tilings, and maybe spherical ones. 

\begin{defi}
Let $(W,S)$ be a Euclidean reflection group (i.e.\ an affine Coxeter group) with 
Coxeter tiling $\Sigma\subset \dR^r$). Then $X$ is said to be a Euclidean 
building of type $(W,S)$ if it is a polysimplicial complex covered by copies of 
$\Sigma$ (called the \emph{apartments}) so that 
\begin{enumerate}[\indent\bf SEB1)]
  \item Any two cells (called \emph{facets}) are contained in a suitable apartment. 
  \item Given any two apartments $\dA$, $\dA'$, there is a simplicial 
    automorphism $\dA\simeq \dA'$ fixing $\dA\cap \dA'$. 
\end{enumerate}
\end{defi}

\begin{enonce}{Example}
Buildings of type $D_\infty$ correspond to trees in which all vertices have
valence $\geqslant 2$. In other words, these are graphs without loops or 
leaves. Apartments are bi-infinite geodesics in the tree. 
\end{enonce}

\begin{enonce}{Example}
For $\widetilde A_2$, we now have the tiling of $\dR^2$ by regular hexagons. 
Imagine gluings of half-tilings along codimension-$1$ walls. 
\end{enonce}

We will see that the group $\speciallinear_3(\dQ_p)$ acts on an 
$\widetilde A_2$-tiling strongly transitively (i.e.\ the group action is 
transitive on the inclusions of a chamber (maximal facet) into 
apartments. In other words, for any $C\subset\dA$ and $C'\subset \dA'$, 
chambers living in apartments, there is $g\in \speciallinear_2(\dQ_3)$ such 
that $g \dA=\dA'$ and $g C=C'$. This is the substitute for homogeneity of 
symmetric spaces. The main outcome of Bruhat-Tits theory is that ``to any 
reductive group $G$ over a local field $k$, is attached a Euclidean building 
$X$ on which $G(k)$ acts strongly transitively.'' (This is the so-called 
\emph{geometric half} of Bruhat-Tits theory. There is another part, which 
investigates models for $G$ over the valuation ring of $k$.) 


\subsubsection{Non-simplicial version of Euclidean buildings}

As a general convention, a \emph{local field} is a locally compact 
topological field, endowed with a non-archimedean absolute value. Such 
fields are classified: they are 
\begin{itemize}
  \item finite extensions of $\dQ_p$
  \item $\dF_q\laurent t$ for $q$ a prime power
\end{itemize}
What happens if $k$ is not discretely valued? This case is also covered by 
Bruhat-Tits theory. However, when the valuation is not discrete, the building 
is no longer a simplicial complex. But it still admits a metric, and is still 
a complete metric space with non-positive curvature whenever $k$ is complete. 
Why should we care about the case where $k$ is not discretely valued?
\begin{enumerate}
  \item Bruhat and Tits did.
  \item When studying the space of (linear) representations of a given finitely-generated 
    discrete group $\Gamma$: 
    $X_n(\Gamma)=\{\varphi:\Gamma \to \generallinear(\dR)\}/\text{conj}$, 
    then $\varphi:\Gamma \to \generallinear_n(\dR)$ corresponds to an action of 
    $\generallinear_n(\dR)/\specialorthogonal(n)=X$. There exists a compactification 
    procedure for $X_n(\Gamma)$, such that the added points at $\infty$ 
    correspond to $\Gamma$-actions on non-simplicial buildings. 
  \item There is a connection between Bruhat-Tits theory over arbitrary valued 
    fields and analytic geometry in the sense of Berkovich. 
\end{enumerate}


\subsubsection{Metric properties of buildings}

We would like to motivate the axioms SEB1 and SEB2. Essentially, they 
exist in order for us to be able to define a metric on the whole building. We 
already have a natural Euclidean metric on each apartment (because they are, by 
definition, subsets of Euclidean spaces). We want to have a metric on $X$ which 
gives the Euclidean one by restriction to any apartment. Essentially, the 
axioms SEB1-2 are ``desinged'' for the local metrics to patch together. For any 
$x,x'\in X$, choose an apartment $\dA\ni x,x'$. We can define the distance 
between $x$ and $x'$ from the metric on $\dA$. This is well-defined by the second 
axiom (after some work). 

\begin{theo}[Bruhat-Tits]
Any Euclidean building $X$ admits a distance $d$ such that $(X,d)$ is a 
complete, $\cat(0)$ metric space. 
\end{theo}

The property ``$\cat(0)$'' essentially captures ``non-positively curved and 
simply connected.'' More precisely, a metric space $(X,d)$ is said to be 
$\cat(0)$ if 
\begin{enumerate}
  \item it is \emph{geodesic} (for all $x,x'\in X$, there is a continuous 
    path $\gamma:[0,d(x,x')] \to X$ such that $\gamma(0)=x$, $\gamma(d(x,x'))=x'$, 
    and $d(\gamma(s),\gamma(t)) = |s-t|$) 
  \item geodesic triangles as thin as in the Euclidean plane (if 
    $x,y,z\in X$, draw the geodesic triangle in $\dR^2$. The length of the 
    paths between a point and half-edge are $\leqslant$ what we get in the 
    Euclidean case)
\end{enumerate}

The notion of a $\cat(0)$ space is due to Gromov. 

\begin{lemm}[Bruhat-Tits fixed point]
Let $G$ act by isometries on a complete $\cat(0)$ metric space. If $G$ has a 
bounded orbit, then it has a fixed point. 
\end{lemm}
\begin{proof}[Proof (Serre)]
Any nonempty bounded subset in $X$ admits a \emph{unique} metrically characterized 
barycenter. 
\end{proof}

For example, whenever a compact group acts by isometries on a complete 
$\cat(0)$ space, then it admits a fixed point. 

Why is this useful? If $G$ is a reductive $k$-group, we have a Euclidean building 
$X$ with a strongly transitive action of $G(k)$. In the archimedean case, 
all maximal compacts are conjugate. This was proved by Emil Cartan by convexity 
arguments that amounted to the negative(?) curvature of the associated symmetric 
space. We can use the action of $G(k)$ on $X$ to classify maximal compact 
subgroups of $Gk$, which is useful for studying unitary representations of $G(k)$. 
Also, the Bruhat-Tits fixed point lemma is used to prove the geometric part of 
Bruhat-Tits 
theory. 

In general, if $G$ is an algebraic group over an arbitrary field $F$, in order 
to understand $G(F)$, one looks at $G(F^s)$ with its $\galois(F^s/F)$-action. 
In other words, $G(F)=\h^0(F,G)$. The idea to attach a Euclidean building to 
$G(k)$ is 
\begin{enumerate}
  \item use a field extension $L/k$ which splits $G$
  \item attach a building $X_L$ to $G(L)$
  \item try to prove that $(X_L)^{\galois(L/k)}$ is a smaller building
\end{enumerate}
To show that the space $(X_L)^{\galois(L/k)}$ is ``large enough'' we use the 
Bruhat-Tits fixed point lemma. 

For classical groups seen as fixed-point sets for involutions on linear 
groups, suitable Bruhat-Tits buildings are often fixed-point sets on 
the building of the ambient $\generallinear_n$ for the natural action of the 
involution. 










\subsection{Bruhat-Tits buildings for \texorpdfstring{$\generallinear(n)$}{GL(n)}}\label{sec:gln}

The spaces for $\generallinear(n)$ were introduced by Goldman and Iwahori 
in \cite{gi63} before the general notion of a building existed. Start with 
$\speciallinear_n(\dR)$. The Archimedean symmetric space is 
$X=\speciallinear_n(\dR)/\specialorthogonal(n)$. To generalize this, we see 
$X$ as the space of normalized scalar products on $\dR^n$. Now let $k$ be an 
ultrametric field, and let $V$ be a $d$-dimensional $k$-vector space. Let 
$\sN=\sN_V$ be the space of non-archimedean norms on $V$. Our goal is to 
show that $\sN$ is a Bruhat-Tits building. 


\subsubsection{Examples of norms}

If we are willing to make choices, this is easy. Choose an ordered basis 
$\boldsymbol e=(e_1,\dots,e_d)$ for $V$. Let $\boldsymbol c=(c_1,\dots,c_d)$ be 
an ordered $d$-tuple of real numbers. Then we have an ultrametric norm 
$\|\cdot\|_{\boldsymbol e,\boldsymbol c}$ on $V$ defined by 
\[
  \left\| \sum \lambda_i e_i \right\|_{\boldsymbol e, \boldsymbol c} = \max\{e^{c_i} |\lambda_i|:1\leqslant i\leqslant d\} .
\]
Given a norm $\|\cdot\|$ on $V$, we say that a basis $\boldsymbol e$ is 
\emph{adapted to} (or \emph{diagonalizes} $\|\cdot\|$ if there is a 
tuple $\boldsymbol c$ such that $\|\cdot\|=\|\cdot\|_{\boldsymbol e,\boldsymbol c}$. 

Weil and Goldman-Iwahori proved that if $k$ is a local field, then given 
any two norms $\|\cdot\|$, $\|\cdot\|'$, there is a basis $\boldsymbol e$ and 
parameters $\boldsymbol c$, $\boldsymbol c'$ such that 
\begin{align*}
  \|\cdot\| &= \|\cdot\|_{\boldsymbol e,\boldsymbol c} \\
  \|\cdot\|' &= \|\cdot\|_{\boldsymbol e, \boldsymbol c'} .
\end{align*}
The proof is similar to that of the classical Grahm-Schmidt theorem. Essentially, 
one uses the compactness of $\dP^d(k)$. 

We have 
\[
  \sN=\bigcup_{\boldsymbol e\text{ basis of }V} \dA_{\boldsymbol e} ,
\]
where 
\[
  \dA_{\boldsymbol e} = \{\|\cdot\|_{\boldsymbol e,\boldsymbol c}:\boldsymbol c\in \dR^d\} \simeq \dR^d .
\]
Fix a basis $\boldsymbol e$. We want to see a Euclidean reflection group act on 
$\dA_{\boldsymbol e}$. This group will be 
$S_d \ltimes \dZ^d$. If $k$ is a local field, write 
$k^\circ$ for the valuation ring of $k$, and $v:k^\times \to \dZ$ for the 
valuation. Then 
\[
  k^\circ = \{x\in k:|x|\leqslant 1\} = \{x\in k:v(x)\geqslant 0\} .
\]
The ring $k^\circ$ is a valuation ring, whose maximal ideal is denoted 
$k^+$. (R\'emy writes $k^{\circ\circ}$.) Let $\kappa=k^\circ/k^+$ be the residue 
field of $k$; since $k$ is locally compact, this is a finite field of order 
$q=p^e$. Lastly, choose a uniformiser $\pi$ satisfying $k^+=\pi k^\circ$. If 
$k=\dQ_p$, then $k^\circ=\dZ_p$, $\pi=p$, and $k^+=p \dZ_p$. If 
$k=\dF_q\laurent t$, then $k^\circ=\dF_q\llbracket t\rrbracket$ and 
$\pi=t$. 

Back to $\dA_{\boldsymbol e}$. The action by $S_d$ is given by permutation of 
the indices. For $\sigma\in S_d$, 
$\sigma\cdot \|\cdot\|_{\boldsymbol e,\boldsymbol c} = \|\cdot\|_{\boldsymbol e, \sigma(\boldsymbol c)}$, 
where $\sigma(\boldsymbol c) = (c_{\sigma(1)},\dots,c_{\sigma(d)})$. 
If $\boldsymbol m\in \dZ^d$, then $\boldsymbol m$ acts by 
$\boldsymbol m \cdot \|\cdot\|_{\boldsymbol e,\boldsymbol c} = \|\cdot\|_{\boldsymbol e,\boldsymbol c+\boldsymbol m}$.

This action of $S_d\ltimes \dZ^d$ lifts to an action by 
$N_{\boldsymbol e}$, the group of monomial matrices (i.e.\ each row and 
column has a unique nonzero entry) where the matrix inducing an isomorphism 
$\generallinear(V)\isomorphism \generallinear_n(k)$ is given by 
$\boldsymbol e$. In general, the group $\generallinear(V)$ acts on 
$\sN_V$ by precomposition, i.e. 
\[
  g \|\cdot\| = \|\cdot \|\circ g^{-1} .
\]
For this action, $N_{\boldsymbol e}$ stabilizes $\dA_{\boldsymbol e}$ and 
gives the previous action by $S_d\ltimes \dZ^d$. 
At last, the group $\generallinear(V)$ acts transitively on the set of 
subspaces $\dA_{\boldsymbol e}$. 

For $\boldsymbol m\in \dZ^d$, write $\pi^{\boldsymbol m}$ for the diagonal matrix 
$(a_{i j})$ with $a_{i i} = \pi^{m_i}$. The group 
$\pi^{\dZ^d}$ is the translation part of $S_d\times \dZ^d$. 


\subsubsection{Connection with Bruhat-Tits theory}

The following theorem is originally due to \cite{gi63}, but was reformulated by 
Bruhat and Tits. 

\begin{theo}
Assume $k$ is a local field. Then 
the quotient space of $\sN_V$ by homothety is a Euclidean building with 
apartments the $\dA_{\boldsymbol e}$ and Weyl 
group $S_d\times \dZ^{d-1}$. 
\end{theo}

For us, \emph{homothety} is the equivalence relation where two norms are identified 
whenever they are proportional. 

The identification $\dR^d\isomorphism \dA_{\boldsymbol e}$, 
$\boldsymbol c\mapsto \|\cdot\|_{\boldsymbol e,\boldsymbol c}$ identifies 
$\dA_{\boldsymbol e}/\sim$ with $\dR^d/\Delta$, where 
$\Delta=\dR\cdot (1,\dots,1)$ is the diagonal. 

Recall $g \|\cdot\| = \|\cdot\|\circ g^{-1}$. The action of diagonalizable matrices at 
least ono the apartment is fixed by the basis diagonalizing the matrix. For the case 
$d=2$, see Serre's book on trees. But be careful! In Serre's book only the vertices 
of the building (here a tree) are taken into account. The vertices correspond 
to homothety classes of $k^\circ$-lattices in $V$. 


\subsubsection{Maximal compact subgroups of \texorpdfstring{$\generallinear(V)$}{GL(V)}}

Using the Bruhat-Tits fixed point lemma, one knows that a 
compact subgroup $K\subset \generallinear(V)$ has to fix some point in 
$\sN_V/\sim$. Thus it is interesting to have a description of stabilizers of 
norms. 

\begin{prop}
We have 
\[
  \stabilizer_{\speciallinear(V)}(\|\cdot\|_{\boldsymbol e,\boldsymbol c}) = \{g\in \speciallinear(V): \log |g_{i j}| \leqslant c_j - c_i\} .
\]
where $(g_{i j})$ is the matrix of $g$ with respect to $\boldsymbol e$. 
In particular, $\stabilizer_{\speciallinear(V)}(\|\cdot\|_{\boldsymbol e,0}) \simeq \speciallinear_d(k^\circ)$. 
\end{prop}

[Warning: R\'emy wrote $|x|=e^{-v(x)}$ rather than 
$|x|=(\# \kappa)^{-v(x)}$.] 

Using concrete computations, one can show that a fundamental domain for the 
$\speciallinear(V)$ action on $\sN_V/\sim$ is given 
\[
  \{\|\cdot\|_{\boldsymbol e,\boldsymbol c}:0\leqslant c_1 \leqslant \cdots \leqslant c_d \leqslant 1\} 
\]
for $\boldsymbol e$ fixed. 

For $d=3$, a fundamental domain looks like an equilateral triangle. For 
$d=2$, a fundamental domain is an edge in the tree 
$\sN_V/\sim$. 

For $d=2$, one can color the vertices with $2$ colors so that any two neighbors 
don't have the same color (or type). The 
$\speciallinear(V)$-action is type preserving. As a conseqence, using 
the Bruhat-Tits lemma, we see that there are exactly two conjugacy classes of 
maximal compact subgroups in $\speciallinear(V)$. (For arbitrary $V$, there are 
$d$ conjugacy classes.) 

The group $\generallinear_d(k)$ is generated by diagonal matrices and elementary 
unipotent matrices. We have described the action of diagonal matrices on 
$\sN_V$. What remains is to describe the action of elementary unipotent matrices 
on $\sN_V$. First we introduce some notation. Fix a basis 
$\boldsymbol e$ of $V$. Then $E_{i j}$ is the matrix (with respect to 
$\boldsymbol e$) with zeros everywhere but the $(i,j)$-th component, where there 
is a $1$. Put $\mu_{i,j}(\lambda) = 1+\lambda E_{i,j}$ for 
$\lambda\in k$. 

\begin{prop}[``folding'' by unipotent matrices]
The intersection 
$\mu_{i j}(\lambda) \dA_{\boldsymbol e} \cap \dA_{\boldsymbol e}$ is the half-space 
of $\dA_{\boldsymbol e}$ defined by 
$c_j-c_i \geqslant \log |\lambda|$. The isometry given by 
$\mu_{i j}(\lambda)$ fixes (pointwise) this intersection, and sends the remaining 
(open) halfspace of $\dA_{\boldsymbol e}$ onto a disjoint complement. 
\end{prop}

\begin{enonce}{Example}
Let $d=2$. Then our matrix is $\smat{1}{\lambda}{}{1} = \mu_+(\lambda)$. 
[\ldots stuff in terms of a picture I couldn't repeat\ldots]
\end{enonce}





\subsection{Borel-Tits' theory}

For our purposes, this is the theory which describes the abstract group structure 
of $G(k)$, for $G$ a semisimpile group and $k$ an arbitrary field; this will be done in 
purely combinatorial terms. The final statement more or less amounts to the existence 
of a spherical building on which $G(k)$ acts strongly transitively. 


\subsubsection{Algebraic terminology and conjugacy theorems}

The main references for this section are the books of Borel, Springer, and 
Waterhouse. Let $G$ be a linear algebraic group over some field $k$. We say 
that $g\in G(k)$ is \emph{unipotent} if in some (a posteriori, any) 
linear representation $\rho$, its image is unipotent (i.e.\ 
$\rho(g)-1$ is nilpotent). A group $G$ is called \emph{unipotent} if all 
its' elements are unipotent. Alternatively, we could require that any 
finite-dimensional representation have a non-zero fixed vector. 

We say that a connected subgroup $T\subset G$ is a \emph{torus} if 
$\bar k[T]\simeq \bar k[X^\ast(T)]$. If $T$ is defined over $k$, we say that 
$T$ is \emph{$k$-split} if $k[T]\simeq k[X_k^\ast(T)]$. 

Let $G$ be a connected linear algebraic group over $k$. The \emph{unipotent 
radical} of $G$ is the (unique) maximal subgroup for the following properties:
\begin{enumerate}
  \item closed 
  \item connected 
  \item unipotent
  \item normal 
\end{enumerate}

The \emph{rational unipotent radical} of $G$ is the (unique) 
maximal unipotent of subgroup satisfying all the above properties, and 
additionally being defined over $k$. We denote it by 
$\sR_{u,k}(G)$. 

We say that $G$ is \emph{reductive} if $\sR_u(G)=1$. We say that $G$ is 
\emph{pseudo-reductive} if $\sR_{u,k}(G) = 1$. The \emph{radical} of $G$ 
is defined by replacing ``unipotent'' by ``solvable'' in the above 
definition. Let $\sR(G)$ be the radical of $G$. We say that $G$ is 
\emph{semisimple} if $\sR(G)=1$. 

It is a fact that ``a group is reductive if and only if it is geometrically 
pseudo-reductive.'' 
The goal in Lie theory is to extract elementary combinatorics out of 
algebraic differential geometric situations. Here, we start with a reductive 
group $G$ defined over $k$. At least in the split case, the outcome of our 
method is a root system. The well-known situation is where $k=\bar k$. 
The main tool is the adjoint representation, given by conjugation on the 
Lie algebra. Put 
$\frakg = \lie(G)=\ker(G(k[\varepsilon]) \to G(k))$, where $k[\varepsilon]=k[X]/X^2$. 
The action of $G$ on $\lie G$ induces a representation 
$\adjoint:G\to \generallinear(\frakg)$. Let $T\subset G$ be a maximal 
$k$-torus (split, given our assumptions). Then the representation 
$\adjoint|_T$ decomposes as 
\[
  \frakg = \frakg_0 \oplus \bigoplus_{\psi\in X^\ast(T)\smallsetminus \{0\}} \frakg_\psi ,
\]
where $\frakg_\psi= \{v\in \frakg :\adjoint(t) v = \psi(t) v\}$. The non-zero 
characters occuring non-trivially in this sum are called the 
\emph{roots} and they form a root system in vector space 
$X^\ast(T)\otimes \dR$ endowed with a $W$-invariant scalar product. Here 
$W=N_G(T)/T$. (For all of this we assumed $k=\bar k$.) When 
$k=\bar k$, the root system is \emph{reduced} (i.e.\ the 
proportionality relation between two roots except opposition). 

Now back to arbitrary $k$. The idea is to replace $T$ by a ``useful'' torus 
defined over $k$ (useful in the sense that it has ``as many'' characters 
defined over $k$ as possible so that the weight decomposition of 
$\frakg$ will be as subtle as possible). More precisely, take $T$ to be a 
maximal $k$-split torus. We have a decomposition 
\[
  \frakg = \frakg_0 \oplus \bigoplus_{\psi\in X_k^\ast(T)\smallsetminus \{0\}} \frakg_\psi.
\]
This is a (possibly non-reduced) root system. We claim that this decomposition 
of the Lie algebra induces a decomposition of $G$. 

[\emph{Warning}: this assumes there is a non-central $k$-split torus $T\subset G$.]

The characters $\alpha\in X^\ast(T)\smallsetminus 0$ with $\frakg_\alpha\ne 0$ are 
called \emph{roots} with respect to $T$. Let $\Phi=\Phi(G,T)$ be the set of 
roots. 

The group $N_G(T)$ acts on $X_k^\ast(T)$ via conjugation on $T$, and it stabilizes 
$\Phi$. It acts via the finite (by rigidity of tori) quotient 
$W=N_G(T)/Z_G(T)$. The $\dR$-linear span $V=X_k^\ast(T)_\dR$ admits a $W$-invariant 
scalar product (because $W$ is finite) for which $\Phi$ is a (possibly non-reduced) 
root system. The Weyl group of the root system $\Phi$ is $W$. 

For each root $\alpha$, we denote by $U_\alpha$ the closed connected subgroup of 
$G$ whose Lie algebra is $\frakg_\alpha$. 

\begin{defi}
Let $G$ be an abstract group. Let $T$ be a subgroup and $\Phi$ a root system. We 
assume we are given for each $\alpha\in \Phi$ a subgroup $U_\alpha$ and a class 
moduo $T$, say $M_\alpha$. We say that 
$(\{U_\alpha\}_{\alpha\in \Phi},\{M_\alpha\}_{\alpha\in \Phi})$ is a 
\emph{root group datum of type $\Phi$} if the following axioms hold:
\begin{enumerate}
  \item for all $\alpha\in \Phi$, $U_\alpha\ne 1$
  \item for all $\alpha,\beta\in\Phi$, $[U_\alpha,U_\beta]\subset \langle U_\gamma:\gamma\dN\alpha+\dN\beta\rangle$
  \item if $\alpha,2\alpha\in \Phi$, then $U_{2\alpha}\subsetneq U_\alpha$
  \item for all $\alpha\in \Phi$, $U_{-\alpha}\smallsetminus 1 \subset U_\alpha M_\alpha U_\alpha$ 
  \item for all $\alpha,\beta\in \Phi$, $m\in M_\alpha$, $m U_\beta m^{-1} \subset U_{r_\alpha(\beta)}$
  \item for all choices of positive subsystems for $\Phi^+$, we have 
    $U^+ T \cap U^-=1$, where $U^+=\langle U_\gamma:\gamma \in \pm \Phi_+\rangle$ 
\end{enumerate}
\end{defi}

It is a good exercise to check these axioms for $\generallinear_n(F)$, where 
$\Phi$ is of type $A_{n-1}$. 

We now give an ``unfair summary'' of Borel-Tits theory. 

\begin{theo}[Borel-Tits '65]
Let $G$ be a reductive group over $k$. We assume that $G$ contains a non-central 
$k$-split torus, say $T$. Denote by $\{U_\alpha\}$ the corresponding root groups. 
Then $G(k)$ admits a root group datum in which the root groups are given by the 
$U_\alpha(k)$, and $T=Z_G(T)(k)$. 
\end{theo}

What is this good for? 
\begin{enumerate}
  \item The existence of a root group datum implies the existence of a weaker 
    combinatorial structure called a \emph{Tits system}. This provided a 
    uniform way to prove projective simplicity for rational points of 
    isotropic simple groups. 
  \item The axioms imply the existence of a spherical building on which $G(k)$ 
    acts strongly transitively. 
\end{enumerate}
If the group is not isotropic, things become much more difficult. 


\subsubsection{Conjugacy results and anisotropic kernel}

So far, the combinatorial objects we constructed have depended on the choice of 
a maximal $k$-split torus $T$. In fact, we have the following theorem. 

\begin{theo}[Borel-Tits '65]
Let $G$ be as above. Then 
\begin{enumerate}
  \item Maximal $k$-split tori are conjugate under the $G(k)$-action. 
  \item Minimal parabolic $k$-subgroups are $G(k)$-conjugate. In fact, 
    $G(k)$ acts transitively on inclusions $T\subset B$ of maximal $k$-split 
    tori into minimal parabolics $B$. 
\end{enumerate}
\end{theo}

We introduce some terminology. Grops of the form $Z_G(T)$ are reductive anisotropic 
over $k$. In other words, $Z_G(T)$ contains no non-central $k$-split torus. (For 
$k$ a local field, a semisimple 
reductive $k$-group is anisotropic if and only if $G(k)$ is compact.) The groups 
$Z_G(T)$ are also $G(k)$-conjugate. This conjugacy class is called the 
\emph{anisotropic kernel} of $G$. In general, semisimple groups over $k$ are 
classified by data consisting of the anisotropic kernel, together with a 
Galois action on the Dynkin diagram. For details, see Tit's Boulder lecture 
notes, or Satake's book. 


\subsubsection{Example of a non-reduced root system}

This is in the context of real Lie groups. Let $F$ be one of $\dR,\dC$, and 
the quaternions. Let $\overline{\cdot}$ be the conjugation (if $F$ is the 
complex numbers or the quaternions). Let 
$\specialunitary(n,1) = \{g\in \speciallinear_n(F):g^\ast J g=J\}$, where 
$J$ is an anti-diagonal matrix with $-1$ everywhere along the reverse 
diagonal. The Lie algebra $\mathfrak{su}(n,1)$ consists of matrices 
$\smat{w}{Z^\ast}{Z}{Y}$, where $Y$ is an $n\times n$ antihermetian matrix 
(i.e.\ $Y+Y^\ast=0$), $w\in F$, and $Z\in F^n$. 

The points of a maximal $\dR$-split torus can be described as follows: let 
\[
  A = \left\{a_t = \begin{pmatrix} \cosh t & 0 & \sinh t \\0 & 1_{n-1} & 0\\ \sinh t & 0& \cosh t \end{pmatrix} : t\in \dR\right\} .
\]
The weight space decomposition for $\frakg=\mathfrak{su}(n,1))$ under the action 
of $A$ is 
\[
  \frakg = \frakg_{-2\alpha} \oplus \frakg_{-\alpha} \oplus \frakg_0 \oplus \frakg_\alpha \oplus \frakg_{2\alpha} .
\]
The $2\alpha$ occur if and only if $F\supsetneq \dR$. Here, 
\[
  \frakg_0 = \left\{\begin{pmatrix} w & 0 & t \\ 0 & m'& 0 \\ t & 0 & w \end{pmatrix}:m\in \mathfrak{u}(n-1,F),t\in \dR\text{ and }w+\overline w=0\right\} .
\]
One has 
\[
  \frakg_\alpha = \left\{\begin{pmatrix} 0 & z\ast & 0 \\ z & 0 & -z \\ 0 & z^\ast & 0 \end{pmatrix} :z\in F^{n-1} \right\} 
\]
of dimension $d(n-1$, where $d=[F:\dR]$. Finally, 
\[
  \frakg_{2\alpha} = \left\{\begin{pmatrix} w & 0 & -w \\ 0 & 0 & 0 \\ -w & 0 & w \end{pmatrix} : w+\overline w =0\right\} ;
\]
this has dimension $d-1$. The anisotropic kernel $\frakg_\alpha$ can be arbitrarily 
large, even though the root system is the (non-reduced) root system of rank 
one called $\mathrm{BC}_1$ with roots 
$\{\pm 2\alpha,\pm \alpha\}\subset \dZ\cdot \alpha$. For 
$F=\dC$, the groups we get are $\specialunitary(n,1)$, and for 
$F=\mathbf H$, the groups are $\symplectic(n,1)$. 





\subsection{Construction of Bruhat-Tits buildings}

The Euclidean buildings we will construct will be obtained by gluing together 
copies of a Euclidean tessellation attached to the affinization of the Weyl 
group given by Borel-Tits. The gluing will be done by an equivalence relation 
which (eventually) prescribes point stabilizers. These stabilizers are defined 
by using a filtration on each root group, which comes from the valuation on the 
ground field. 


\subsubsection{Gluings and foldings}

We need:
\begin{enumerate}
  \item a model for the apartments
  \item a gluing / equivalence relation 
\end{enumerate}

Let $G$ be a semisimple grop over a local field $k$. We assume that $G$ is 
$k$-isotropic (i.e.\ it contains a non-central $k$-split torus). So 
$G(k)$ is noncompact. (This is due to Bruhat-Tits-Rousseau, Prasad.) 

First we construct the apartment. Let $T$ be a maximal $k$-split torus and 
$N=N_G(T)$. We denote by $\Sigma_\vect = X_{\ast,k}(T)\otimes\dR$, where 
$X_{\ast,k}(T)=\hom_{k\textnormal{-}\mathsf{gp}}(\dG_\mult,T)$. This should 
be seen as an analogue of $\dA_{\boldsymbol e}$. There is a (well-defined) 
affine space $\Sigma$ under $\Sigma_\vect$ admitting an action by 
$N(k)$, say $\nu:N(k) \to \operatorname{Aff}(\Sigma)$, such that 
\begin{enumerate}
  \item There exists a scalar product on $\Sigma_\vect$ such that 
    $\nu(N) \subset \operatorname{Isom}(\Sigma)$. 
  \item The vectorial part of $\nu(N)$ is the Weyl group of $\Phi$. 
  \item The translation (normal) subgroup in $\nu(N)$ has compact fundamental 
    domain. In fact, $\nu(N)$ is an affine Coxeter group. 
\end{enumerate}

This is a generalization of the action on $\dA_{\boldsymbol e}$ in 
Section \ref{sec:gln} by the permutation matrices and the diagonal 
matrices $\pi^{\boldsymbol m}$. There, we had 
$\nu(N)\simeq S_d\ltimes \dZ^{d-1}$. 

Now we want to define an equivalence relation in order to glue infinitely 
many copies of $\Sigma$. The idea is to associate a compact subgroup 
$P_x$ (a ``parahoric'' subgroup) to each point $x\in \Sigma$. This is done 
using the filtration on the root groups $U_\alpha$ given by Borel-Tits 
theory. 

One should keep the example $\speciallinear_2(\dQ_p)$ in mind. Its 
maximal torus $T$ consists of diagonal matrices $\smat{t}{}{}{t^{-1}}$, 
and this give a geodesic in the tree. The subgroup 
$\smat{1}{\fp^m}{}{1}$ fixes a ray in one direction whose ```depth'' depends 
on $m$. A group $\smat{1}{}{\fp^\ell}{1}$ fixes a ray in the opposite 
direction. Once the building is obtained, the filtration corresponds to the 
size of the fixed ray. 

If $G$ is split, everything is easy. For general $G$, things are more 
difficult. The split case is easy because there exist compatible parameterizations 
of the root groups $U_\alpha$ (all one-dimensional). One calls these 
parameterizations \emph{pinnings} (``\'epinglages'' in French). We will 
base-change to make our group split, then use a (2-step) descent argument. 

The abstract abstract context used by Bruhat-Tits consists in formalizing the 
existence of compatible filtrations $U_{\alpha,\bullet}$ on groups admitting 
an abstract root group datum. This is done in \cite{bt72}, which is basically 
just combinatorics and buildings. The existence by descent and use of integral 
forms (over $k^\circ$) for $G$ is in \cite{bt84}. 

From now on, we assume we have suitable filtrations $U_{\alpha,\bullet}$ on 
root groups. In particular, $\bigcup_{r\in \dZ} U_{\alpha,r} = U_\alpha(k)$ 
and $\bigcap_{r\in \dZ} U_{\alpha,r} = 1$. Choose $x\in \Sigma$. We define 
$N_x=\stabilizer_{N(k)}(x)$ for the $\nu$-action. For all 
$\alpha\in \Phi$, we define $U_{\alpha,x}$ to be the largest subgroup 
$U_{\alpha,r}$ such that the positive subspace associated to the affine linear form 
$\alpha+r$ contains $x$, i.e.\ $(\alpha+r)(x)\geqslant 0$. At last, we 
define $P_x=\langle N_x,U_{\alpha,x}:\alpha\in \Phi\rangle$. Then, we define the 
relation $\sim$ on $G(k)\times \Sigma$ by 
$(g,x)\sim(h,y)$ if and only if there exists $n\in N(k)$ such that 
$y=\nu(n)\cdot x$ and $g^{-1} h n\in P_n$. 

\begin{theo}[Bruhat-Tits]
The relation $\sim$ is an equivalence, and the quotient space 
$G(k)\times \Sigma/\sim$ is a building of type $\Sigma$ on which $G(k)$ acts 
strongly transitively. 
\end{theo}





\subsection{Gluings and foldings}

To sum up. ``Let $G$ be a group with a root group datum 
$((U_\alpha)_{\alpha\in \Phi},(M_\alpha)_{\alpha\in \Phi})$. Assume that 
this datum admits a compatible system of valuations. Then there exists a 
Euclidean building on which $G$ acts strongly transitively. Moreover, the model 
for the apartment is explicit.'' As mentioned above, this was by Bruhat and 
Tits in their paper \cite{bt72}. This is almost a necessary and sufficient condition. 
In other words, suitably transitive actions on buildings imply the existence of 
a root group datum with valuation. 

We briefly treat descent. Let $k$ be a local field (or possibly a more general 
valued field). Let $G$ be a semisimple group over $k$. The point is to find a 
Galois extension $F/k$ which splits $G$. Since $G_F$ is the base-change 
of a Chevalley-Demazure group scheme (which are defined over $\dZ$) we 
get a pinning, which can be used to construct the valuation on our root group 
datum. 

At this stage we have a building $\sB(X,F)$ by the previous gluing procedure. 
We have an action of $\Gamma=\galois(F/k)$ on $G(F)$. Thanks to the connection 
between the combinatorics of $G(F)$ and the geometry of $\sB(G,F)$, we can 
define an action of $\Gamma$ on $\sB(G,F)$. \emph{Hint}: let $p$ be the 
residue characteristic of $k$, and assume $G$ is simply connected. Then the 
chamber stabilizers are the normalizers of the pro-$p$ Sylow subgroups of 
$G(F)$. 

Suitably decompose $F/k$ into two extensions so that $G(k)=G(F)^\Gamma$ admits 
a root group datum with valuation so that the associated building is 
$\sB(X,F)^\Gamma$. This doesn't always work. One needs some assumptions on the 
ramification of the extension $F/k$. See \cite{bt84} for details. One may 
need to choose an intermediate extension $k\subset F'\subset F$ so that 
$G$ is quasi-split over $F'$ (i.e.~contains a Borel subgroup over $F'$). 
Equivalently, the centralizer of an $F'$-split torus is a torus. Quasi-split 
groups have no anisotropic kernel. In good cases, the extension $F/F'$ is 
unramified, and we use integral structure (group schemes over valuation 
rings). 

\begin{theo}[Bruhat-Tits]
Let $k$ be a complete, discretely-valued field with perfect residue field. Then 
the group $G$ admits a Euclidean building over $k$ on which $G(k)$ acts 
strongly transitively. 
\end{theo}

In \cite{r77}, it is proved that the building $\sB(X,-)$ behaves functorially 
with respect to the ground field. In good cases, one has 
$\sB(G,F)^\Gamma = \sB(G,k)$. 




\subsection{Compact subgroups}

This is closely related to integral structures for $G$, as well as decomposition 
theorems for $G(k)$, which are used in the study of representations of $G(k)$. 

As before, assume $G$ is simply connected and $k$ is local. Then the building 
$\sB(G,k)$ is \emph{locally finite}, i.e.~given any facet $\sigma$, there are only 
finitely many facets whose closure contains $\sigma$. For any $x\in \sB(G,k)$, the 
stabilizer $P_x$ is compact and open in $G(k)$. If $p$ is the residue characteristic 
of $k$, then $P_x$ is virtually pro-$p$. In other words, each $P_x$ contains a 
pro-$p$ subgroup of finite index. 

\begin{theo}
1. For all $x\in \sB(G,k)$, there is a smooth group scheme over $k^\circ$, 
$\sG_x$, such that $\sG_x\otimes_{k^\circ} k = G$. 

2. The set of facets whose 
closure contains $x$ is a spherical finite building $\sB_x$. 

3. $\sB_x$ is the spherical building of the (semisimple quotient of) 
$\sG_x\otimes_{k^\circ} \kappa$.  
\end{theo}

If $x\in \dA$ is a vertex, we say that $x$ is \emph{special} if 
$\stabilizer_W(x)$ is the full vectorial part of the affine Weyl group 
$W$. 

\begin{enonce}{Complement}
Up to passing to a huge non-archimedean extension $K/k$, any point 
$x\in \sB(G,k)$ can be seen as a special vertex in $\sB(G,K)$. 
\end{enonce}

\begin{enonce}[remark]{Example}
Let $x\in \sB(\speciallinear_3,\dQ_p)$ which is the barycenter of some chamber. 
Up to a totally ramified extension $K$ of degree $3$, $x$ can be seen as a special 
vertex in $\sB(\speciallinear_2,K)$. 
\end{enonce}

\begin{enonce}[remark]{Example}
Here we give a vertex which is not special. The group is $\symplectic_4$. A 
typical chamber looks like 
\[\xymatrix{
  \bullet \ar@{-}[d] \ar@{-}[dr] \\
  \bullet \ar@{-}[r] 
    & \bullet 
}\]
The vertices on the ends of the long edge are special, but the common vertex 
of the short edges is not special. 
\end{enonce}

\begin{theo}
Let $G$ be a simply connected group over a local field $k$. Then the 
$G(k)$-conjugacy classes of maximal compact subgroups of $G(k)$ are in 
bijection with vertices in the closure of any chamber in $\sB(G,k)$. 
\end{theo}

Note the contrast with the case of real groups, where all maximal compacts 
are conjugate. 


\subsection{Two decompositions}

Recall the Cartan and Iwasawa decompositions. If $G$ is a real Lie group, we 
have 
\begin{align*}
  G &= K \overline A K && \text{``Cartan decomposition''} \\
  G &= K A N && \text{	``Iwasawa decomposition''}
\end{align*}

Let $\dA$ be an apartment, and let $c\subset \dA$ be a chamber. We pick a 
vertex $o\in \overline c$. From our gluing relation, we get that walls in 
the Euclidean tesselation $\dA$ correspond to the zero-sets of the affine 
roots $\alpha+r$ for $\alpha\in \Phi$, $r\in \dZ$. Half-spaces bounded by 
walls correspond to roots (at least, when $\Phi$ is reduced). The group 
$U_{\alpha,r}$ fixes $\alpha+r$ (the latter seen as a half-space in $\dA$) 
and folds the other half of $\dA$. 

The $G(k)$-action is transitive enough on $\sB(G,k)$ to show that for any 
$x\in \sB(G,k)$, there exists $g\in G(k)$ fixing $C$, such that 
$g x\in \dA$. \emph{Hint}: draw a geodesic segmentfrom $c$ to $x$, and fold (by 
finite induction) along the walls which are 
crossed by the segments or its intermediate transforms. This is almost a proof 
of the Cartan decomposition. 

Take $K$ to be $P_0=\stabilizer_{G(k)}(o)$. Pick $g\in G(k)$. Fold $x=g\cdot o$ onto 
$\dA$. Then $g\in P_c\subset K$. Then $k^{-1} g\cdot o\in \dA$ for some $k$. 
Then use a translation in the maximal split torus corresponding to $\dA$. 

\begin{theo}
Let $G$, $k$ be as before. If $o$ is a special vertex, then 
$G(k)=K \overline T^+ K$, where $\overline T^+$ is the subsemigroup of $T$ 
(the maximal $k$-split torus corresponding to $\dA$) defined by the condition 
that for $t\in T(k)$, we have $t\in \overline T^+$ if and only if 
$t\cdot o$ is an element of the some fixed Weyl chamber in $\dA$. 
\end{theo}

We say that $K$ is special if the vertex defining it is. So $K$ is special if 
and only if $K$ contains elements lifting all the elements in the spherical 
Weyl group, i.e.\ ``$K$ contains all the non-commutative part of the affine 
Weyl group.'' 

In harmonic analysis, one looks at spaces 
$\sH(G,K)=C_c^\infty(K\backslash G(k)/K)$ of locally constant, compactly supported, 
$K$-bi-invariant functions. We say that $(G(k),K)$ is a \emph{Gelfand pair} if 
convolution gives $\sH(G,K)$ the structure of a commutative ring. The Hecke 
algebra $\sH(G,K)$ is commutative if and only if $K$ is special. 
See Macdonald's book \cite{m71}. 

There is also a $p$-adic analogue of the Iwasawa decomposition: it looks like 
$G(k)= K T(k) U^+(k)$. It is based on the same ``folding'' idea, except that 
here the foldings are stepwise required to fix the boundary at infinity of the Weyl 
chamber $Q\subset \dA$ defined by $o$ and $c$. 




\subsection{Unitary groups in three variables}

Let $F/k$ be a quadratic extension of local fields. Let $V$ be a three-dimensional 
$k$ vector space. Choose a basis $e_0,e_{\pm 1}$, and define the Hermitian form 
\[
  h(z_{-1},z_0,z_1) = z_{-1} \sigma(z_0) + z_0 \sigma(z_1) + z_1 \sigma(z_{-1}) ,
\]
where $\sigma\in \galois(F/k)\smallsetminus 1$. We want to understand the 
group $\specialunitary(V,h) = \speciallinear(V)^\sigma$, where $\sigma$ is 
(another involution) defined by 
\[
  \sigma(M) = \begin{pmatrix} 0 & 0 & 1 \\ 0 & 1 & 0 \\ 1 & 0 & 0\end{pmatrix} \prescript{\mathrm{t}}{}{M}^{-1} \begin{pmatrix} 0 & 0 & 1 \\ 0 & 1 & 0 \\ 1 & 0 & 0\end{pmatrix} .
\]
The apartment of $\speciallinear(V)$ defined by $\theta=(e_{-1},e_0,e_1)$ consists 
of the homothety classes of lattices $\sum \fp^{m_i} e_i$. We have three positive 
root groups in $\speciallinear_3$: 
\begin{align*}
  U_{1 2} &= \begin{pmatrix} 1 & \ast & 0 \\ 0 & 1 & 0 \\ 0 & 0 & 1 \end{pmatrix} \\
  U_{1,3} &= ?\\
  U_{2,3} &= ?
\end{align*}
The group $U_{1 3}$ is strongly $\sigma$-stable, and $\sigma$ interchanges 
$U_{1 2}$ and $U_{2 3}$. 

What are maximal split tori in $\specialunitary(V,h)$? Let 
\[
  \tau=\begin{pmatrix} \pi & 0 & 0 \\ 0 & 1 & 0 \\ 0 & 0 & \pi^{-1} \end{pmatrix}. 
\]
The Zariski closure of $\langle\tau\rangle$ is a maximal split torus. 
According to whether $F/k$ is ramified or not we don't get the same tree. 
For example, if $F=\dF_q\laurent X$ and $k=\dF_q\laurent{X^2}$, then for 
$\pi=X$, we have $\sigma(X)=-\sigma$. Then 
$\tau\not\in \specialunitary(V,H)(k)$, but $\tau^2$ is in this group. 
If on the other hand we look at the unramified extension 
$\dF_{q^2}\laurent X/\dF_q\laurent X$, then $\sigma(\pi)=\pi$, so 
$\tau\in \specialunitary(V,h)(k)$. 
Be careful with keeping or not all the intersections of walls with fixed point 
sets as vertices in the non-split building. When $F/k$ is ramified, 
the building is a homogeneous tree (all valences are the same). But when 
$F/k$ is unramified, the building is only semi-homogeneous (valences 
are $q$ or $q+1$). 




