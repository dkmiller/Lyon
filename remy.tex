\documentclass{article}

\usepackage{lyon-style}

\title{Bruhat-Tits theory}
\author{Bertrand R\'emy}
\date{June 9--13, 2014}

\begin{document}
\maketitle
\tableofcontents





\section*{Introduction}

Bruhat-Tits theory is the non-archimedean analogue of the theory of symmetric 
spaces for semisimple real Lie groups 

\begin{example}
Put $G=\speciallinear_2(\dR)$. This is a semisimple Lie group. To understand the 
structure of $G$, it is useful to use the $G$-action on its symmetric space 
$X=G/K$, where $K$ is a maximal compact subgroup. The symmetric space $X$ is 
naturally a $C^\infty$ Riemannian manifold with nice metric properties. In 
fact, it is non other than the well-known upper-half plane 
$\{z\in \dC:\Im z>0\}$. 
\end{example}

The starting point for Bruhat-Tits theory is a non-archimedean valued field $k$ 
(instead of the \emph{archimedean} valued field $\dR$) and $G$, a semisimple 
algebraic $k$-group. We would like there to be a useful / interesting 
$G(k)$-space $X$ such that the $G(k)$-action allows us to derive structure 
properties for the group $G(k)$ of rational points. This action should have 
certain properties:
\begin{enumerate}
  \item transitivity, or some reasonable weakening of it 
  \item non-positive curvature properties for $X$
\end{enumerate}

The group $G(k)$ and its homogeneous spaces are all totally disconnected. 
Usually, the maximal compact subgroups of $G(k)$ are \emph{not} all conjugate. 
So we really need new ideas (rather than taking a homogeneous space) to attach 
a nice metric space with $G(k)$-action to $G$. 





\section{Euclidean buildings}


\subsection{Simplicial definition}

This definition is due to Jacques Tits, from the late 1950s. See Bourbaki's 
exercises for [Lie IV]. A good general reference is the [A..., Brown]. 

Roughly speaking, a Euclidean building is a simplicial complex covered by 
subcomplexes isomorphic to a given Coxeter tiling, with some incidence 
properties for the copies of the tiling. The ``slices'' in the building are 
called Coxeter complexes. 

Let $(W,S)$ be a Coxeter system. Then there (always) exists a polysimplicial 
complex (product of simplicial complexes) $\Sigma$, on the maximal 
cells of which $W$ acts freely and transitively. As an ordered set, 
$\Sigma=\{w W_I:I\subset S,W_I=\langle I\rangle, w\in W\}$. 

\begin{example}
Let $W=D_\infty$, $S=\{s_0,s_1\}$ (reflections about $0$ and $1$). Then 
$\Sigma$ is the real line tessellated by the integers. 
\end{example}

\begin{example}
This is the $\widetilde A_2$ case. Here $\Sigma$ is $\dR^2$, tiled by regular 
triangles. 
\end{example}

The complex $\Sigma$ can also be a spherical tiling (e.g.\ a tiling of the 
circle by similar segments) or a hyperbolic tiling. Here, we will only be 
interested in Euclidean tilings, and maybe spherical ones. 

\begin{definition}
Let $(W,S)$ be a Euclidean reflection group (i.e.\ an affine Coxeter group) with 
Coxeter tiling $\Sigma\subset \dR^r$). Then $X$ is said to be a Euclidean 
building of type $(W,S)$ if it is a polysimplicial complex covered by copies of 
$\Sigma$ (called the \emph{apartments}) so that 
\begin{enumerate}[\indent\bf SEB1)]
  \item Any two cells (called \emph{facets}) are contained in a suitable apartment. 
  \item Given any two apartments $\dA$, $\dA'$, there is a simplicial 
    automorphism $\dA\simeq \dA'$ fixing $\dA\cap \dA'$. 
\end{enumerate}
\end{definition}

\begin{example}
Buildings of type $D_\infty$ correspond to trees in which all vertices have
valence $\geqslant 2$. In other words, these are graphs without loops or 
leaves. Apartments are bi-infinite geodesics in the tree. 
\end{example}

\begin{example}
For $\widetilde A_2$, we now have the tiling of $\dR^2$ by regular hexagons. 
Imagine gluings of half-tilings along codimension-$1$ walls. 
\end{example}

We will see that the group $\speciallinear_3(\dQ_p)$ acts on an 
$\widetilde A_2$-tiling strongly transitively (i.e.\ the group action is 
transitive on the inclusions of a chamber (maximal facet) into 
apartments. In other words, for any $C\subset\dA$ and $C'\subset \dA'$, 
chambers living in apartments, there is $g\in \speciallinear_2(\dQ_3)$ such 
that $g \dA=\dA'$ and $g C=C'$. This is the substitute for homogeneity of 
symmetric spaces. The main outcome of Bruhat-Tits theory is that ``to any 
reductive group $G$ over a local field $k$, is attached a Euclidean building 
$X$ on which $G(k)$ acts strongly transitively.'' (This is the so-called 
\emph{geometric half} of Bruhat-Tits theory. There is another part, which 
investigates models for $G$ over the valuation ring of $k$.) 


\subsection{Non-simplicial version of Euclidean buildings}

As a general convention, a \emph{local field} is a locally compact 
topological field, endowed with a non-archimedean absolute value. Such 
fields are classified: they are 
\begin{itemize}
  \item finite extensions of $\dQ_p$
  \item $\dF_q\laurent t$ for $q$ a prime power
\end{itemize}
What happens if $k$ is not discretely valued? This case is also covered by 
Bruhat-Tits theory. However, when the valuation is not discrete, the building 
is no longer a simplicial complex. But it still admits a metric, and is still 
a complete metric space with non-positive curvature whenever $k$ is complete. 
Why should we care about the case where $k$ is not discretely valued?
\begin{enumerate}
  \item Bruhat and Tits did.
  \item When studying the space of (linear) representations of a given finitely-generated 
    discrete group $\Gamma$: 
    $X_n(\Gamma)=\{\varphi:\Gamma \to \generallinear(\dR)\}/\text{conj}$, 
    then $\varphi:\Gamma \to \generallinear_n(\dR)$ corresponds to an action of 
    $\generallinear_n(\dR)/\specialorthogonal(n)=X$. There exists a compactification 
    procedure for $X_n(\Gamma)$, such that the added points at $\infty$ 
    correspond to $\Gamma$-actions on non-simplicial buildings. 
  \item There is a connection between Bruhat-Tits theory over arbitrary valued 
    fields and analytic geometry in the sense of Berkovich. 
\end{enumerate}


\subsection{Metric properties of buildings}

We would like to motivate the axioms SEB1 and SEB2. Essentially, they 
exist in order for us to be able to define a metric on the whole building. We 
already have a natural Euclidean metric on each apartment (because they are, by 
definition, subsets of Euclidean spaces). We want to have a metric on $X$ which 
gives the Euclidean one by restriction to any apartment. Essentially, the 
axioms SEB1-2 are ``desinged'' for the local metrics to patch together. For any 
$x,x'\in X$, choose an apartment $\dA\ni x,x'$. We can define the distance 
between $x$ and $x'$ from the metric on $\dA$. This is well-defined by the second 
axiom (after some work). 

\begin{theorem}[Bruhat-Tits]
Any Euclidean building $X$ admits a distance $d$ such that $(X,d)$ is a 
complete, $\cat(0)$ metric space. 
\end{theorem}

The property ``$\cat(0)$'' essentially captures ``non-positively curved and 
simply connected.'' More precisely, a metric space $(X,d)$ is said to be 
$\cat(0)$ if 
\begin{enumerate}
  \item it is \emph{geodesic} (for all $x,x'\in X$, there is a continuous 
    path $\gamma:[0,d(x,x')] \to X$ such that $\gamma(0)=x$, $\gamma(d(x,x'))=x'$, 
    and $d(\gamma(s),\gamma(t)) = |s-t|$) 
  \item geodesic triangles as thin as in the Euclidean plane (if 
    $x,y,z\in X$, draw the geodesic triangle in $\dR^2$. The length of the 
    paths between a point and half-edge are $\leqslant$ what we get in the 
    Euclidean case)
\end{enumerate}

The notion of a $\cat(0)$ space is due to Gromov. 

\begin{lemma}[Bruhat-Tits fixed point]\label{lemma:cat0}
Let $G$ act by isometries on a complete $\cat(0)$ metric space. If $G$ has a 
bounded orbit, then it has a fixed point. 
\end{lemma}
\begin{proof}[Proof (Serre)]
Any nonempty bounded subset in $X$ admits a \emph{unique} metrically characterized 
barycenter. 
\end{proof}

For example, whenever a compact group acts by isometries on a complete 
$\cat(0)$ space, then it admits a fixed point. 

Why is this useful? If $G$ is a reductive $k$-group, we have a Euclidean building 
$X$ with a strongly transitive action of $G(k)$. In the archimedean case, 
all maximal compacts are conjugate. This was proved by Emil Cartan by convexity 
arguments that amounted to the negative(?) curvature of the associated symmetric 
space. We can use the action of $G(k)$ on $X$ to classify maximal compact 
subgroups of $Gk$, which is useful for studying unitary representations of $G(k)$. 
Also, Lemma \ref{lemma:cat0} is used to prove the geometric part of Bruhat-Tits 
theory. 

In general, if $G$ is an algebraic group over an arbitrary field $F$, in order 
to understand $G(F)$, one looks at $G(F^s)$ with its $\galois(F^s/F)$-action. 
In other words, $G(F)=\h^0(F,G)$. The idea to attach a Euclidean building to 
$G(k)$ is 
\begin{enumerate}
  \item use a field extension $L/k$ which splits $G$
  \item attach a building $X_L$ to $G(L)$
  \item try to prove that $(X_L)^{\galois(L/k)}$ is a smaller building
\end{enumerate}
To show that the space $(X_L)^{\galois(L/k)}$ is ``large enough'' we use 
Lemma \ref{lemma:cat0}. 

For classical groups seen as fixed-point sets for involutions on linear 
groups, suitable Bruhat-Tits buildings are often fixed-point sets on 
the building of the ambient $\generallinear_n$ for the natural action of the 
involution. 










\section{Bruhat-Tits buildings for \texorpdfstring{$\generallinear(n)$}{GL(n)}}

The spaces for $\generallinear(n)$ were introduced by Goldman and Iwahori 
in \cite{gi63} before the general notion of a building existed. Start with 
$\speciallinear_n(\dR)$. The Archimedean symmetric space is 
$X=\speciallinear_n(\dR)/\specialorthogonal(n)$. To generalize this, we see 
$X$ as the space of normalized scalar products on $\dR^n$. Now let $k$ be an 
ultrametric field, and let $V$ be a $d$-dimensional $k$-vector space. Let 
$\sN=\sN_V$ be the space of non-archimedean norms on $V$. Our goal is to 
show that $\sN$ is a Bruhat-Tits building. 


\subsection{Examples of norms}

If we are willing to make choices, this is easy. Choose an ordered basis 
$\boldsymbol e=(e_1,\dots,e_d)$ for $V$. Let $\boldsymbol c=(c_1,\dots,c_d)$ be 
an ordered $d$-tuple of real numbers. Then we have an ultrametric norm 
$\|\cdot\|_{\boldsymbol e,\boldsymbol c}$ on $V$ defined by 
\[
  \left\| \sum \lambda_i e_i \right\|_{\boldsymbol e, \boldsymbol c} = \max\{e^{c_i} |\lambda_i|:1\leqslant i\leqslant d\} .
\]
Given a norm $\|\cdot\|$ on $V$, we say that a basis $\boldsymbol e$ is 
\emph{adapted to} (or \emph{diagonalizes} $\|\cdot\|$ if there is a 
tuple $\boldsymbol c$ such that $\|\cdot\|=\|\cdot\|_{\boldsymbol e,\boldsymbol c}$. 

Weil and Goldman-Iwahori proved that if $k$ is a local field, then given 
any two norms $\|\cdot\|$, $\|\cdot\|'$, there is a basis $\boldsymbol e$ and 
parameters $\boldsymbol c$, $\boldsymbol c'$ such that 
\begin{align*}
  \|\cdot\| &= \|\cdot\|_{\boldsymbol e,\boldsymbol c} \\
  \|\cdot\|' &= \|\cdot\|_{\boldsymbol e, \boldsymbol c'} .
\end{align*}
The proof is similar to that of the classical Grahm-Schmidt theorem. Essentially, 
one uses the compactness of $\dP^d(k)$. 





\bibliographystyle{alpha}
\bibliography{lyon-sources}

\end{document}
