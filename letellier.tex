\documentclass{article}

\usepackage{lyon-style}

\title{Geometric representation theory}
\author{Emmanuel Lettier}
\date{June 2-6, 2014}

\begin{document}
\maketitle
\tableofcontents





\section*{Introduction}

The goal is to compute character tables for ``finite groups of Lie type.'' Let 
$G$ be a connected reductive (linear algebraic) group over $\dF_q$. Then 
$G(\dF_q)$ is a finite group. 

\begin{problem}
Compute the (complex) character table of $G(\dF_q)$. 
\end{problem}

There are two difficulties here. First, we need to classify isomorphism classes 
of irreducible representations $\rho:G(\dF_q) \to \generallinear(V)$, where $V$ 
is a finite-dimensional $\dC$-vector space. Then, we need to compute the values 
(on conjugacy classes) of $\chi_\rho:G(\dF_q) \to \dC$ given by 
$g\mapsto \trace(\rho(g))$. 
\begin{center}
\begin{tabular}{cccc}
date & result & persons & tools \\ \hline
1905 & $\generallinear_2(\dF_q)$, $\speciallinear_2(\dF_q)$ & Jordan and Schur \\
\cite{g55} & $\generallinear_n(\dF_q)$ & Green & combinatorics \\
\cite{dl76} & general $G$ & Deligne-Lusztig & $\ell$-adic cohomology \\
1981--1990 & character-sheaf theory & Lusztig & perverse sheaves \\
\cite{s95a}, \cite{s95b} & Lusztig conjecture if $Z=Z^\circ$ & Shoji \\
2001 & for $\specialorthogonal_n$ and $\symplectic_{2 n}$ & Waldspurger \\
2003 & for $\speciallinear_n$ & Bonnal\'e 
\end{tabular}
\end{center}
Green's computation was highly combinatorial. On the other hand, Deligne and 
Lusztig constructed the representations ``directly'' using $\ell$-adic 
\'etale cohomology. Lusztig conjectured that there was a relation between 
``character sheaves'' and irreducible characters of $G(\dF_q)$. 

We will illustrate the general machinery using the specific example of 
$\generallinear_n$. 





\section{Combinatorial background}

A good combinatorial reference for all of this is \cite{m95}. 


\subsection{The ring of symmetric functions}

For $r\geqslant 1$, put 
$\Lambda_r = \dZ[x_1,\dots,x_r]^{S_r} = \bigoplus_{k\geqslant 0} \Lambda_r^k$, 
where $\Lambda_r^k$ is the set of homogeneous symmetric polynomials of degree 
$k$. For $m\geqslant r$, we have a map 
$\rho_r^m:\dZ[x_1,\dots,x_m] \to \dZ[x_1,\dots,x_r]$, given by 
\[
  x_i \mapsto \begin{cases} x_i & i\leqslant r \\ 0 & i>r \end{cases}
\]
This induces a map $\rho_r^m:\Lambda_m^k \to \Lambda_r^k$. Put 
$\Lambda^k = \varprojlim_r \Lambda_r^k$ and 
$\Lambda = \bigoplus_{k\geqslant 0} \Lambda^k$. We call $\Lambda$ the 
\emph{ring of symmetric functions in infinitely many variables}. Elements of 
$\Lambda$ look like e.g.\ $P_2(x) = x_1^2 + x_2^2 + \cdots$. 

A \emph{partition} is an infinite non-increasing sequence 
$\lambda=(\lambda_1,\lambda_2,\dots)$ of non-negative integers, with finitely 
many nonzero terms. We put $|\lambda| = \sum \lambda_i$ and 
$\ell(\lambda) = \max\{i:\lambda_i\ne 0\}$. Let $\cP$ be the set of all 
partitions, and $\cP_n$ the set of partitions of size $n$. 


\subsection{Monomial symmetric functions}

For $\lambda=(\lambda_1,\dots,\lambda_r,0,\dots)$, let 
\[
  m_\lambda(x_1,\dots,x_r) = \sum x_1^{\alpha_1} \dotsm x_r^{\alpha_r} ,
\]
where the sum runs over all distinct permutations $(\alpha_1,\dots,\alpha_r)$ of 
$(\lambda_1,\dots,\lambda_r)$. For example, 
\begin{equation}\label{eq:m-poly}
  m_{(2,2,1)}(x_1,x_2,x_3) = x_1^2 x_2^2 x_3 + x_1^2 x_2 x_3^2 + x_1 x_2^2 x_3^2 .
\end{equation}
This defines an element $m_\lambda\in \Lambda$ such that 
$m_\lambda(x_1,\dots,x_r)$ is the right-hand side of \eqref{eq:m-poly} if 
$\ell(\lambda)\leqslant r$, and $0$ otherwise. If $\lambda=0$, put 
$m_\lambda=1$. Then $\{m_\lambda:\lambda\in \cP\}$ is a $\dZ$-basis for 
$\Lambda$. 


\subsection{Elementary symmetric functions}

For each $n\geqslant 0$, define $e_n\in \Lambda$ by 
\[
  e_n = \begin{cases} m_{(1^n)} & n\geqslant 1 \\ 1 & n=0 \end{cases} 
\] 
For $\lambda=(\lambda_1,\lambda_2,\dots)\in \cP$, we define 
$e_\lambda = e_{\lambda_1} e_{\lambda_2} \cdots$. It turns out that 
$\{e_\lambda:\lambda\in \cP\}$ is a $\dZ$-basis for $\Lambda$, and 
$\Lambda=\dZ[e_1,e_2,\dots]$. 


\subsection{Power-sums}

For $n\geqslant 0$, define $p_n\in \Lambda$ by 
\[
  p_n = \begin{cases} m_{(n^1)} & n\geqslant 1 \\ 1 & n=0 \end{cases}
\]
For example, 
\[
  p_3 = x_1^3 + x_2^3 + x_3^3 + x_4^3 + \cdots .
\]
For $\lambda=(\lambda_1,\lambda_2,\dots)$, put 
$p_\lambda = p_{\lambda_1} p_{\lambda_2} \dotsm$. The $p_\lambda$ form 
a $\dQ$-basis of $\Lambda_\dQ = \Lambda\otimes \dQ$. Thus 
$\Lambda_\dQ=\dQ[p_1,p_2,\dots]$. 


\subsection{Shur functions}

Given $r$ variables $x_1,\dots,x_r$, put $\delta_r = (r-1,r-2,\dots,1,0)$. For 
any $\lambda=(\lambda_1,\lambda_2,\dots)$, put 
\[
  a_{\lambda+\delta_r} = \det
  \begin{pmatrix}
    x_1^{\lambda_1+\delta_1} & x_2^{\lambda_2 + \delta_2} & \cdots & x_1^{\lambda_r} \\ 
    \vdots & \vdots & & \vdots \\ 
    x_r^{\lambda_1 + \delta_1} & x_r^{\lambda_2 + \delta_2} & \cdots & x_r^{\lambda_r}  
  \end{pmatrix}
  = \sum_{w\in S_r} \varepsilon(w) w(x_1^{\lambda_1+\delta_1} \dotsm x_r^{\lambda_r+\delta_r}) .
\]
This is skew-symmetric, and we have $a_{\delta_r} \mid a_{\lambda+\delta_r}$ in 
$\dZ[x_1,\dots,x_r]$. The \emph{Schur function} is 
\begin{equation}\label{eq:schur-poly}
  S_\lambda(x_1,\dots,x_r) = \frac{a_\lambda+\delta_r}{a_{\delta_r}} \in \dZ[x_1,\dots,x_r]^{S_r} .
\end{equation}

This defines an element $S_\lambda\in \Lambda$, determined by 
\[
  S_\lambda(x_1,\dots,x_r) = \begin{cases} \text{right-hand side of \eqref{eq:schur-poly}} & \ell(\lambda) \leqslant r \\ 0 & \text{otherwise} \end{cases}
\]
Moreover, $\{S_\lambda:\lambda\in \cP\}$ is a $\dZ$-basis for $\Lambda$. 


\subsection{Relationship with representations of the symmetric group}

There is a natural bijection 
\[
  S_n^\natural = \{\text{conjugacy classes in }S_n\} \isomorphism \cP_n ,
\]
sending a conjugacy class $c$ to the partition coming from lengths of cycles in 
the decomposition of an element of $c$ into disjoint cycles. For example, an 
$n$-cycle maps to the partition $(n,0,0,\dots)$, and the identity maps to 
$(1^n,0,\dots)$. Let $\mu\mapsto c_\mu$ be the inverse of this bijection. 
So we have a bijection $\cP_n \isomorphism \irreducible(S_n)$, 
written $\lambda\mapsto \chi^\lambda$. One has 
\[
  S_\lambda = \sum_\mu \frac{1}{z_\mu} \chi_\mu^\lambda p_\mu , 
\]
where $z_\mu = \# C_{S_n}(\sigma_\mu)$, and 
$\chi_\mu^\lambda$ is the value of $\chi^\lambda$ at $c_\mu$. A trivial 
corollary is that 
\[
  \chi^\lambda = \sum_\mu \frac{1}{z_\mu} \chi_\mu^\lambda 1_{c\mu} .
\]
So $\{\chi^\lambda\}$ and $\{1_{c_\mu}\}$ are both $\dC$-bases for 
$\hom(S_n^\natural,\dC)$. 

We would like to do something similar for $\generallinear_n(\dF_q)$. 


\subsection{Hall-Littlewood symmetric functions}

Let $t$ be an extra variable. Consider the ring 
$\Lambda_{\dZ[t]} = \Lambda[t] = \Lambda\otimes \dZ[t]$. For $\lambda\in \cP$ 
with $\ell(\lambda)\leqslant r$, put 
\begin{align*}
  R_\lambda(x_1,\dots,x_r;t) 
    &= \sum_{w\in S_r} w\left(x_1^{\lambda_1} \dotsm x_r^{\lambda_r} \prod_{i<j} \frac{x_i - t x_j}{x_i - x_j}\right) \\
    &= \frac{1}{a_{\delta_r}} \sum_{w\in S_r} \varepsilon(w) w\left(x_1^{\lambda_1} \dotsm x_r^{\lambda_r} \prod_{i<j} x_i - t x_j\right) .
\end{align*}
These are elements of $\Lambda_r[t]$. Unfortunately, 
$R_\lambda(x_1,\dots,x_r,0;t)\ne R_\lambda(x_1,\dots,x_r;t)$ as elements of 
$\Lambda_{r+1}[t]$, so we need to ``fudge things'' a bit to get a well-defined 
element of $\Lambda[t]$. For $m\geqslant 1$, put 
\[
  v_m(t) = \sum_{w\in S_m} w\left(\prod_{i<j} \frac{x_i-t x_j}{x_i - x_j}\right) .
\]
For $\lambda=(\lambda_1,\dots,\lambda_r,0,\dots)$, denote by $m_i$ the multiplicity 
of $i$ in $\lambda$ if $i\geqslant 1$, and by 
$m_0 = \#\{i\leqslant r:\lambda_i = 0\}$. Note that $m_0$ depends heavily on $r$. 
Define 
\[
  v_{\lambda,r}(t) = \prod_{i\geqslant 0} v_{m_i}(t) .
\]
Then 
\[
  R_\lambda(x_1,\dots,x_r;t) = v_\lambda(t) \sum_{w\in S_r/S_r^\lambda} w\left(x_1^{\lambda_1} \dots x_r^{\lambda_r} \prod_{\lambda_i>\lambda_j} \frac{x_i-t x_j}{x_i-x_j}\right) ,
\]
where as before 
$S_r^\lambda = \{w\in S_r:\lambda_{w(i)} = \lambda_i\text{ for }1\leqslant i\leqslant r\} \simeq \prod_{i\geqslant 0} S_{m_i}$. Define 
\[
  P_\lambda(x_1,\dots,x_r;t) = \frac{1}{v_\lambda(t)} R_\lambda(x_1,\dots,x_r;t) ; 
\]
this is the \emph{Hall-Littlewood symmetric function}. 

\begin{proposition}
1. $P_\lambda(x_1,\dots,x_r;0) = S_\lambda(x_1,\dots,x_r)$. 

2. $P_\lambda(x_1,\dots,x_r;1) = m_\lambda(x_1,\dots,x_r)$. 

3. $P_\lambda(x_1,\dots,x_r,0;t) = P_\lambda(x_1,\dots,x_r;t)$. 
\end{proposition}

By 3, we have a well-defined element $P_\lambda\in \Lambda[t]$. Moreover, 
$\{P_\lambda:\lambda\in \cP\}$ is a $\dZ[t]$-basis for $\Lambda[t]$. 

The \emph{Kostka-Foulkes polynomials} $\{K_{\lambda\mu}(t)\}$ are by 
the relation 
\[
  S_\lambda = \sum_\mu K_{\lambda\mu}(t) P_\mu .
\]

\begin{theorem}[Lascoux-Sch\"utzenberger, Lusztig]
$K_{\lambda\mu}(t)\in \dZ_{\geqslant 0}[t]$. 
\end{theorem}





\section{The character table of \texorpdfstring{$\generallinear_n(\dF_q)$}{GLnFq}}


\subsection{The conjugacy classes}

Fix $n\geqslant 1$. Let $M=\overline{\dF_q}^\times$, considered as an abstract 
group. Let 
\[
  \widetilde\dT_n = \left\{\varphi:M\to \cP:\|\varphi\| := \sum_{m\in M} |\varphi(m)| = n\right\} . 
\]
We claim that $\generallinear_n(\overline{\dF_q})^\natural \isomorphism \widetilde\dT_n$. 
We illustrate this with an example. If $n=3$, then the conjugacy class of 
\[
  \begin{pmatrix} \alpha & 1 \\ & \alpha \\ & & \beta \\ & & & \beta \end{pmatrix}
\]
is sent to the map $M\to \cP$ given by 
\[
  \gamma\mapsto 
  \begin{cases}
    (2,0,0,\dots) & \gamma = \alpha \\ 
    (1,1,0,0,\dots) & \gamma = \beta \\
    0 & \text{otherwise} 
  \end{cases}
\]
For the general case, let $F:\overline{\dF_q} \to \overline{\dF_q}$ be the Frobenius 
$x\mapsto x^q$, and let $\Phi=\{\text{$F$-orbits of $M$}\}$. Every $F$-orbit is 
of the form $\{x,x^q,\dots,x^{q^{d-1}}\}$, for the minimal $d$ with 
$x^{q^d} = x$. If $f\in \Phi$, put 
$d(f) = \# f$ (cardinality of $f$ as an orbit set). 

\begin{proposition}
Put $\dT_n = \{\varphi:\Phi\to \cP:\|\varphi\| = \sum_{f\in \Phi} d(f)|\varphi(f)| = n\}$. 
Then $\generallinear_n(\dF_q)^\natural \isomorphism \dT_n$. 
\end{proposition}
\begin{proof}
Write $\generallinear_n = \generallinear_n(\overline{\dF_q})$, and 
let $F:\generallinear_n\to \generallinear_n$ be the Frobenius 
$(a_{i j})\mapsto (a_{i j}^q)$. We will show that 
\[
  \{\text{$F$-stable conjugacy classes of $\generallinear_n$}\}\isomorphism \dT_n .
\]
Indeed, in any conjugacy class $C$, there is an element of the form 
\[
  \begin{pmatrix} \Delta_1 \\ & \Delta_2 \\ & & \ddots \end{pmatrix} ,
\]
where 
\[
  \Delta_i = \begin{pmatrix} \alpha_i & 1 \\ & \alpha_i & 1 \\ \end{pmatrix}
\]
If $F(C)=C$, then $\Delta$ and $F(\Delta)$ are $\generallinear_n$-conjugate. Thus 
$F$ permutes the blocks $\Delta_i$, so $F(\alpha_i) = \alpha_j$ for some $j$. Thus 
$\Delta_i$ and $\Delta_j$ have the same Jordan form. 
\end{proof}

We claim that 
\[
  \generallinear_n(\dF_q)^\natural\isomorphism \{\text{$F$-stable conjugacy classes of $\generallinear_n$}\} . 
\]
Note that this fails for general algebraic groups. In general, suppose $G$ 
is a connected affine algebraic group acting on an algebraic variety $X$ (all 
over $\overline{\dF_q}$). Assume $G$, $X$, and the action of $G$ on $X$ are all 
defined over $\dF_q$. Let $F:G\to G$ and $F:X\to X$ be the relative Frobenii. 
Since the action of $G$ on $X$ is defined over $\dF_q$, we have 
$F(g\cdot x) = F(g) \cdot F(x)$. 
\begin{enumerate}
  \item If $O$ is an $F$-stable $G$-orbit of $X$, then 
    $O^F=\{x\in O:F(x)=x\}=O(\dF_q)$ is not empty (this uses the fact that 
    $g\mapsto g^{-1} F(g)$, $G\to G$ is surjective). Thus the above map is 
    surjective for any $G$. 
  \item If $x\in X^F=X(\dF_q)$ and $\stabilizer_G(x) = \stabilizer_G(x)^\circ$, 
    then $(G\cdot x)^F = G^F\cdot x$. 
\end{enumerate}

In $\generallinear_n$, the stabilizers $\stabilizer_{\generallinear_n}(x)$ are 
\emph{always} connected ($x\in \generallinear_n$). 

\begin{exercise}
Prove that $\stabilizer_{\speciallinear_2}\smat{1}{1}{}{1}$ has two connected 
components if $2\nmid q$. 
\end{exercise}

Each $g\in \generallinear_n(\dF_q)$ defines an $\dF_q[t]$-structure on 
$V=(\dF_q)^n$ as follows: $t\cdot v = g(v)$ for all $v\in V$. This structure 
depends only on the conjugacy class of $g$. If 
$C\subset \generallinear_n(\dF_q)$ is a conjugacy class corresponding to 
$\varphi:\Phi \to \cP$, denote by $V_\varphi$ the corresponding 
$\dF_q[t]$-structure on $V$. From commutative algebra, we know that 
$V_\varphi \simeq \bigoplus_{f\in \Phi} \bigoplus_{i\geqslant 1} \dF_q[t] / (P_f)^{\varphi(f)_i}$, and 
$P_f = \prod_{i=1}^{d-1} (t-x^{q^i})$. 

\begin{example}[$\generallinear_2(\dF_q)$]
Consider the following table, in which $\varphi(f)=0$ unless it is defined to 
be otherwise:
\begin{center}
\begin{tabular}{c|ccc}
conjugacy class & $\varphi:\Phi\to \cP$ \\ \hline
$\smat{a}{}{}{a}$, $a\in \dF_q^\times$ & $\{a\}\mapsto (1,1,0,\ldots)$ \\
$\smat{a}{}{}{b}$, $a\ne b\in \dF_q^\times$ & $\{a\},\{b\}\mapsto (1,0,\dots)$ \\
$\smat{a}{1}{}{a}$, $a\in \dF_q^\times$ & $\{a\}\mapsto (2,0,\dots)$ \\
$\smat{x}{}{}{x^q}$, $x\in \dF_{q^2}\smallsetminus \dF_q^\times$ & $\{x,x^q\}\mapsto (1,0,\dots)$
\end{tabular}
\end{center}
\end{example}

\begin{example}[$\generallinear_3(\dF_q)$]
Conventions here are the same. [do this]
\end{example}





\subsection{Parabolic induction}

For any finite group $H$, write $\cC H=\dC^{H^\natural}$ for the space of 
functions $H\to \dC$ that are constant on conjugacy classes. 

Let $\lambda=(n_1,n_2,\dots,n_r,0,\dots)$ be a partition of $n$. 
Let $\generallinear_\lambda\subset \generallinear_n$ be the group consisting of 
matrices of the form $A_1\oplus \cdots \oplus A_r$, with each $A_i$ an 
invertible $n_i\times n_i$ matrix. Note that 
$\generallinear_\lambda\simeq \prod_i \generallinear_{n_i}$. If we write 
$L=\generallinear_\lambda$, we will define functions 
$R_L^{\generallinear_n}:\cC(L(\dF_q)) \to \cC(\generallinear_n(\dF_q))$. 

Let $P$ be the standard parabolic of ``block upper-triangular matrices'' 
associated to $\lambda$, and let $U_P$ be the unipotent radical of $P$. 
One has $P=\generallinear_\lambda \cdot U_P$. If 
$f\in \cC(L(\dF_q))$, define 
$R_L^{\generallinear_n}(f)=\induce_P^{\generallinear_n}(\widetilde f)$, 
where $\widetilde f:P(\dF_q) \to \dC$ is given by 
$\ell u\mapsto f(\ell)$ for $\ell\in L$, $u\in U_P$. If 
$f\in \dN[\irreducible(L(\dF_q))]$, then 
$R_L^{\generallinear_n}(f)\in \dN[\irreducible(\generallinear_n(\dF_q))]$. 
It turns out that $R_L^{\generallinear_n}(f)$ does not depend on the choice of 
a rational $P$ having $L$ as a Levi factor. 

Put $A_n = \cC(\generallinear_n(\dF_q)$ and $A=\bigoplus_{n\geqslant 0} A_n$. 
We define an inner product on $A$: if 
$f\in A_{n_1}$ and $g\in A_{n_2}$, put 
\[
  f\circ g = R_{\generallinear(n_1)\times \generallinear(n_2)}^{\generallinear(n_1+n_2)}(f,g) .
\]
This is commutative and associative.  

Let $\varphi:\Phi\to \cP$ have finite support. Denote by 
$\pi_\varphi\in \cC(\generallinear_{\|\varphi\|}(\dF_q))$ the characteristic 
function of the associated conjugacy class $c_\varphi$. Then 
$\{\pi_\varphi\}$ is a $\dC$-basis for $A$, and $\pi_0$ is the multiplicative 
unit of $A$. 

\begin{lemma}
If $\|\varphi\| = \sum \| \varphi_i\|$, then 
$\pi_{\varphi_1}\circ \pi_{\varphi_2} \circ \cdots \circ \pi_{\varphi_r}(c_\varphi)$ 
is the number of sequences $0=W^{(0)}\subset W^{(1)} \subset \cdots \subset W^{(r)}=V_\varphi$ 
of submodules of $V_\varphi$ such that $W^{(i)}/W^{(i-1)} \simeq V_{\varphi_i}$. 
\end{lemma}

Here $V_\varphi$ is the $\dF_q[t]$-structure on $(\dF_q)^n$ defined by 
$\varphi$. 





\subsection{The characteristic map}

For each $f\in \Phi$, let $\{x_{1,f},x_{2,f},\dots\}$ be infinitely many 
variables. Let $x=\{x_1,x_2,\dots\}$ be a set of infinitely many variables. 
Put $\Lambda_{\dQ(q)} = \Lambda(x)\otimes \dQ(q)$. If 
$u\in \Lambda_{\dQ(q)}$, we denote by 
$u(f)$ the corresponding function in 
$\Lambda_{\dQ(q)}(x_f)$. 

Put $B=\dC[e_n(f):n\geqslant 0, f\in \Phi]$. Then 
$B\subset \dQ(q)\otimes \bigotimes_{f\in \dQ} \Lambda(x_f)$. The Hall pairing on 
$\Lambda(x)$ makes the Schur functions an orthonormal basis. We grade $B$ by 
$\deg(x_i,f) = d(f)$. For $\lambda\in \cP$, $f\in \Phi$, put 
$\widetilde P_\lambda(f) = q^{-n(\lambda) d(f)} P_\lambda(f; q^{-d(f)})$. Recall 
that $n(\lambda) = \sum_{i\geqslant 1} (i-1) \lambda_i$. 
For $\varphi:\Phi \to \cP$ with finite support, we put 
\[
  \widetilde P_\varphi = \prod_{f\in \Phi} \widetilde P_{\varphi(f)} (f) \in B .
\]
Since the Hall-Littlewood symmetric functions form a $\dZ$-basis of 
$\Lambda(x)$, the $\widetilde P_\varphi$ form a $\dC$-basis of $B$. 

The \emph{characteristic map} $\characteristic:A\to B$, given by 
$\pi_\varphi\mapsto \widetilde P_\varphi$, is an (isometric) isomorphism of 
graded $\dC$-algebras. 


\subsection{Construction of the irreducible characters of \texorpdfstring{$\generallinear_n(\dF_q)$}{GLnFq}}

Let $M_n=\dF_{q^n}^\times$. Let $\widehat{M_n} = \hom(M_n,\dC^\times)$, and let 
$L=\varinjlim_n \widehat{M_n}$ and $L_n=\image(\widehat{M_n} \to L)$. The 
Frobenius $F$ acts on $L$; let $\Theta$ be the set of $F$-orbits in $L$. 
We want $\irreducible(\generallinear_n(\dF_q))$ to be in bijection with 
\[
  \left\{\psi:\Theta\to \cP\text{ such that }\|\psi\| := \sum_{\theta\in \Theta} d(\theta) \cdot |\psi(\theta)| = n\right\} .
\]
The set $\irreducible(\generallinear_n(\dF_q))$ is an orthonormal basis for 
$\langle \cdot,\cdot\rangle$. We look for an orthonormal basis of $B$. For 
$\chi\in \irreducible(\generallinear_n(\dF_q))$, we expect 
$\characteristic(\chi)$ to be ``related to'' Schur functions. 

For $f\in \Phi$ and $x\in f$, put 
\[
  \widetilde P_n(x) = \begin{cases} P_{n/d}(f) & d=d(f)\mid n \\ 0 & \text{otherwise} \end{cases}
\]
Then for each $\xi\in L$, define 
\[
  \widetilde P_n(\xi) = 
  \begin{cases}
    (-1)^{n-1} \sum_{x\in M_n} \langle x,\xi\rangle_n \widetilde P_n(x) & \xi\in L_n \\ 0 & \text{otherwise} 
  \end{cases}
\]
Note that $\widetilde P_n(\xi)$ depends only on the $F$-orbit of $\xi$. For 
$\theta\in \Theta$, put $\widetilde P_r(\theta) = \widetilde P_{rd}(\xi)$, for 
any $\xi\in \theta$, and $d=d(\theta)$. 

\begin{example}
If $\xi=1\in L$ and $n=2$, then 
\[
  \widetilde P_n(\xi) = - \sum_{\substack{f\in \Phi \\ d(f)=1}} \sum_i x_{i,f}^2 - 2 \sum_{\substack{f\in \Phi \\ d(f) = 2}} \sum_i x_{i,f} .
\]
Make the change of variables $\{x_{i,f}\} \to \{y_{i,\theta}\}$, such that 
$\widetilde P_r(\theta)$ are power-sums in $\{y_{i,\theta}\}$. 
\end{example}

For $\lambda\in \cP$, put 
$\widetilde P_\lambda(\theta) = P_{\lambda_1}(\theta) P_{\lambda_2}(\theta) \dotsm$ and 
$S_\lambda(\theta) = \sum_\mu \frac{1}{z_\mu} \chi_\mu^{-1} \widetilde P_\mu(\theta)$. 
For $\psi:\Theta \to \cP$ with finite support, put 
$S_\psi = \prod_{\theta\in \Theta} S_{\psi(\theta)}(\theta)$. 

\begin{proposition}
1. $\{S_\psi\}$ is an orthornormal $\dC$-basis of $B$. 

2. Assume that $|\lambda|=|\mu|$. Then 
\[
  S_\lambda(1) = \sum_\mu \widetilde K_{\lambda\mu}(q) \widetilde P_\mu(1) .
\]
Here we are using the definition 
$\widetilde K_{\lambda\mu}(q) = q^{n(\mu)} K_{\lambda\mu}(q^{-1})$. 
\end{proposition}


\begin{theorem}
The map $A\to B$ induces 
$\irreducible(\generallinear_n(\dF_q)) \to\{S_\psi:\|\psi\| = n\}$. 
\end{theorem}
\begin{proof}
If $\theta\in \Theta$ and $n\geqslant 0$, then $e_n(\theta)$ is the characteristic 
function of a character of $\generallinear_{nd(\theta)}(\dF_q)$.For 
$\psi:\Theta\to \cP$, the $S_\psi$ are polynomials in the $e_n(\theta)$, hence 
have coefficients in $\dZ$. This implies that 
$S_\psi=\characteristic(\text{virtual character})$, say $\sX^\psi$. One has 
$\langle S_\psi,S_\psi\rangle = \langle \sX^\psi,\sX^\psi\rangle$, so 
$\sX^\psi$ or $-\sX^\psi$ is irreducible. Thus $\sX^\psi(1)>0$. 
\end{proof}

To conclude, call ``unipotent character'' the irreducible characters 
$\sX^\psi$ with $\psi\in \{\Theta\xrightarrow\psi\cP:\text{support of }\psi=1\}$. 

For the rest of these notes, we will give a geometric construction of the 
$\sX^\psi$. 





\section{\texorpdfstring{$\ell$}{l}-adic sheaves and perverse sheaves}

The $\widetilde K_{\lambda\mu}(q)$ can be realized as Khazdan-Lusztig polynomials 
using perverse sheaves. 


\subsection{Sheaves on topological spaces}

Let $X$ be a topological space. A \emph{presheaf} on $X$ is a (contravariant) 
functor $F:\open(X) \to \cC$, where $\cC$ is the category of groups, sets, 
rings, modules, \ldots. Morphisms of presheaves are natural transformations of 
functors. A presheaf is a \emph{sheaf} if for any family $\{U_i\}$ of open 
subsets of $X$, and any family $\{s_i\in F U_i\}$ such that 
$s_i|_{U_i\cap U_j} = s_j|_{U_i\cap U_j}$, there exists a unique 
$s\in F U$ such that $s|_{U_i} = s_i$. We often use script letters for sheaves, 
e.g.\ $\sF$, $\sG$, \ldots. If $\sF$ is a sheaf on $X$ and $x\in X$, write 
$\sF_x=\varinjlim_{U\ni x} \sF(U)$ for the \emph{stalk} of $\sF$ at $x$. 

[\ldots I know what sheaves are\ldots]


\subsection{\texorpdfstring{$\ell$}{l}-adic analogue}

If $X$ is a variety over $\dC$, we can give $X(\dC)$ the structure of a topological 
space using the topology on $\dC$. In characteristic $p$, a variety has no natural 
topology that gives a good sheaf theory. Fortunately, a very deep theory of 
Grothendieck gives us something called the ``\'etale topology,'' so we can talk 
about sheaves on $X_\etale$. 

We would like to use the \'etale topology to define spaces $\h^i(X_\etale,k)$ ($k$ a 
field of characteristic zero) for $X$ over $\dF_q$. Unfortunately, if we try to 
define $\h^i(X_\etale, k)$ in the naive manner, we get trivial cohomology. 

[\ldots Grothendieck's insight $\to\h^i(X_\etale, \dQ_\ell)$\ldots]





\bibliographystyle{alpha}
\bibliography{lyon-sources}

\end{document}
