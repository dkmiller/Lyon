\documentclass{article}

\usepackage{lyon-style}

\title{Geometric representation theory}
\author{Emmanuel Lettier}
\date{June 2-6, 2014}

\begin{document}
\maketitle
\tableofcontents





\section*{Introduction}

The goal is to compute character tables for ``finite groups of Lie type.'' Let 
$G$ be a connected reductive (linear algebraic) group over $\dF_q$. Then 
$G(\dF_q)$ is a finite group. 

\begin{problem}
Compute the (complex) character table of $G(\dF_q)$. 
\end{problem}

There are two difficulties here. First, we need to classify isomorphism classes 
of irreducible representations $\rho:G(\dF_q) \to \generallinear(V)$, where $V$ 
is a finite-dimensional $\dC$-vector space. Then, we need to compute the values 
(on conjugacy classes) of $\chi_\rho:G(\dF_q) \to \dC$ given by 
$g\mapsto \trace(\rho(g))$. 
\begin{center}
\begin{tabular}{cccc}
date & result & persons & tools \\ \hline
1905 & $\generallinear_2(\dF_q)$, $\speciallinear_2(\dF_q)$ & Jordan and Schur \\
\cite{g55} & $\generallinear_n(\dF_q)$ & Green & combinatorics \\
\cite{dl76} & general $G$ & Deligne-Lusztig & $\ell$-adic cohomology \\
1981--1990 & character-sheaf theory & Lusztig & perverse sheaves \\
\cite{s95a}, \cite{s95b} & Lusztig conjecture if $Z=Z^\circ$ & Shoji \\
2001 & for $\specialorthogonal_n$ and $\symplectic_{2 n}$ & Waldspurger \\
2003 & for $\speciallinear_n$ & Bonnal\'e 
\end{tabular}
\end{center}
Green's computation was highly combinatorial. On the other hand, Deligne and 
Lusztig constructed the representations ``directly'' using $\ell$-adic 
\'etale cohomology. Lusztig conjectured that there was a relation between 
``character sheaves'' and irreducible characters of $G(\dF_q)$. 

We will illustrate the general machinery using the specific example of 
$\generallinear_n$. 





\section{Combinatorial background}

A good combinatorial reference for all of this is \cite{m95}. 


\subsection{The ring of symmetric functions}

For $r\geqslant 1$, put 
$\Lambda_r = \dZ[x_1,\dots,x_r]^{S_r} = \bigoplus_{k\geqslant 0} \Lambda_r^k$, 
where $\Lambda_r^k$ is the set of homogeneous symmetric polynomials of degree 
$k$. For $m\geqslant r$, we have a map 
$\rho_r^m:\dZ[x_1,\dots,x_m] \to \dZ[x_1,\dots,x_r]$, given by 
\[
  x_i \mapsto \begin{cases} x_i & i\leqslant r \\ 0 & i>r \end{cases}
\]
This induces a map $\rho_r^m:\Lambda_m^k \to \Lambda_r^k$. Put 
$\Lambda^k = \varprojlim_r \Lambda_r^k$ and 
$\Lambda = \bigoplus_{k\geqslant 0} \Lambda^k$. We call $\Lambda$ the 
\emph{ring of symmetric functions in infinitely many variables}. Elements of 
$\Lambda$ look like e.g.\ $P_2(x) = x_1^2 + x_2^2 + \cdots$. 

A \emph{partition} is an infinite non-increasing sequence 
$\lambda=(\lambda_1,\lambda_2,\dots)$ of non-negative integers, with finitely 
many nonzero terms. We put $|\lambda| = \sum \lambda_i$ and 
$\ell(\lambda) = \max\{i:\lambda_i\ne 0\}$. Let $\cP$ be the set of all 
partitions, and $\cP_n$ the set of partitions of size $n$. 


\subsection{Monomial symmetric functions}

For $\lambda=(\lambda_1,\dots,\lambda_r,0,\dots)$, let 
\[
  m_\lambda(x_1,\dots,x_r) = \sum x_1^{\alpha_1} \dotsm x_r^{\alpha_r} ,
\]
where the sum runs over all distinct permutations $(\alpha_1,\dots,\alpha_r)$ of 
$(\lambda_1,\dots,\lambda_r)$. For example, 
\begin{equation}\label{eq:m-poly}
  m_{(2,2,1)}(x_1,x_2,x_3) = x_1^2 x_2^2 x_3 + x_1^2 x_2 x_3^2 + x_1 x_2^2 x_3^2 .
\end{equation}
This defines an element $m_\lambda\in \Lambda$ such that 
$m_\lambda(x_1,\dots,x_r)$ is the right-hand side of \eqref{eq:m-poly} if 
$\ell(\lambda)\leqslant r$, and $0$ otherwise. If $\lambda=0$, put 
$m_\lambda=1$. Then $\{m_\lambda:\lambda\in \cP\}$ is a $\dZ$-basis for 
$\Lambda$. 


\subsection{Elementary symmetric functions}

For each $n\geqslant 0$, define $e_n\in \Lambda$ by 
\[
  e_n = \begin{cases} m_{(1^n)} & n\geqslant 1 \\ 1 & n=0 \end{cases} 
\] 
For $\lambda=(\lambda_1,\lambda_2,\dots)\in \cP$, we define 
$e_\lambda = e_{\lambda_1} e_{\lambda_2} \cdots$. It turns out that 
$\{e_\lambda:\lambda\in \cP\}$ is a $\dZ$-basis for $\Lambda$, and 
$\Lambda=\dZ[e_1,e_2,\dots]$. 


\subsection{Power-sums}

For $n\geqslant 0$, define $p_n\in \Lambda$ by 
\[
  p_n = \begin{cases} m_{(n^1)} & n\geqslant 1 \\ 1 & n=0 \end{cases}
\]
For example, 
\[
  p_3 = x_1^3 + x_2^3 + x_3^3 + x_4^3 + \cdots .
\]
For $\lambda=(\lambda_1,\lambda_2,\dots)$, put 
$p_\lambda = p_{\lambda_1} p_{\lambda_2} \dotsm$. The $p_\lambda$ form 
a $\dQ$-basis of $\Lambda_\dQ = \Lambda\otimes \dQ$. Thus 
$\Lambda_\dQ=\dQ[p_1,p_2,\dots]$. 


\subsection{Shur functions}

Given $r$ variables $x_1,\dots,x_r$, put $\delta_r = (r-1,r-2,\dots,1,0)$. For 
any $\lambda=(\lambda_1,\lambda_2,\dots)$, put 
\[
  a_{\lambda+\delta_r} = \det
  \begin{pmatrix}
    x_1^{\lambda_1+\delta_1} & x_2^{\lambda_2 + \delta_2} & \cdots & x_1^{\lambda_r} \\ 
    \vdots & \vdots & & \vdots \\ 
    x_r^{\lambda_1 + \delta_1} & x_r^{\lambda_2 + \delta_2} & \cdots & x_r^{\lambda_r}  
  \end{pmatrix}
  = \sum_{w\in S_r} \varepsilon(w) w(x_1^{\lambda_1+\delta_1} \dotsm x_r^{\lambda_r+\delta_r}) .
\]
This is skew-symmetric, and we have $a_{\delta_r} \mid a_{\lambda+\delta_r}$ in 
$\dZ[x_1,\dots,x_r]$. The \emph{Schur function} is 
\begin{equation}\label{eq:schur-poly}
  S_\lambda(x_1,\dots,x_r) = \frac{a_\lambda+\delta_r}{a_{\delta_r}} \in \dZ[x_1,\dots,x_r]^{S_r} .
\end{equation}

This defines an element $S_\lambda\in \Lambda$, determined by 
\[
  S_\lambda(x_1,\dots,x_r) = \begin{cases} \text{right-hand side of \eqref{eq:schur-poly}} & \ell(\lambda) \leqslant r \\ 0 & \text{otherwise} \end{cases}
\]
Moreover, $\{S_\lambda:\lambda\in \cP\}$ is a $\dZ$-basis for $\Lambda$. 


\subsection{Relationship with representations of the symmetric group}

There is a natural bijection 
\[
  S_n^\natural = \{\text{conjugacy classes in }S_n\} \isomorphism \cP_n ,
\]
sending a conjugacy class $c$ to the partition coming from lengths of cycles in 
the decomposition of an element of $c$ into disjoint cycles. For example, an 
$n$-cycle maps to the partition $(n,0,0,\dots)$, and the identity maps to 
$(1^n,0,\dots)$. Let $\mu\mapsto c_\mu$ be the inverse of this bijection. 
So we have a bijection $\cP_n \isomorphism \irreducible(S_n)$, 
written $\lambda\mapsto \chi^\lambda$. One has 
\[
  S_\lambda = \sum_\mu \frac{1}{z_\mu} \chi_\mu^\lambda p_\mu , 
\]
where $z_\mu = \# C_{S_n}(\sigma_\mu)$, and 
$\chi_\mu^\lambda$ is the value of $\chi^\lambda$ at $c_\mu$. A trivial 
corollary is that 
\[
  \chi^\lambda = \sum_\mu \frac{1}{z_\mu} \chi_\mu^\lambda 1_{c\mu} .
\]
So $\{\chi^\lambda\}$ and $\{1_{c_\mu}\}$ are both $\dC$-bases for 
$\hom(S_n^\natural,\dC)$. 

We would like to do something similar for $\generallinear_n(\dF_q)$. 


\subsection{Hall-Littlewood symmetric functions}

Let $t$ be an extra variable. Consider the ring 
$\Lambda_{\dZ[t]} = \Lambda[t] = \Lambda\otimes \dZ[t]$. For $\lambda\in \cP$ 
with $\ell(\lambda)\leqslant r$, put 
\begin{align*}
  R_\lambda(x_1,\dots,x_r;t) 
    &= \sum_{w\in S_r} w\left(x_1^{\lambda_1} \dotsm x_r^{\lambda_r} \prod_{i<j} \frac{x_i - t x_j}{x_i - x_j}\right) \\
    &= \frac{1}{a_{\delta_r}} \sum_{w\in S_r} \varepsilon(w) w\left(x_1^{\lambda_1} \dotsm x_r^{\lambda_r} \prod_{i<j} x_i - t x_j\right) .
\end{align*}
These are elements of $\Lambda_r[t]$. Unfortunately, 
$R_\lambda(x_1,\dots,x_r,0;t)\ne R_\lambda(x_1,\dots,x_r;t)$ as elements of 
$\Lambda_{r+1}[t]$, so we need to ``fudge things'' a bit to get a well-defined 
element of $\Lambda[t]$. For $m\geqslant 1$, put 
\[
  v_m(t) = \sum_{w\in S_m} w\left(\prod_{i<j} \frac{x_i-t x_j}{x_i - x_j}\right) .
\]
For $\lambda=(\lambda_1,\dots,\lambda_r,0,\dots)$, denote by $m_i$ the multiplicity 
of $i$ in $\lambda$ if $i\geqslant 1$, and by 
$m_0 = \#\{i\leqslant r:\lambda_i = 0\}$. Note that $m_0$ depends heavily on $r$. 
Define 
\[
  v_{\lambda,r}(t) = \prod_{i\geqslant 0} v_{m_i}(t) .
\]
Then 
\[
  R_\lambda(x_1,\dots,x_r;t) = v_\lambda(t) \sum_{w\in S_r/S_r^\lambda} w\left(x_1^{\lambda_1} \dots x_r^{\lambda_r} \prod_{\lambda_i>\lambda_j} \frac{x_i-t x_j}{x_i-x_j}\right) ,
\]
where as before 
$S_r^\lambda = \{w\in S_r:\lambda_{w(i)} = \lambda_i\text{ for }1\leqslant i\leqslant r\} \simeq \prod_{i\geqslant 0} S_{m_i}$. Define 
\[
  P_\lambda(x_1,\dots,x_r;t) = \frac{1}{v_\lambda(t)} R_\lambda(x_1,\dots,x_r;t) ; 
\]
this is the \emph{Hall-Littlewood symmetric function}. 

\begin{proposition}
1. $P_\lambda(x_1,\dots,x_r;0) = S_\lambda(x_1,\dots,x_r)$. 

2. $P_\lambda(x_1,\dots,x_r;1) = m_\lambda(x_1,\dots,x_r)$. 

3. $P_\lambda(x_1,\dots,x_r,0;t) = P_\lambda(x_1,\dots,x_r;t)$. 
\end{proposition}

By 3, we have a well-defined element $P_\lambda\in \Lambda[t]$. Moreover, 
$\{P_\lambda:\lambda\in \cP\}$ is a $\dZ[t]$-basis for $\Lambda[t]$. 

The \emph{Kostka-Foulkes polynomials} $\{K_{\lambda\mu}(t)\}$ are by 
the relation 
\[
  S_\lambda = \sum_\mu K_{\lambda\mu}(t) P_\mu .
\]

\begin{theorem}[Lascoux-Sch\"utzenberger, Lusztig]
$K_{\lambda\mu}(t)\in \dZ_{\geqslant 0}[t]$. 
\end{theorem}





\section{The character table of \texorpdfstring{$\generallinear_n(\dF_q)$}{GLnFq}}


\subsection{The conjugacy classes}

Fix $n\geqslant 1$. Let $M=\overline{\dF_q}^\times$, considered as an abstract 
group. Let 
\[
  \widetilde\dT_n = \left\{\varphi:M\to \cP:\|\varphi\| := \sum_{m\in M} |\varphi(m)| = n\right\} . 
\]
We claim that $\generallinear_n(\overline{\dF_q})^\natural \isomorphism \widetilde\dT_n$. 
We illustrate this with an example. If $n=3$, then the conjugacy class of 
\[
  \begin{pmatrix} \alpha & 1 \\ & \alpha \\ & & \beta \\ & & & \beta \end{pmatrix}
\]
is sent to the map $M\to \cP$ given by 
\[
  \gamma\mapsto 
  \begin{cases}
    (2,0,0,\dots) & \gamma = \alpha \\ 
    (1,1,0,0,\dots) & \gamma = \beta \\
    0 & \text{otherwise} 
  \end{cases}
\]
For the general case, let $F:\overline{\dF_q} \to \overline{\dF_q}$ be the Frobenius 
$x\mapsto x^q$, and let $\Phi=\{\text{$F$-orbits of $M$}\}$. Every $F$-orbit is 
of the form $\{x,x^q,\dots,x^{q^{d-1}}\}$, for the minimal $d$ with 
$x^{q^d} = x$. If $f\in \Phi$, put 
$d(f) = \# f$ (cardinality of $f$ as an orbit set). 

\begin{proposition}
Put $\dT_n = \{\varphi:\Phi\to \cP:\|\varphi\| = \sum_{f\in \Phi} d(f)|\varphi(f)| = n\}$. 
Then $\generallinear_n(\dF_q)^\natural \isomorphism \dT_n$. 
\end{proposition}
\begin{proof}
Write $\generallinear_n = \generallinear_n(\overline{\dF_q})$, and 
let $F:\generallinear_n\to \generallinear_n$ be the Frobenius 
$(a_{i j})\mapsto (a_{i j}^q)$. We will show that 
\[
  \{\text{$F$-stable conjugacy classes of $\generallinear_n$}\}\isomorphism \dT_n .
\]
Indeed, in any conjugacy class $C$, there is an element of the form 
\[
  \begin{pmatrix} \Delta_1 \\ & \Delta_2 \\ & & \ddots \end{pmatrix} ,
\]
where 
\[
  \Delta_i = \begin{pmatrix} \alpha_i & 1 \\ & \alpha_i & 1 \\ \end{pmatrix}
\]
If $F(C)=C$, then $\Delta$ and $F(\Delta)$ are $\generallinear_n$-conjugate. Thus 
$F$ permutes the blocks $\Delta_i$, so $F(\alpha_i) = \alpha_j$ for some $j$. Thus 
$\Delta_i$ and $\Delta_j$ have the same Jordan form. 
\end{proof}

We claim that 
\[
  \generallinear_n(\dF_q)^\natural\isomorphism \{\text{$F$-stable conjugacy classes of $\generallinear_n$}\} . 
\]
Note that this fails for general algebraic groups. In general, suppose $G$ 
is a connected affine algebraic group acting on an algebraic variety $X$ (all 
over $\overline{\dF_q}$). Assume $G$, $X$, and the action of $G$ on $X$ are all 
defined over $\dF_q$. Let $F:G\to G$ and $F:X\to X$ be the relative Frobenii. 
Since the action of $G$ on $X$ is defined over $\dF_q$, we have 
$F(g\cdot x) = F(g) \cdot F(x)$. 
\begin{enumerate}
  \item If $O$ is an $F$-stable $G$-orbit of $X$, then 
    $O^F=\{x\in O:F(x)=x\}=O(\dF_q)$ is not empty (this uses the fact that 
    $g\mapsto g^{-1} F(g)$, $G\to G$ is surjective). Thus the above map is 
    surjective for any $G$. 
  \item If $x\in X^F=X(\dF_q)$ and $\stabilizer_G(x) = \stabilizer_G(x)^\circ$, 
    then $(G\cdot x)^F = G^F\cdot x$. 
\end{enumerate}

In $\generallinear_n$, the stabilizers $\stabilizer_{\generallinear_n}(x)$ are 
\emph{always} connected ($x\in \generallinear_n$). 

\begin{exercise}
Prove that $\stabilizer_{\speciallinear_2}\smat{1}{1}{}{1}$ has two connected 
components if $2\nmid q$. 
\end{exercise}

Each $g\in \generallinear_n(\dF_q)$ defines an $\dF_q[t]$-structure on 
$V=(\dF_q)^n$ as follows: $t\cdot v = g(v)$ for all $v\in V$. This structure 
depends only on the conjugacy class of $g$. If 
$C\subset \generallinear_n(\dF_q)$ is a conjugacy class corresponding to 
$\varphi:\Phi \to \cP$, denote by $V_\varphi$ the corresponding 
$\dF_q[t]$-structure on $V$. From commutative algebra, we know that 
$V_\varphi \simeq \bigoplus_{f\in \Phi} \bigoplus_{i\geqslant 1} \dF_q[t] / (P_f)^{\varphi(f)_i}$, and 
$P_f = \prod_{i=1}^{d-1} (t-x^{q^i})$. 

\begin{example}[$\generallinear_2(\dF_q)$]
Consider the following table, in which $\varphi(f)=0$ unless it is defined to 
be otherwise:
\begin{center}
\begin{tabular}{c|ccc}
conjugacy class & $\varphi:\Phi\to \cP$ \\ \hline
$\smat{a}{}{}{a}$, $a\in \dF_q^\times$ & $\{a\}\mapsto (1,1,0,\ldots)$ \\
$\smat{a}{}{}{b}$, $a\ne b\in \dF_q^\times$ & $\{a\},\{b\}\mapsto (1,0,\dots)$ \\
$\smat{a}{1}{}{a}$, $a\in \dF_q^\times$ & $\{a\}\mapsto (2,0,\dots)$ \\
$\smat{x}{}{}{x^q}$, $x\in \dF_{q^2}\smallsetminus \dF_q^\times$ & $\{x,x^q\}\mapsto (1,0,\dots)$
\end{tabular}
\end{center}
\end{example}

\begin{example}[$\generallinear_3(\dF_q)$]
Conventions here are the same. [do this]
\end{example}





\subsection{Parabolic induction}

For any finite group $H$, write $\cC H=\dC^{H^\natural}$ for the space of 
functions $H\to \dC$ that are constant on conjugacy classes. 

Let $\lambda=(n_1,n_2,\dots,n_r,0,\dots)$ be a partition of $n$. 
Let $\generallinear_\lambda\subset \generallinear_n$ be the group consisting of 
matrices of the form $A_1\oplus \cdots \oplus A_r$, with each $A_i$ an 
invertible $n_i\times n_i$ matrix. Note that 
$\generallinear_\lambda\simeq \prod_i \generallinear_{n_i}$. If we write 
$L=\generallinear_\lambda$, we will define functions 
$R_L^{\generallinear_n}:\cC(L(\dF_q)) \to \cC(\generallinear_n(\dF_q))$. 

Let $P$ be the standard parabolic of ``block upper-triangular matrices'' 
associated to $\lambda$, and let $U_P$ be the unipotent radical of $P$. 
One has $P=\generallinear_\lambda \cdot U_P$. If 
$f\in \cC(L(\dF_q))$, define 
$R_L^{\generallinear_n}(f)=\induce_P^{\generallinear_n}(\widetilde f)$, 
where $\widetilde f:P(\dF_q) \to \dC$ is given by 
$\ell u\mapsto f(\ell)$ for $\ell\in L$, $u\in U_P$. If 
$f\in \dN[\irreducible(L(\dF_q))]$, then 
$R_L^{\generallinear_n}(f)\in \dN[\irreducible(\generallinear_n(\dF_q))]$. 
It turns out that $R_L^{\generallinear_n}(f)$ does not depend on the choice of 
a rational $P$ having $L$ as a Levi factor. 

Put $A_n = \cC(\generallinear_n(\dF_q)$ and $A=\bigoplus_{n\geqslant 0} A_n$. 
We define an inner product on $A$: if 
$f\in A_{n_1}$ and $g\in A_{n_2}$, put 
\[
  f\circ g = R_{\generallinear(n_1)\times \generallinear(n_2)}^{\generallinear(n_1+n_2)}(f,g) .
\]
This is commutative and associative.  

Let $\varphi:\Phi\to \cP$ have finite support. Denote by 
$\pi_\varphi\in \cC(\generallinear_{\|\varphi\|}(\dF_q))$ the characteristic 
function of the associated conjugacy class $c_\varphi$. Then 
$\{\pi_\varphi\}$ is a $\dC$-basis for $A$, and $\pi_0$ is the multiplicative 
unit of $A$. 

\begin{lemma}
If $\|\varphi\| = \sum \| \varphi_i\|$, then 
$\pi_{\varphi_1}\circ \pi_{\varphi_2} \circ \cdots \circ \pi_{\varphi_r}(c_\varphi)$ 
is the number of sequences $0=W^{(0)}\subset W^{(1)} \subset \cdots \subset W^{(r)}=V_\varphi$ 
of submodules of $V_\varphi$ such that $W^{(i)}/W^{(i-1)} \simeq V_{\varphi_i}$. 
\end{lemma}

Here $V_\varphi$ is the $\dF_q[t]$-structure on $(\dF_q)^n$ defined by 
$\varphi$. 





\subsection{The characteristic map}

For each $f\in \Phi$, let $\{x_{1,f},x_{2,f},\dots\}$ be infinitely many 
variables. Let $x=\{x_1,x_2,\dots\}$ be a set of infinitely many variables. 
Put $\Lambda_{\dQ(q)} = \Lambda(x)\otimes \dQ(q)$. If 
$u\in \Lambda_{\dQ(q)}$, we denote by 
$u(f)$ the corresponding function in 
$\Lambda_{\dQ(q)}(x_f)$. 

Put $B=\dC[e_n(f):n\geqslant 0, f\in \Phi]$. Then 
$B\subset \dQ(q)\otimes \bigotimes_{f\in \dQ} \Lambda(x_f)$. The Hall pairing on 
$\Lambda(x)$ makes the Schur functions an orthonormal basis. We grade $B$ by 
$\deg(x_i,f) = d(f)$. For $\lambda\in \cP$, $f\in \Phi$, put 
$\widetilde P_\lambda(f) = q^{-n(\lambda) d(f)} P_\lambda(f; q^{-d(f)})$. Recall 
that $n(\lambda) = \sum_{i\geqslant 1} (i-1) \lambda_i$. 
For $\varphi:\Phi \to \cP$ with finite support, we put 
\[
  \widetilde P_\varphi = \prod_{f\in \Phi} \widetilde P_{\varphi(f)} (f) \in B .
\]
Since the Hall-Littlewood symmetric functions form a $\dZ$-basis of 
$\Lambda(x)$, the $\widetilde P_\varphi$ form a $\dC$-basis of $B$. 

The \emph{characteristic map} $\characteristic:A\to B$, given by 
$\pi_\varphi\mapsto \widetilde P_\varphi$, is an (isometric) isomorphism of 
graded $\dC$-algebras. 


\subsection{Construction of the irreducible characters of \texorpdfstring{$\generallinear_n(\dF_q)$}{GLnFq}}

Let $M_n=\dF_{q^n}^\times$. Let $\widehat{M_n} = \hom(M_n,\dC^\times)$, and let 
$L=\varinjlim_n \widehat{M_n}$ and $L_n=\image(\widehat{M_n} \to L)$. The 
Frobenius $F$ acts on $L$; let $\Theta$ be the set of $F$-orbits in $L$. 
We want $\irreducible(\generallinear_n(\dF_q))$ to be in bijection with 
\[
  \left\{\psi:\Theta\to \cP\text{ such that }\|\psi\| := \sum_{\theta\in \Theta} d(\theta) \cdot |\psi(\theta)| = n\right\} .
\]
The set $\irreducible(\generallinear_n(\dF_q))$ is an orthonormal basis for 
$\langle \cdot,\cdot\rangle$. We look for an orthonormal basis of $B$. For 
$\chi\in \irreducible(\generallinear_n(\dF_q))$, we expect 
$\characteristic(\chi)$ to be ``related to'' Schur functions. 

For $f\in \Phi$ and $x\in f$, put 
\[
  \widetilde P_n(x) = \begin{cases} P_{n/d}(f) & d=d(f)\mid n \\ 0 & \text{otherwise} \end{cases}
\]
Then for each $\xi\in L$, define 
\[
  \widetilde P_n(\xi) = 
  \begin{cases}
    (-1)^{n-1} \sum_{x\in M_n} \langle x,\xi\rangle_n \widetilde P_n(x) & \xi\in L_n \\ 0 & \text{otherwise} 
  \end{cases}
\]
Note that $\widetilde P_n(\xi)$ depends only on the $F$-orbit of $\xi$. For 
$\theta\in \Theta$, put $\widetilde P_r(\theta) = \widetilde P_{rd}(\xi)$, for 
any $\xi\in \theta$, and $d=d(\theta)$. 

\begin{example}
If $\xi=1\in L$ and $n=2$, then 
\[
  \widetilde P_n(\xi) = - \sum_{\substack{f\in \Phi \\ d(f)=1}} \sum_i x_{i,f}^2 - 2 \sum_{\substack{f\in \Phi \\ d(f) = 2}} \sum_i x_{i,f} .
\]
Make the change of variables $\{x_{i,f}\} \to \{y_{i,\theta}\}$, such that 
$\widetilde P_r(\theta)$ are power-sums in $\{y_{i,\theta}\}$. 
\end{example}

For $\lambda\in \cP$, put 
$\widetilde P_\lambda(\theta) = P_{\lambda_1}(\theta) P_{\lambda_2}(\theta) \dotsm$ and 
$S_\lambda(\theta) = \sum_\mu \frac{1}{z_\mu} \chi_\mu^{-1} \widetilde P_\mu(\theta)$. 
For $\psi:\Theta \to \cP$ with finite support, put 
$S_\psi = \prod_{\theta\in \Theta} S_{\psi(\theta)}(\theta)$. 

\begin{proposition}
1. $\{S_\psi\}$ is an orthornormal $\dC$-basis of $B$. 

2. Assume that $|\lambda|=|\mu|$. Then 
\[
  S_\lambda(1) = \sum_\mu \widetilde K_{\lambda\mu}(q) \widetilde P_\mu(1) .
\]
Here we are using the definition 
$\widetilde K_{\lambda\mu}(q) = q^{n(\mu)} K_{\lambda\mu}(q^{-1})$. 
\end{proposition}


\begin{theorem}
The map $A\to B$ induces 
$\irreducible(\generallinear_n(\dF_q)) \to\{S_\psi:\|\psi\| = n\}$. 
\end{theorem}
\begin{proof}
If $\theta\in \Theta$ and $n\geqslant 0$, then $e_n(\theta)$ is the characteristic 
function of a character of $\generallinear_{nd(\theta)}(\dF_q)$.For 
$\psi:\Theta\to \cP$, the $S_\psi$ are polynomials in the $e_n(\theta)$, hence 
have coefficients in $\dZ$. This implies that 
$S_\psi=\characteristic(\text{virtual character})$, say $\sX^\psi$. One has 
$\langle S_\psi,S_\psi\rangle = \langle \sX^\psi,\sX^\psi\rangle$, so 
$\sX^\psi$ or $-\sX^\psi$ is irreducible. Thus $\sX^\psi(1)>0$. 
\end{proof}

To conclude, call ``unipotent character'' the irreducible characters 
$\sX^\psi$ with 
\[
  \psi\in \{\Theta\xrightarrow\psi\cP:\text{support of }\psi=1\} .
\]
Later on, we will give a geometric construction of the 
$\sX^\psi$. 





\section{\texorpdfstring{$\ell$}{l}-adic sheaves and perverse sheaves}

The $\widetilde K_{\lambda\mu}(q)$ can be realized as Khazdan-Lusztig polynomials 
using perverse sheaves. 


\subsection{Sheaves on topological spaces}

Let $X$ be a topological space. A \emph{presheaf} on $X$ is a (contravariant) 
functor $F:\open(X) \to \cC$, where $\cC$ is the category of groups, sets, 
rings, modules, \ldots. Morphisms of presheaves are natural transformations of 
functors. A presheaf is a \emph{sheaf} if for any family $\{U_i\}$ of open 
subsets of $X$, and any family $\{s_i\in F U_i\}$ such that 
$s_i|_{U_i\cap U_j} = s_j|_{U_i\cap U_j}$, there exists a unique 
$s\in F U$ such that $s|_{U_i} = s_i$. We often use script letters for sheaves, 
e.g.\ $\sF$, $\sG$, \ldots. If $\sF$ is a sheaf on $X$ and $x\in X$, write 
$\sF_x=\varinjlim_{U\ni x} \sF(U)$ for the \emph{stalk} of $\sF$ at $x$. 

[\ldots I know what sheaves are\ldots]


\subsection{\texorpdfstring{$\ell$}{l}-adic analogue}

If $X$ is a variety over $\dC$, we can give $X(\dC)$ the structure of a topological 
space using the topology on $\dC$. In characteristic $p$, a variety has no natural 
topology that gives a good sheaf theory. Fortunately, a very deep theory of 
Grothendieck gives us something called the ``\'etale topology,'' so we can talk 
about sheaves on $X_\etale$. 

We would like to use the \'etale topology to define spaces $\h^i(X_\etale,k)$ ($k$ a 
field of characteristic zero) for $X$ over $\dF_q$. Unfortunately, if we try to 
define $\h^i(X_\etale, k)$ in the naive manner, we get trivial cohomology. 

[\ldots Grothendieck's insight $\to\h^i(X_\etale, \dQ_\ell)$\ldots]


\subsection{\texorpdfstring{$\overline{\dQ_\ell}$}{barQl}-sheaves}

Let $k$ be an algebraically closed field, $X$ a $k$-variety. Let $\ell$ be 
a prime invertible in $k$. Start with a sheaf $\sF$ of 
$\dZ/\ell^n$-modules on $X_\etale$. We say that $\sF$ is \emph{constructible} 
if there is a finite partition $X=\bigsqcup X_\alpha$ of locally closed 
subsets such that $\sF|_{X_\alpha}$ is locally constant, and moreover 
the stalks $\sF_{\bar x}$ be finite for all geometric points $\bar x$ of $X$. 
Denote by $\sheaf_c(X_\etale, \dZ/\ell^n)$ the category of constructible 
$\dZ/\ell^n$-sheaves. 

If $k=\dC$, then there is a morphism of topoi $\varepsilon:X(\dC) \to X_\etale$, 
inducing an equivalence of categories 
$\varepsilon^\ast:\sheaf_c(X_\etale,\dZ/\ell^n)\isomorphism \sheaf_c(X(\dC),\dZ/\ell^n)$. 

A \emph{constructible sheaf of $\dZ_\ell$-modules} is a family 
$(\sF_n,f_{n+1}:\sF_{n+1} \to \sF_n)$ such that 
\begin{enumerate}
  \item $\sF_n\in \sheaf_c(X_\etale,\dZ/\ell^n)$ 
  \item $f_{n+1}$ induces an isomorphism $\sF_{n+1}\otimes \dZ/\ell^n \isomorphism \sF_n$ 
\end{enumerate}
Note that we could have just talked about sheaves of $\dZ_\ell$-modules, but 
this does not give us an interesting category. [pro-\'etale topology lets you 
avoid this] We define 
\[
  \hom_{\dZ_\ell}(\sF,\sG) = \varprojlim_n \hom(\sF_n,\sG_n) .
\]
This gives us a perfectly good category $\sheaf_c(X_\etale,\dZ_\ell)$. 

Next we define a category of constructible $\dQ_\ell$-sheaves. Its objects are 
constructible $\dZ_\ell$-sheaves, but morphisms are 
\[
  \hom_{\dQ_\ell}(\sF,\sG) = \hom_{\dZ_\ell}(\sF,\sG)\otimes \dQ .
\]
In other words, $\sheaf_c(X_\etale,\dQ_\ell)$ is the localization of 
$\sheaf_c(X_\etale,\dZ_\ell)$ at all morphisms of the form ``multiply 
$\ell$.'' 

If $E$ is a finite extension of $\dQ_\ell$, we can repeat the whole process for 
$E$ (start with $O_E/\fp^n$, take a projective limit, then tensor with $\dQ$). This 
gives us a category $\sheaf_c(X_\etale,E)$. If $E\subset E'$ are finite extensions 
of $\dQ_\ell$, we get a functor 
$\sheaf_c(X_\etale,E) \to \sheaf_c(X_\etale,E')$, written 
$\sF\mapsto \sF\otimes_E E'$. One has 
\[
  \hom_{E'}(\sF\otimes_E E',\sG\otimes_E E') = \hom(\sF,\sG)\otimes_E E' .
\]

Finally we construct the category of $\overline{\dQ_\ell}$-sheaves. Its 
objects consist of constructible $E$-sheaves for varying finite extensions $E$ of 
$\dQ_\ell$. If $\sF$ is an $E$-sheaf and $\sG$ is an $E'$-sheaf, choose a finite 
extension $F$ containing $E$ and $E'$, and define 
\[
  \hom_{\overline{\dQ_\ell}}(\sF,\sG) = \hom(\sF\otimes_E F,\sG\otimes_{E'} F)\otimes_F \overline{\dQ_\ell} .
\]
Our category $\sheaf_c(X_\etale,\overline{\dQ_\ell})$ should be thought of the 
analogue of the category of constructible $\dC$-sheaves on the space $X(\dC)$ for 
$X$ a complex variety. When there is no danger of confusion, write $\sheaf_c(X)$ 
instead of $\sheaf_c(X_\etale,\overline{\dQ_\ell})$. 

If $\sF\in \sheaf_c(X)$, then $\sF$ is a \emph{local system} (or a smooth sheaf) if 
each $\sF_n$ is locally constant. \emph{Warning}: $\sF$ is not necessarily globally 
locally constant, i.e.\ there may not exist a partition that simultaneously 
trivializes each $\sF_n$. If $\sF\in \sheaf_c(X)$ and $\bar x$ is a geometric point 
of $X$, then we define the stalk of $\sF$ at $\bar x$ by 
\[
  \sF_{\bar x} = (\varprojlim_n \sF_{n,\bar x})\otimes \overline{\dQ_\ell} .
\]
These are finite-dimensional $\overline{\dQ_\ell}$-vector spaces. 

As in the classical case, when can define a ``bounded derived category of 
constructible sheaves'' $\derived_c^b(X)$. Its objects are complexes 
$\sK^\bullet$ with $\sH^\bullet\sK^\bullet$ being constructible 
$\overline{\dQ_\ell}$-sheaves. 
For any morphism $f:X\to Y$, we have functors 
\begin{align*}
  f_\ast:\derived_c^b(X) &\to \derived_c^b(Y) \\
  f_! : \derived_c^b(X) &\to \derived_c^b(Y) \\
  f^\ast : \derived_c^b(Y) &\to \derived_c^b(X) \\
  f^!:\derived_c^b(Y) &\to \derived_c^b(X) .
\end{align*}
If $f$ is proper, then $f_\ast=f_!$, and if $f$ is an open immersion, then 
$f^\ast=f^!$. 


\subsection{Analogy with functions}

Let $f:E\to F$ be a map of finite sets. Let $k$ be a field, and write 
$k^E$, $k^F$ for the spaces of functions $E\to k$, $F\to k$. We have functions 
\begin{align*}
  f^\ast:k^F &\to k^E && h\mapsto h\circ f \\
  f_\ast:k^E &\to k^F && h\mapsto x\mapsto \sum_{y\in f^{-1}(x)} h(y)
\end{align*}
which should be thought of as analogies of our functors between derived 
categories. 


\subsection{Connection with cohomology}

Let $X$ be a variety over an algebraically closed field $k$. Let 
$p:X\to \{\mathrm{pt}\}$ be the unique morphism. For $\sK\in \derived_c^b(X)$, 
we define 
\begin{align*}
  \hh^i(X,\sK) &= \sH^i(p_\ast \sK^\bullet) \\
  \hh_c^i(X,\sK^\bullet) &= \sH^i(p_! \sK^\bullet) 
\end{align*}
In particular, for $\sK^\bullet = \cdots \to 0 \to \overline{\dQ_\ell} \to 0 \to \cdots$, 
we put 
\begin{align*}
  \h^i(X,\overline{\dQ_\ell}) &= \hh^i(X,\sK^\bullet) \\
  \h_c^i(X,\overline{\dQ_\ell}) &= \hh_c^i(X,\sK^\bullet) 
\end{align*}
More directly, one has 
\[
  \h_c^i(X,\overline{\dQ_\ell}) = \left(\varprojlim \h_c^i(X_\etale,\dZ/\ell^n)\right)\otimes \overline{\dQ_\ell} .
\]


\subsection{Verdier duality}

We have a functor $D_X:\derived_c^b(X) \to \derived_c^b(X)$ such that 
\begin{enumerate}
  \item $D_X^2=1$
  \item if $f:X\to Y$ is a morphism, then $D_Y f_! = f_\ast D_X$ and $f^! D_Y = D_X f^\ast$ 
  \item $\hh_c^{-i}(X,\sK^\bullet) = \hh^i(X,D_X \sK^\bullet)$
  \item if $\sE$ is a local system and $X$ is smooth, then 
    $D_X(\sE[\dim X]) = \sE^\vee[\dim X]$
\end{enumerate}
In particular, if $\sE\simeq \sE^\vee$, then 
$\h_c^i(X,\sE) \simeq \h^{2\dim X-i}(X,\sE)$. 


\subsection{Intersection cohomology}

Let $Y$ be an irreducible nonsingular locally closed subset of $X$. Goresky, 
MacPherson and Deligne defined, for any local system $\sE$ on $Y$, a complex 
$\intersectioncohomology(\overline Y,\sE)\in \derived_c^b(\overline Y)$ such 
that 
\[
  D_{\overline Y}(\intersectioncohomology(\overline Y,\sE)[\dim Y]) \simeq \intersectioncohomology(\overline Y,\sE^\vee)[\dim Y] .
\]
The complex $\sK=\intersectioncohomology(\overline Y,\sE)[\dim Y]$ is 
characterized by the following properties:
\begin{enumerate}
  \item $\sH^{-\dim Y} \sK^\bullet|_Y \simeq \sE$
  \item $\sH^i\sK^\bullet=0$ if $i<-\dim Y$
  \item $\dim(\support \sH^i\sK^\bullet)<-i$ for $i>-\dim Y$
  \item $\dim(\support\sH^i D \sK^\bullet) < -i$ if $i>-\dim Y$
\end{enumerate}

If $Y$ is smooth, then $\intersectioncohomology(Y,\sE) \simeq \sE$. 

\begin{exercise}
Assume $Z$ is a nonsingular open subset of $\overline Y$. If $\sL$ is a local 
system on $Z$ such that $\sL|_{Z\cap Y}\simeq \sE|_{Z\cap Y}$, then 
$\intersectioncohomology(\overline Y,\sE) \simeq \intersectioncohomology(\overline Y,\sL)$. 
\end{exercise}





\subsection{Perverse sheaves}

We define $\perverse(X)$ to be the full subcategory of 
$\sK^\bullet\in \derived_c^b(X)$ such that 
\begin{enumerate}
  \item $\dim(\support \sH^i\sK^\bullet) \leqslant -i$ 
  \item $\dim(\support \sH^i D \sK^\bullet)\leqslant -i$
\end{enumerate}
Clearly $\perverse(X)$ contains all $\intersectioncohomology(\overline Y,\sE)[\dim Y]$ 
extended by zero on $X\smallsetminus \overline Y$. 

\begin{theorem}[Beilinson, Bernstein, Deligne, Gabber]
Any simple perverse sheaf is of the form 
$\intersectioncohomology(\overline Y,\sE)[\dim Y]$ extended by zero on 
$X\smallsetminus \overline Y$. 
\end{theorem}

As in the classical case, $\perverse(X)$ is a semisimple abelian category in 
which all objects have finite length. The inclusion 
$\perverse(X)\hookrightarrow \derived_b^c(X)$ induces an equivalence between 
the derived category of $\perverse(X)$ and $\derived_b^c(X)$. 

\emph{Warning}: The functors $f_!,f^!,f_\ast,f^\ast$ do not always carry 
one category of perverse sheaves into another (i.e.\ they do not preserve 
``perversity''). 


\subsection{Stratifications}

We say that $X=\bigsqcup X_\alpha$ is a \emph{stratification} of $X$ if it is a 
finite partition of $X$ into equidimensional nonsingular locally closed subsets 
of $X$, such that if $X_\beta\cap \overline{X_\alpha}$, then 
$X_\beta\subset \overline{X_\alpha}$. 

\begin{proposition}
Let $f:X\to Y$ be a proper surjective map with $X$ irreducible (or just 
equidimensional), and let $X=\bigsqcup X_\alpha$ be a stratification of $X$. 
For $y\in Y$, put $f^{-1}(y)_\alpha = f^{-1}(y)\cap X_\alpha$. Assume that 
\[
  \dim\left\{y\in Y:\dim f^{-1}(y)_\alpha\geqslant \frac 1 2 (i-\codim X_\alpha)\right\} \leqslant \dim Y - i 
\]
for all $\alpha$ and $i$. Then $f_\ast:\perverse(X) \to \perverse(Y)$ is 
well-defined. 
\end{proposition}

\begin{corollary}
If $f:X\to Y$ is surjective, proper, and $X$ is nonsingular irreducible, then 
if for all $i$, 
\begin{equation}\label{eq:semi-small}
  \dim\left\{y\in Y:\dim f^{-1}(y) \geqslant i\right\} \leqslant \dim Y- 2 i
\end{equation}
the functor $f_\ast:\perverse(X) \to\perverse(Y)$ is well-defined. 
\end{corollary}

A surjective proper map satisfying \eqref{eq:semi-small} is called 
\emph{semi-small}. 

\begin{exercise}
Let $k$ be an algebraically closed field. Consider the varieties 
\begin{align*}
  \mathfrak{gl}_2(k) &= \text{affine space over }k\\
  W &= \{\text{nilpotent matrices}\} \\
  B &= \smat{\ast}{\ast}{}{\ast} 
\end{align*}
Let 
\[
  \widetilde W = \left\{(x,g B)\in \mathfrak{gl}_2 \times \generallinear_2/B:g^{-1} x g\in \smat{}{\ast}{}{}\right\} .
\]
Show that the map $\widetilde W \to W$, $(x,g B)\mapsto x$ is semi-small. 
\end{exercise}

\begin{theorem}[special case of the decomposition theorem]
If $f:X\to Y$ is a proper smooth map, then 
$f_\ast(\overline{\dQ_\ell})$ is semisimple. 
\end{theorem}

\begin{corollary}
If $f:X\to Y$ is semi-small, $X$ nonsingular irreducible, then 
\[
  f_\ast(\overline{\dQ_\ell}[\dim X]) \simeq \intersectioncohomology(Y,\overline{\dQ_\ell})[\dim Y]\oplus \bigoplus_{\substack{(Z,\xi) \\ \overline Z\subsetneq Y}} V_{Z,\xi} \otimes \intersectioncohomology(\overline Z,\xi)[\dim Z] 
\]
\end{corollary}

In terms of cohomology, this tells us that 
\[
  \h_c^i(X,\overline{\dQ_\ell}) \simeq \operatorname{IH}_c^i(\overline Y,\overline{\dQ_\ell}) \oplus \bigoplus_{(Z,\xi)} V_{Z,\xi} \otimes \operatorname{IH}_c^{i+\bullet}(\overline Z,\xi) .
\]


\subsection{\texorpdfstring{$F$}{F}-equivariance}

Suppose $X$ is defined over $\dF_q$ with (relative) Frobenius 
$F:X\to X$. We have a functor $F^\ast:\derived_c^b(X) \to \derived_c^b(X)$. 
We say that $\sK\in \derived_c^b(X)$ is \emph{$F$-stable} if 
$F^\ast \sK\simeq \sK$. An \emph{$F$-equivariant complex} on $X$ is a pair 
$(\sK,\phi)$ with $\phi:F^\ast \sK\isomorphism \sK$. Morphisms 
$\varphi:(\sK,\phi) \to (\sK',\phi')$ are defined via the commutative diagram 
\[\xymatrix{
  F^\ast \sK \ar[r]^-{F^\ast\varphi} \ar[d]^-\phi 
    & F^\ast \sK' \ar[d]^-{\phi'} \\
  \sK \ar[r]^-\varphi 
    & \sK' 
}\]
Write $\derived_c^b(X)_F$ for the category of $F$-equivariant complexes. 


\subsection{Characteristic function of \texorpdfstring{$(\sK,\phi)$}{(K,phi)}}

We define $\chi_{\sK,\phi}:X^F=X(\dF_q) \to \overline{\dQ_\ell}$ by 
\[
  x\mapsto \sum_i (-1)^i \trace\left(\phi_x^i, \sH_x^i\sK\right) .
\]

\begin{lemma}
If $\sK\simeq \sK'$ are simple perverse sheaves, and if 
$\phi:F^\ast\sK \isomorphism \sK$ and $\phi':F^\ast \sK'\isomorphism \sK'$, 
then there is a unique $c_{\phi\phi'}\in \overline{\dQ_\ell}$ such that 
$\chi_{\sK,\phi} = c_{\phi\phi'} \chi_{\sK',\phi'}$. If 
$c_{\phi\phi'}=1$, then $(\sK,\phi)\simeq (\sK',\phi')$. 
\end{lemma}

Any $\phi:F^\ast\sE\to \sE$, with $\sE$ a local system on $Y$ (an $F$-stable, 
nonsingular, locally closed subset of $X$) induces a canonical 
$\phi:F^\ast\intersectioncohomology(\overline Y,\sE)) \isomorphism \intersectioncohomology(\overline Y,\sE)$. 

\begin{proposition}
If $f:X\to Y$ commutes with Frobenius $F$, then 
$f_!:\derived_c^b(X)_F \to \derived_c^b(Y)_F$ and 
$f^\ast:\derived_c^b(Y)_F \to \derived_c^b(X)_F$. 
\end{proposition}
\begin{proof}
Let $(\sK,\phi)\in \derived_c^b(X)_F$. From 
$\phi:F^\ast\sK \isomorphism \sK$ we get 
$f_! F^\ast \sK \xrightarrow{f_!\phi} f_! \sK$. We apply the proper base change 
theorem to $f:X\to Y$ to get a canonical isomorphism 
$F^\ast f_! \simeq f_! F^\ast$. From all this we get 
$\widetilde\phi:F^\ast f_! \sK \isomorphism f_! \sK$. Our functor is 
$(\sK,\phi)\mapsto (f_! \sK,\widetilde\phi)$. 
\end{proof}

\begin{theorem}[trace formula]
Assume $f:X\to Y$ commutes with $F$. This gives $f:X^F \to Y^F$ between finite 
sets. Then 
\begin{enumerate}
  \item $f_\ast \chi_{\sK,\phi} = \chi_{f_!(\sK,\phi)}$, i.e.\ the following 
    diagram commutes:
    \[\xymatrix{
      \operatorname{Fun}(X^F,\overline{\dQ_\ell}) \ar[r]^-{f_!} 
        & \operatorname{Fun}(Y^F,\overline{\dQ_\ell}) \\
      \derived_c^b(X)_F \ar[r]^-{f_!} \ar[u]^-\chi 
        & \derived_c^b(Y)_F \ar[u]^-\chi 
    }\]
  \item The following diagram commutes:
    \[\xymatrix{
      \operatorname{Fun}(Y^F,\overline{\dQ_\ell}) \ar[r]^-{f^\ast} 
        & \operatorname{Fun}(X^F,\overline{\dQ_\ell}) \\
      \derived_c^b(Y)_F \ar[r]^-{f^\ast} \ar[u]^-\chi 
        & \derived_c^b(X)_F \ar[u]^-\chi 
    }\]
\end{enumerate}
\end{theorem}

The proof uses Grothendieck's trace formula. For any 
$(\sK,\phi)\in \derived_c^b(X)_F$, $\phi$ gives us an action of Frobenius on 
$\hh_c^i(X,\sK)$. The theorem is that 
\[
  \sum_{x\in X^F} \chi_{\sK,\phi}(x) = \sum_i (-1)^i \trace(F,\hh_c^i(X,\sK)) .
\]
In particular, if $\sK=\overline{\dQ_\ell}$ and 
$\phi:F^\ast \overline{\dQ_\ell} \to \overline{\dQ_\ell}$ induces the identity 
on stalks of $X^F$, then 
\[
  \# X^F = \sum (-1)^i \trace(F^\ast,\h_c^i(X,\overline{\dQ_\ell})) .
\]

The website of Alberto Arabia has some great notes on all of this. The 
following fixes an error. 

\begin{proposition}
Let $f:X\to Y$ be a surjective proper map with $X$ irreducible, and let 
$X=\bigsqcup X_\alpha$ be a stratification of $X$. For 
$y\in Y$, put $f^{-1}(y)_\alpha = X_\alpha\cap f^{-1}(y)$. Assume 
\[
  \dim\left\{y\in Y:\dim f^{-1}(y)_\alpha \geqslant \frac 1 2 (i-\codim X_\alpha)\right\} \leqslant \dim Y-i 
\]
for all $\alpha$ and $i$. Then 
$f_\ast(\intersectioncohomology(X,\sE)[\dim X])\in \perverse(Y)$ for 
$\sE$ a local system on $X_0$. 
\end{proposition}

Let $\sK\in \perverse(X)$ obtained as $\sK=i_\ast \sK'$ for 
$\sK'\in \perverse(X')$, where $X'\subset X$ is closed. [\ldots stuff 
I didn't understand\ldots] 

\begin{example}
Let $G=\generallinear_2(\dC)$, $B\subset G$ the standard Borel, and 
$U\subset B$ the unipotent radical. Let $N$ be the ($2$-dimensional) variety of 
nilpotent matrices in $\mathfrak{gl}_2$. Let 
$\widetilde W = \{(x,g B)\in \mathfrak{gl}_2\times G/B:g^{-1} x g\in \lie(U)\}$. 
There is a map $\widetilde W\to W$, called the Springer resolution. It fits into 
a cartesian diagram 
\[\xymatrix{
  \dP^1\simeq G/N \ar[r]^-{\pi^0} \ar[d] 
    & \{0\} \ar[d] \\
  \widetilde W \ar[r]^\pi 
    & W
}\]
The complex $\overline{\dQ_\ell}[1]$ is a pervrse sheaf on $G/B$. The Springer 
resolution is semi-small. However, 
$\pi^0(\overline{\dQ_\ell}[1]) = \overline{\dQ_\ell}\oplus \overline{\dQ_\ell}[2]$, 
which is not perverse. On the other hand, 
$\overline{\dQ_\ell}[2]\in \perverse(\widetilde M)$, and 
$\pi_\ast \overline{\dQ_\ell}[2] = \intersectioncohomology(W,\overline{\dQ_\ell}) \oplus \intersectioncohomology(\{0\},\overline{\dQ_\ell})$. 
In other words, 
$\pi_\ast(\overline{\dQ_\ell}[2])|_{\{0\}} \simeq \pi_\ast^0(\overline{\dQ_\ell}[1])[1]$. 
\end{example}





\subsection{\texorpdfstring{$G$}{G}-equivariance}

Let $G$ be a connected linear algebraic group defined over $\dF_q$. Let $X$ be a 
variety defined over $\dF_q$. Assume we are given an action of $G$ on $X$, also 
defined over $\dF_q$. Let $\pi:G\times X\to X$ be the projection map, and let 
$\rho:G\times X\to X$ be the map encoding the action of $G$ on $X$. We say that 
a perverse sheaf $\sK\in \perverse(X)$ is \emph{$G$-equivariant} if 
$\pi^\ast \sK\simeq \rho^\ast \sK$. We would like to create a category 
$\perverse_G(X)$ of ``$G$-equivariant perverse sheaves'' on $X$. Let 
$\alpha:G\times G\to G$ be the multiplication morphism, let 
$p_2:G\times G\to G$ be projection onto the second coordinate, and let 
$i:X\to G\times X$ be the inclusion $x\mapsto (1,x)$.  

\begin{lemma}
Assume $\sK\in \perverse(X)$ is $G$-equivariant. Then there is a unique 
isomorphism $\phi:\pi^\ast \sK \isomorphism \rho^\ast \sK$ such that 
\begin{enumerate}
  \item $i^\ast \phi = 1_\sK$ 
  \item $(\alpha\times 1_X)^\ast \phi = (1_G\times \rho)^\ast \phi \circ (p_2\times 1_X)^\ast \phi$
\end{enumerate}
\end{lemma}

With this under our belt, we can define the category $\perverse_G(X)$ to 
be the subcategory of $\perverse(X)$ whose objects are $G$-equivariant 
perverse sheaves on $X$. Morphisms in $\perverse_G(X)$ are isomorphisms 
$\psi:\sK\to \sK'$ such that the following diagram commutes:
\[\xymatrix{
  \pi^\ast \sK \ar[r]^-{\pi^\ast \psi} \ar[d]^-{\phi_\sK} 
    & \pi^\ast \sK' \ar[d]^-{\phi_{\sK'}} \\
  \rho^\ast \sK \ar[r]^-{\rho^\ast \psi} 
    & \rho^\ast \sK' 
}\]

\begin{lemma}
The category $\perverse_G(X)$ is a full subcategory of $\perverse(X)$. 
\end{lemma}

\begin{proposition}
1. The duality functor $D_X$ induces an auto-equivalence 
$D_X:\perverse_G(X)\isomorphism \perverse_G(X)$. 

2. If $\sK\in \perverse_G(X)$, then so are all subquotients of $\sK$. 

3. The simple objects of $\perverse_G(X)$ are the 
$\intersectioncohomology(\overline Y,\sE)[\dim Y]$, for $\sE$ a $G$-equivariant 
local system on a $G$-stable closed subset $Z\subset Y$. 
\end{proposition}

If $\sK\in \perverse_G(X)$ is $F$-stable with $\phi:F^\ast \sK\isomorphism \sK$, 
then the characteristic function $\chi_{\sK,\phi}:X(\dF_q) \to \overline{\dQ_\ell}$ 
is constant on $G(\dF_q)$-orbits in $X(\dF_q)$. We will use this fact to construct 
two natural bases for $\cC_{G^F}(X^F)$, the space of $G^F$-equivariant functions 
$X^F \to \overline{\dQ_\ell}$. Consider pairs 
$(O,\xi)$, where $O$ is a $G$-orbit in $X$, and $\xi$ is an irreducible $G$-equivariant 
local system on $O$. We say that $(O,\xi)\simeq (O',\xi')$ if 
$O=O'$ and $\xi\simeq \xi'$. Let $I$ be the set of isomorphism classes of 
$F$-stable pairs $(O,\xi)$. For each $i\in I$, choose a representative 
$(O_i,\xi_i)$, together with an isomorphism 
$\phi_i:F^\ast \xi_i \to \xi_i$.  

For each $i\in I$, define a functions $X_i,Y_i:X^F \to \overline{\dQ_\ell}$ by 
$x\mapsto \chi_{\intersectioncohomology(\overline{O_i},\xi_i),\phi_i}$ and
$x\mapsto \trace(\phi_i,(\xi_i)_x)$. Then 
$\{X_i\}$ and $\{Y_i\}$ are $\overline{\dQ_\ell}$-bases of $\cC_{G^F}(X^F)$. 
If $\stabilizer_G(O)=\stabilizer_G(O)^\circ$, then there is only one irreducible 
$G$-equivariant local system (up to isomorphism) on $O$ -- namely 
$\overline{\dQ_\ell}$. In that case, we can choose $\phi_i$ so that 
$\phi_i:F^\ast \overline{\dQ_\ell}\to \overline{\dQ_\ell}$ induces the identity 
on stalks. In that case, $Y_i=1_{O_i}$. In general, 
$O_i^F$ is a disjoint union of $G^F$-orbits. 





\section{Geometric realization of the unipotent characters of \texorpdfstring{$\generallinear_n(\dF_q)$}{GLnFq}}


\subsection{Parabolic induction}

Throughout, everything lives over $\overline{\dF_q}$. 

First we do this with functions. Let $L\subset \generallinear_n$ be a 
standard Levi associated to a partition $\lambda\vdash n$. Let $P$ 
be the associated parabolic; note that $P=L\ltimes U_P$, where 
$U$ is the unipotent radical of $P$. Let 
$F:\generallinear_n \to \generallinear_n$ be the Frobenius map (over 
$\dF_q$). Then $\generallinear_n^F = \generallinear_n(\dF_q)$, and 
$\cC(\generallinear_n^F)$ consists of functions 
$\generallinear_n(\dF_q) \to \overline{\dQ_\ell}$ that are constant 
on conjugacy classes. We define 
$R_L^{\generallinear_n}:\cC(L^F) \to \cC(\generallinear_n^F)$ via 
the following. Let $\pi_P:P \to L$ be the projection 
$\ell u\mapsto u$. We define 
$R_L^{\generallinear_n}$ to be the composite 
\[\xymatrix{
  \cC(L^F) \ar[r]^-{\pi_P^\ast} 
    & \cC(P^F) \ar[r]^-{\induce_{P^F}^{\generallinear_n^F}} 
    & \cC(\generallinear_n^F) .
}\]
We also define 
\begin{align*}
  V_1 &= \{(x,g)\in \generallinear_n^2:g^{-1} x g\in P\} \\
  V_2 &= \{(x,g P)\in \generallinear_n\times \generallinear_n / P:g^{-1} x g\in P\} 
\end{align*}
We define maps 
\begin{align*}
  \pi:V_1 &\to L && (x,g)\mapsto \pi_P(g^{-1} x g) \\
  \rho_1:V_1 &\to V_2 && (x,g) \mapsto (x, g P) \\
  \rho_2:V_2 &\to \generallinear_n{} && (x, g P)\mapsto x .
\end{align*}
Note that $R_L^{\generallinear_n} = (\rho_2)+\ast \widetilde \rho_1 \pi^\ast$, 
where 
$\widetilde \rho_1:\cC_{\generallinear_n\times P}(V_1^F) \to \cC_{\generallinear_n}(V_2^F)$ 
sends $f$ to the unique map $h$ on $V_2^F$ such that 
$\rho_1^\ast h = f$. 

Now we repeat the above, but with perverse sheaves instead of functions. 
We will define a functor 
$R_L^{\generallinear_n}:\perverse_L(L) \to \derived_c^b(\generallinear_n)$. 
First, note that our morphism $\pi$ is smooth with connected fibers of dimension 
$m=\dim \generallinear_n + \dim P$. Thus 
$\pi^\ast[m]:\perverse_L(L) \to \perverse_{\generallinear(n)\times P}(V_1)$ is 
well-defined, where $\generallinear(n)$ acts on $V_1$ by 
$g\cdot (x,h) = (g x, g h)$. 

The map $\rho_1$ is a ocally trivial fibration for the Zariski topology. Thus 
$\rho_1^\ast[\dim P]:\perverse_{\generallinear(n)}(V_2) \isomorphism \perverse_{\generallinear(n)\times P}(V_1)$. It follows that for each 
$\sK\in \perverse_L(L)$, there is a unique 
$\widetilde\sK\in \perverse_{\generallinear(n)}(V_2)$ such that 
$\rho_1^\ast[\dim P]\widetilde\sK \simeq \pi^\ast[m]\sK$. Define 
$R_L^{\generallinear(n)}(\sK) = (\rho_2)_\ast \widetilde\sK$; note that this is 
``exactly the same'' as our definition of $R_L^{\generallinear(n)}$ for 
functions. Again, note that $R_L^{\generallinear(n)}\sK$ is \emph{not} 
generally perverse, even when $\sK$ is. 

For simplicity, write $G=\generallinear(n)_{\overline{\dF_q}}$. 

If $\phi:F^\ast \sK\isomorphism \sK$, then $R_L^G$ induces a 
canonical $\widetilde\phi:F^\ast R_L^G\sK \isomorphism R_L^G \sK$. 
So the ``Harish-Chandra induction'' $R_L^G:\cC(L^F) \to \cC(G^F)$ has 
an analogue $R_L^G:\perverse_L(L)_F \to \derived_c^b(G)_F$. If we 
let $\chi$ denote ``take characteristic function'' then the following 
diagram commutes:
\[\xymatrix{
  \perverse_L(L)_F \ar[r]^-{R_L^G} \ar[d]^-\chi 
    & \derived_c^b(G)_F \ar[d]^-\chi \\
  \cC(L^F) \ar[r]^-{R_L^G} 
    & \cC(G^F) 
}\]


\subsection{Characters of \texorpdfstring{$\generallinear_n(\dF_q)$}{GLnFq}}

Let $R_{(1^n)}=\sum_{\lambda\in \cP_n} \chi_{(1^n)}^{\lambda'} \sX^\lambda = R_L^G(\textnormal{id})$. Here $\sX^\lambda$ is the unipotent character of 
$\generallinear_n(\dF_q)$ associated to $\lambda$, and $\lambda'$ is the 
dual partition of $\lambda$. For $\mu\in \cP_n$, let 
$R_\mu = \sum_{\lambda\in \cP_n} \chi_\mu^{\lambda'} \sX^\lambda = \operatorname{Ch}^{-1}(P_\mu(\mathrm{id}))$. We can invert things to give 
$\sX^\lambda = \sum_\mu \frac{1}{z_\mu} \chi_\mu^{\lambda'} R_\mu$. Recall that 
$\{R_\mu\}$ and $\{\sX^\lambda\}$ are both bases of the same subspace of 
$\cC(G^F)$. 


\subsection{Induction picture for \texorpdfstring{$L=T$}{L=T}}

Let $T\subset G$ be the standard maximal torus. We have maps 
\[
  T \leftarrow \{(g,h B)\in G\times G/B:h^{-1} g h \in B\} \to G .
\]
Note that 
$R_L^G(\overline{\dQ_\ell}[\dim T]) = (\rho_2)_\ast \overline{\dQ_\ell}[\dim G]$. 

\begin{theorem}
The map $\rho_2$ is semi-small (small, in fact). Moreover 
$(\rho_2)_\ast (\overline{\dQ_\ell}[\dim G])$ is a semisimple $G$-equivariant 
perverse sheaf on $G$. 
\end{theorem}

In fact, we can prove that 
$(\rho_2)_\ast (\overline{\dQ_\ell}[\dim G]) \simeq \intersectioncohomology(G,\xi)[\dim G]$. 
We need to describe $\xi$. Note that $g\in G$ is semisimple regular if and only if 
$C_G(g)$ is conjugate to $T$. Consider $G_\regular$, the subset of $G$ consisting 
of semisimple regular elements. This is a nonsingular irreducible Zariski-open 
subset of $G$. Let 
\[
  X_2 = \{(g, h T)\in G\times G/T:h^{-1} g h\in T_\regular\} .
\]
We have the following diagram: 
\begin{equation}\label{eq:perv}
\xymatrix{
  V_2 \ar[r] 
    & G \\
  X_2 \ar[r]^-\alpha \ar[u] 
    & G_\regular \ar[u] .
}
\end{equation}
Recall that $S_n=N_G(T)/T$, and this acts on $X_2$ by 
$w\cdot (g, h T) = (g, h w^{-1} T)$. The map $\alpha$ is a Galois covering 
with Galois group $S_n$. One has 
\[
  \alpha_\ast \overline{\dQ_\ell} = \bigoplus_{\chi\in \irreducible(S_n)}\sL_\chi^{\chi(1)}
\]
It turns out that the diagram \eqref{eq:perv} is Cartesian. The base-change 
theorem tells us that 
$\rho_2(\overline{\dQ_\ell}[\dim G])|_{G_\regular} \simeq \alpha_\ast (\overline{\dQ_\ell}[\dim G])$. We know that 
$\rho_2(\overline{\dQ_\ell}[\dim G]) \simeq \intersectioncohomology(G,\xi)[\dim G]$. 
We can take $\xi$ to be $\alpha_\ast \overline{\dQ_\ell}$ on 
$G_\regular$. As a result, we get 
\[
  (\rho_2)_\ast (\overline{\dQ_\ell}[\dim G]) \simeq \bigoplus_{\chi\in \irreducible(S_n)} V_\chi \otimes \intersectioncohomology(G,\sL_\chi)[\dim G] .
\]
Beause $S_n$ acts on $X-2$, we can define an action of $S_n$ on 
$\alpha_\ast \overline{\dQ_\ell}$. For $w\in S_n$, we get 
$\theta_w:R_T^G(\overline{\dQ_\ell}) \isomorphism R_T^G(\overline{\dQ_\ell})$ 
and $F^\ast R_T^G\overline{\dQ_\ell} \xrightarrow{\widetilde\phi} R_T^G\overline{\dQ_\ell} \rightarrow{\theta_w} R_T^G\overline{\dQ_\ell}$. 

\begin{theorem}[Lusztig, 1998]
If $w\in S_n$ is of cycle type and $\mu\in \cP_n$, then 
\[
  (-1)^\ast X_{R_T^G(\overline{\dQ_\ell}), \theta_w\widetilde\phi} = R_\mu .
\]
\end{theorem}

\begin{theorem}[Lusztig, 1981]
1. $\intersectioncohomology(G, \sL_{X^\lambda})|_{G_\regular} \simeq \intersectioncohomology(\overline{C_\lambda}, \overline{\dQ_\ell})$. 

2. $q^{n(\lambda)} X_{\intersectioncohomology(\overline{C_\lambda}, \overline{\dQ_\ell})}(c_\mu) = \widetilde K_{\lambda\mu}(q)$. 

3. $X_{\intersectioncohomology(\overline{C_\chi},\overline{\dQ_\ell})}(c_\mu) = \sum_i \dim \sH_{c_\mu}^{2 i} \intersectioncohomology(\overline{C_\lambda}) q^i$. 
\end{theorem}

From 2 and 3, we get that $\widetilde K_{\lambda\mu}\in \dN[q]$. 

[here I write $X_?$ instead of $\chi_?$ for the character associated to a sheaf 
$?$]





\bibliographystyle{alpha}
\bibliography{lyon-sources}

\end{document}
