\documentclass{article}

\usepackage{lyon-style}

\title{Kazhdan-Lusztig theory}
\author{Peng Shan}
\date{June 2-6, 2014}

\begin{document}
\maketitle





\section{Hecke algebras and the Kazhdan-Lusztig basis}


\subsection{Coxeter systems}

Recall that a \emph{Coxeter system} is a pair $(W,S)$, where $W$ is a group 
generated by $S\subset W$, via the presentation 
\[
  W = \langle s\in S:s^2 = 1, (s t)^{m_{s t}} = 1\rangle ,
\]
where $m_{s t}\in \{2,3,\ldots,\infty\}$. Call $\# S$ the \emph{rank} of 
$(W,S)$. 

\begin{example}
The symmetric group $W=S_n$ is a Coxeter group, generated by 
$S=\{s_i=(i, i+1):1\leqslant i<n\}$, with 
\begin{align*}
  m_{s_i,s_j} &= 2 && |i-j|>1 \\
  m_{s_i,s_{i+1}} &= 3 
\end{align*}
\end{example}

\begin{example}
If $V$ is a Euclidean vector space and 
$\Gamma\subset \orthogonal(V)$ is generated by finitely many reflections, then 
$\Gamma$ is a Coxeter group. 
\end{example}

\begin{example}
If $\dim_\dR(V)=2$, then $\Gamma=\langle s,t:s^2=t^2=(s t)^{m_{s t}} = 1\rangle$, 
the Dihedral group, is a Coxeter group. 
\end{example}

These examples have been finite, but there are infinite Coxeter groups as well. 

\begin{example}[affine reflection group]
Let $\widetilde S_2=\langle s,t:s^2=t^2=1\rangle$; this can be realized as the 
group of affine transformations of $\dR$ generated by the reflections about 
$0$ and $1$. As a group, $\widetilde S_2\simeq \dZ/2\ast \dZ/2$. 
\end{example}

If $(W,S)$ is a Coxeter group and $w\in W$,an \emph{expression} of $w$ is a tuple 
$\underline w = (s_1,\dots,s_n)$, with $w=s_1\dotsm s_n$ and $s_i\in S$. A 
\emph{reduced expression} for $w$ is an expression $\underline w = (s_1,\dots,s_n)$ 
with $n$ minimal. Put $n=\ell(w)$, and call this number the \emph{length} of 
$w$. The subset of \emph{reflections} in $W$ is 
$T=\bigcup_{w\in W} w S w^{-1}$. The group $W$ has a canonical order, called the 
\emph{Bruhat order}. It is the partial order generated by 
$x t<x$ for $t\in T$, so long as 
$\ell(x t) < \ell(x)$. 

Since $x t=(x t x^{-1}) x$, and $x t x^{-1}\in T$, we could have put $t$ to the 
left of $x$ in the definition of the Bruhat order. Alternatively, 
$y\leqslant x$ if and only if we could find a reduced expression 
$x=s_1\dotsm s_m$ and indices $j_1<\dots<j_r$, such that 
$y=s_{j_1} \dotsm s_{j_r}$. 

\begin{example}
Write $S_3=\langle s,t\rangle$, where $s=s_1$ and $t=s_2$. The Bruhat order is 
the following poset:
\[\xymatrix@=.2cm{
  & s t s \ar@{-}[dr] \ar@{-}[dl] \\
  s t \ar@{->}[d] \ar@{-}[drr] & & t s \ar@{-}[dll] \ar@{-}[d] \\
  s \ar@{-}[dr] & & t \ar@{-}[dl] \\
  & 1
}\]
\end{example}


\subsection{The Hecke algebra}

Let $v$ be a formal variable. We will work over the ring 
$\dZ[v^{\pm 1}]$. 

\begin{definition}
Let $(W,S)$ be a Coxeter system. The associated \emph{Hecke algebra} 
$\cH=\cH(W,S)$ is the unital $\dZ[v^{\pm 1}]$-algebra generated by variables 
$H_s,s\in S$, subject to the relations 
\begin{align*}
  \overbrace{H_t H_s \dotsm }^{m_{t s}} &= \overbrace{H_s H_t \dotsm}^{m_{s t}} \\
  H_s^2 &= (v^{-1} - v) H_s + 1
\end{align*}
\end{definition}
For any $x\in W$, choose a reduced expression 
$x=s_1\dotsm s_m$ and put $H_x = H_{s_1} \dotsm H_{s_m}$. 

\begin{theorem}[Tits]
1. $H_x$ does not depend on the choice of reduced expression for $x$. 

2. $\cH(W,S)=\bigoplus_{x\in W} \dZ[v^{\pm 1}] H_x$. 
\end{theorem}

So $\cH$ is a flat deformation of the group algebra $\dZ[W]$. As motivation 
do the following exercise. 

\begin{exercise}
Let $G=\generallinear_n(\dF_q)$, and let $B$ be the standard Borel subgroup of 
upper triangular matrices with coefficients in $\dF_q$. Define 
$\cH(G,B)$ to be the convolution algebra of bi-invariant $\dC$-valued functions 
on $G$. So multiplication is 
\[
  (f\star g)(z) = \frac{1}{\# B} \sum_{x\in G} f(x) g(x^{-1} z) .
\]
Show that $\cH(G,B) \simeq \cH(S_n)|_{v=q^{-1/2}}$. \emph{Hint}: use the 
Bruhat decomposition $G=\coprod_{w\in W} B w B$. Send the characteristic function 
$1_w$ to $v^{-\ell(w)} H_w$. 
\end{exercise}

Since it appears so often, we call $T_w=v^{-\ell(w)} H_w$. 


\subsection{THe Kazhdan-Lusztig basis}

In the algebra $\cH=\cH(W,S)$, we have $H_s^{-1} = H_s+(v-v^{-1})$. It follows 
that all $H_x$ are invertible. We can write 
$H_{x^{-1}}^{-1} = \sum_{y\leqslant x} r_{y,x} H_y$. This is an immediate 
consequence of the above. 

\begin{lemma}
There exists an involutive endomorphism $\cH\to \cH$, written 
$x\mapsto \overline x$, such that $\overline v = v^{-1}$ and 
$\overline{H_s} = H_s^{-1}$. One has $\overline{H_x} = H_{x^{-1}}^{-1}$. 
\end{lemma}

We call $h\in \cH$ \emph{self-dual} if $\bar h=h$. 

\begin{theorem}[Kazhdan-Lusztig '79]
There is a uniqu $\dZ[v^{\pm 1}]$-basis $\{\underline{H_x}:x\in W\}$ of 
$\cH$ such that 
\begin{enumerate}
  \item (self-duality): $\overline{\underline{H_x}} = \underline{H_x}$. 
  \item (Bruhat upper triangularity): $\underline Hx = H_x + \sum_{y<x} h_{y,x} H_y$, 
    for some $h_{y,x}\in v \dZ[v]$. 
\end{enumerate}
\end{theorem}
\begin{proof}
1. Uniqueness: assume $\underline H_x'$ satisfy the same condition. Then 
$d=\underline H_x - \underline H_x' = \sum_{y<x} g(y) H_y$ for some 
$g(y)\in v\dZ[v]$. If $0\ne d$, take $z$ a maximal element such that 
$g(z)\ne 0$. Then $\bar d=d$ implies 
\[
  \sum_y g(y) H_y = \sum_y \overline{g(y)} \cdot H_{y^{-1}}^{-1} = \sum_y \overline{g(y)} (H_y+\text{lower terms}) .
\]
The coefficient of $H_z$ is now given. We get $g(z)=\overline{g(z)}$, whence 
$g(z)\in v \dZ[v]$. It follows that $=0$, so $\underline H_x = \underline H_x'$. 

2. Existence: we use induction. If $x=e$, then $H_e = 1$. For $x=s$, we have 
$\underline H_s = H_s + v$. Now suppose we've defined $H_y$ for 
$y<x$. Choose $s\in S$ such that $s x<x$. Then 
\begin{align*}
  H_x H_s 
    &= \begin{cases} H_x & \ell(x s)>\ell(x) \\ H_{x s} + (v-v^{-1}) H_x & \ell(x s) < \ell(x) \end{cases} \\
  H_x \underline H_s &= 
  \begin{cases} H_{x s} + v H_x & \ell(x s)>\ell(x) \\ 
  H_{x s} + v^{-1} x & \ell(x s) < \ell(x) \end{cases} \\
  \underline H_{x s} \underline H_s &= (H_{x s} + \sum_{y<x s} h_{y, x s} H_y)\cdot \underline H_s = H_x + v H_{x s} + \sum_{y<x} g(y) H_y 
\end{align*}
where $g(y)\in \dZ[v]$. This allows us to define 
\[
  \underline H_x = \underline H_{x s} \underline H_s = \sum_{y<x} g_y(0) \underline H_y ,
\]
whenre $\underline H_x$ is self-dual. Finally, 
\[
  \underline H_x = H_x + v H_{x s} + \sum_{y<x} (g_y - g_y(0)) H_y .
\]
\end{proof}

We call such a basis the \emph{Kazhdan-Lusztig} (or canonical) basis of $\cH$. Up 
to normalization, we call the $h_{y,x}$ the \emph{Kazhdan-Lusztig polynomial}. 
Note that $g_y(0)\ne 0\Rightarrow y < y$. In that case, $g_y(0)$ is the coefficient 
of $v$ in $h y, x s$, generally written $\mu_{y,x,s}$. Thus if $x<x s$, we have
\[
  \underline H_{ s} \underline H_s = \underline H_x + \sum_{\substack{y<x s \\ y s<y}} \mu_{y, x} H_y .
\]
If $x s<x$, then $\underline H_x \underline H_s = (v+v^{-1}) \underline H_x$. 

So with respect to the basis $\{\underline H_x\}$, the $\underline H_x$ are 
(right) eigenvalues for the $H_s$. 
\[
  L^2(Z(k)\backslash \mathbf G,\omega)^\circ = \bigoplus \pi .
\]

[\ldots stopped following -- too formula-heavy]






\end{document}
